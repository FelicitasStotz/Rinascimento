\documentclass[10pt,a5paper]{book}
 \usepackage[ngerman]{babel}
 \usepackage[utf8]{inputenc}
 \usepackage{graphicx}
 
 \pagestyle{plain}
 
 %\frenchspacing
 
\parindent0em \parskip1.0ex plus0.5ex minus 0.5ex

%\topmargin0pt
\headsep0pt
\headheight0pt
\textheight14cm

\footskip1cm




 \author{von\\Johanna Mollydottir}
 \title{Rinascimento}
 
\newcommand{\sterne}{\par{\centering ***\par}}




 \begin{document}
 
 \maketitle
\tableofcontents
 
 
\section*{Epilog}
\addcontentsline{toc}{section}{Epilog}



„Eine Gabe ist eine Aufgabe.“ Käthe Kollwitz

Sind Sie auf die Welt gekommen, vergessend, was vorher war?

Sie leben,  atmen, erfahren, begreifen Ihr Dasein als eine Abfolge des sich fortfahrenden „Jetzt“, beginnend mit den ersten Erinnerungen Ihres Lebens- einzigartig, neu und einmalig?

Sie leben und erinnern sich nicht? Welche Gnade.

Bei mir ist es anders, dieses  mal.

Denn ich erinnere mich.

Meine Erinnerung ist gross, sie quillt hervor aus allen Ritzen meines Seins.

Glauben Sie an Reinkarnation?

Stellen Sie sich vor, Sie haben das eine oder andere Leben gelebt, \dots und würden sich daran erinnern. Sie erinnern sich ein Feldherr zu sein, \dots Sie riechen den Schweiss und das Blut, \dots spüren den Triumph des Sieges, den Rausch der Macht\dots, denken an die Freuden, die die Weiber bereiten, \dots und finden Sie sich wieder in dem Körper einer Frau, mittleren Alters, verheiratet, zwei Kinder, Auto, Haus und Hund\dots

Wer sind Sie?

Wie viel des einen Lebens gehört zu dem anderen?

Wieso leben Sie dieses?

Ach, seliges Vergessen.

Sie „glauben“ nicht an Reinkarnation?
 
Na, und?

„Glauben“ Sie an den Eiffelturm?

Warum? Haben Sie ihn selbst gesehen? Wenn nicht, woher „wissen“ Sie von seiner Existenz? Von Bildern? Erzählungen anderer?

Dabei ist es nicht schwer sich zu erinnern. Wir erinnern uns ständig, nur, wir wissen nicht, dass wir uns an ein vergangenes Leben erinnern, glauben, es seien Motivationen, Konsequenzen des jetzigen Lebens, aus denen heraus wir handeln.

Was, wären wir ein Puzzle. Ein Puzzle unserer vergangenen Leben, das sich neu fügt, neue Teile einfügt, die das Bild erweitern. Erweitern für das nächste Puzzleteil, bis es vollkommen und harmonisch ist, sich alle Teile zu einer Einheit fügen.

Wo beginnt die Geschichte, die Verschlingungen und Lösungen? Wo beginnt es, wenn die Vergangenheit vorbei ist, sich selbst aber wieder und wieder neu gebärt ins Jetzt. Verändert sich die Vergangenheit nicht mit jedem Gedanken, der sie neu beschwört? Sie verändert mich im gegenwärdtlichen Sein, wann immer ich ihr Raum gebe und sie verändert die Räume in mir, die sie betritt. In diesen Gedanken öffnet und verschliesst sie Türen. Türen, die die eine Zukunft öffnen oder schliessen. Es gibt keine Vergangenheit und die Zukunft entspringt immer neu der Kreation der Gedanken.

Die Flut und Ebbe, der Sturm und die ruhige Regung von Gedanken und Emotionen erschaffen uns sekündlich neu. Es gibt für den Körper Zeit und Raum, den Anker, in der orientierungslosen Flut der Gedanken. Alles ist Maya! Maya, die grosse Illusion. Trug- und Wahrbild. Theater, Theater der Seele!
Vorhang auf für Maya.



\part*{Teil I}
\addcontentsline{toc}{part}{Teil I}


\section*{Paradies I (Oldendorf, Dithmarschen 1975-1977)}
\addcontentsline{toc}{section}{Paradies I(Oldendorf, Dithmarschen 1975-1977)}



„Du bist die Frau und ich bin der Mann! Und nun machen wir noch ein Kind.“ „Okay!“ Johanna legt sich auf die Matratze. Sie ist aufgeregt.

Es ist heiss auf dem Dachboden. Die Sonne scheint in das Fenster und taucht den Giebel mit dem dunklen Holz ins Licht. Und die Hitze, die sich dort zur Mittagsruhe gelegt hatte, hüllt die Haut mit zwickenden Fingern ein.

Gleich bekomme ich ein Kind, denkt Johanna und ist besorgt, denn sie weiss nicht, wie das Kind in ihren Bauch passen wird.

Conny legt sich ganz vorsichtig auf sie. Die Bäuche berühren sich und zwischen den Beinen spürt Johanna das weiche Glied des Freundes ihre Vagina berühren. Nein, nein, mit Kindern will Johanna es nicht versuchen.

Beide springen auf, auseinander mit klopfendem Herzen, ist es zu spät? Aber Johannas Bauch ist wie zuvor. Schwitzend sehen sie sich an. Und die Hitze zwickt mit langen Fingern, die nackte Haut und sie spüren sich nackend, sie spüren, der Mann und die Frau, sie warten.

„Aber wir heiraten später, wenn wir gross sind!“ „Ja, und ich werde Müller und habe eine eigene Mühle auf dem Hof und einen Esel!“ „Mit der Mühle ist in Ordnung, aber ich weiss nicht, ob sich meine Kühe mit dem Esel vertragen?“ „Aber Müller brauchen einen Esel!“ „Na, gut.“

Johanna und Conny steigen auf ihre Fahrräder und fahren zu Connys Bauernhof. Der Hof liegt einsam eingebettet in den satt grünen, von Knicks umrandeten Koppeln.

Ein kleiner Hof mit dreissig Kühen, Kälbern und einigen Feldern ringsum. Einer Grossmutter, die kaum ihren Altenteil verlässt und griesgrämig die dünnen Finger aneinander  reibt. Sprechend dicht mit ihrem dünnen, scharfkantigen Gesicht heranrückt und spitze Bemerkungen ausstossend wie ein Huhn zurück ruckt, das Gesicht verzogen.

Die Bäuerin kräftig mit geröteten Wangen, dunklen, kurzen Locken, ist grösser als ihr Mann Klaus. Sie liebt es Schauerliches herauf zu beschwören und dadurch auf Dinge hin zu weisen, die es zu erproben gälte, um für das Leben vorbereitet zu sein.“Aua, aua, mein Zahn, die Kuh hat mich wieder getreten“, jammert sie. Vormelken muss sie, damit die Pfropfen der Melkmaschine die Milch aus den Eutern saugen können.

Bei den gutmütigen Kühen üben sich die Kinder, aber Johanna schafft es nicht, der rosigen, rauen, wackeligen Zitze einen Tropfen Milch zu entlocken. „Drücken und ziehen musst du.“ Conny lacht, während zwischen seinen Fingern die Milch aus dem Euter spritzt. Wie gut, dass ich ja Müller werde, denkt sich Johanna, der Conny muss seine Kühe selbst melken.


\section*{Tod I}
\addcontentsline{toc}{section}{Tod I}


Vor dem Bauern hat Johanna Angst, der spricht selten und poltrig, laut und weiss von Schlimmeren zu berichten als seine Frau. Er verengt seine wässrigen, hellblauen Augen zu Schlitzen und sein Gesicht ist feuerrot, umrahmt von gelben, zerzausten Haaren.

Er lässt die Kinder auf dem Trecker mitfahren. Hinten angehängt eine Maschine um Löcher für neue Zaunpfähle zu bohren. „Wenn du da, an diese Stange mit dem Kopf stösst, gehst du tot.“ Der Bauer schaut Johanna scharf an. Zitternden Leibes sitzt Johanna auf ihrem metallischen, Kissen gepolsterten Sitz, die Finger um die Stange gekrallt, die sie mit dem Tod bedroht.

Der Trecker ruckt und kippt, während er über die Koppel fährt und der Bohrer sich in die Erde dreht.

Da! Der erste Schlag an den Kopf. Johanna blinzelt überrascht, sie ist nicht tot\dots Aber sie ist sicher, sie wird es bald sein. Der Trecker springt und bockt wie ein junges Fohlen, unvorhersehbar und der Kopf springt wieder und wieder gegen die Stange. Wie lange wird es dauern\dots Johanna taucht in sich selbst ein, so weit es geht. Jetzt ist eine Schutzschicht wie dickes, wässriges Glas zwischen ihr und der Welt, sie kann Conny und den Bauern nicht richtig sehen und hören, aber dort ist es sicherer.

Warum lacht Conny? Die Freude kann Johanna nicht verstehen. Der Freund ist ganz unbekümmert und scheint keine Gefahr zu spüren.

Am Ende der Fahrt ist Johanna enttäuscht, sie lebt und fragt sich, ob sie um eine Erfahrung betrogen wurde oder nur um den Glauben in die Worte des Bauern.



\section*{Paradies II}
\addcontentsline{toc}{section}{Paradies II}



Es ist ein beschauliches Dasein. Der Geruch der Marsch hüllt alles ein. Moorig, wässrig vom warmen Sonnenlicht ausgelöst, verwandelt sich die schwere Erde in kräftigen, säuerlichen-süssen, holzigen Geruch, der ummantelt, nicht durchdringt. Durchbrochen von den zahllosen Knicks, die mit Büschen, losem Strauchwerk und kleinen Bäumen die Wege und verschiedenen Koppeln säumen, gibt das flache Land grosszügig die Himmelskuppel frei. Eine grünes Brett unter einer hellblauen, dunstigen Käseglocke. 

Eine aquarellige Nass-in-Nass Landschaft, dominiert von den Himmelsvariationen und dem, weit in die Ferne in immer hellerem, verwischt Geschichtetem, sich verlierenden Blick. Mal zeigen sich weisslich, milchig, von einer verschleierten Sonne durch lichtete Wolken, der Maler hat das nasse Blau mit einem trockenen Pinsel durchfahren und die beiden unterschiedlichen Flächen machen sich miteinander vertraut. Mal strahlt Blau. Mal ziehen in endloser Formation, auf Schnüren hintereinander aufgeperlt, dicke, flauschige,  von unten platt gedrückt und grau, Cumuluswolken mit dem Wind aus dem Westen.

Die Kinder sind ganz für sich. Es braucht unter der Käseglocke keine Erwachsenen. 

Der kleine Fluss hinter dem Wäldchen bei Johannas Haus ist tabu. Und der Garten der einzigen Nachbarn. Vor dem Haus ist die Landstrasse, die nach Oldendorf führt, dahinter lehmig, gelb die Kiesgrube. Dort sind die Kinder selten. Neben dem Haus ist ein Wäldchen und auf der anderen Seite das gleiche Haus mit den Nachbarn, Auffahrt, Feldweg zu Connys Hof. An der Landstrasse dann ein kleines Stück weiter das Kraftwerk indem der Vater arbeitet.



\section*{Humunculus}
\addcontentsline{toc}{section}{Humunculus}



Gasturbinenkraftwerk. Zwei grosse Klötze nebeneinander mit drei grossen, silbrigmatten Schornsteinen, die weithin sichtbar nach jeder Ausfahrt das Zuhause ankündigen. Zwei Gastanks, rund sind sie und riesig. Es ist eine summende, brummende, vibrierende Welt. Sie riecht nach Öl und Metall, der Geruch beisst in der Nase wie der vom Kuhmist. Aber dieser Geruch fühlt sich an wie kleine, tote Nadeln, und der ranzigen Schwere von Erdöl, dem Leichensaft von Mutter Erde, er hüllt nicht ein wie ein müffelnder, warmer Pelz, sondern tastet vielmehr ab, er durchdringt und sucht,\dots Dann gibt es den Geruch vom Büro, von altem Papier, Arbeitsanzügen, Holzschränke, Kaffee und dem kalten Rauch von Papas Pfeife, der ist wie Heimat.

Wenn Johanna hinter den haushohen Turbinen in die hintersten Winkel der Werkhalle schleicht, es, trotz Beleuchtung, dunkler wird, die Füsse auf dem Laufgitter einen mehrstimmigen, federnden Klang erzeugen, die Rohre, gelb und rot lackiert, die Ventile, Rädchen und Räder zum drehen, aus ihren Winkeln vorrücken, Gesichter bekommen und sie anstarren, dann wird sie vom Grauen ergriffen und rennt zurück zum Eingangstor, das die ganze vordere Seite der Halle ausfüllt und offen und grosszügig Tageslicht herein lässt. Der Vater wäscht seelenruhig das Auto oder bastelt daran, denn Johanna ist nur im Kraftwerk, wenn es Wochenende ist und der Vater sich seinem Auto widmet. Der Vater weiss nichts, von den geheimnisvollen Wesen, die in  der hinteren Ecke der Halle lebendig werden. Und im Tageslicht betrachtet, findet Johanna selbst, dass dort ausser Maschinen und Rohren nichts sein kann,\dots Sie schleicht zurück, um sich zu beweisen, dass alle Dinge Dinge sind, steife, starre Dinge\dots

Fasziniert beobachtet Johanna, während sie sich erneut an schleicht, wie die Angst in ihrem Inneren wächst und wächst, mit den beruhigenden Gedanken und dem Vertrauen ringt und plötzlich vorschiesst und sie mit Entsetzten ausfüllt und dann rennt sie auch schon zurück. Je mehr sie sich zur Vernunft zwingt, je langsamer sie Schritt für Schritt ins Dunkle geht, sich genau umschaut und alles betrachtet, um so heftiger kommt die Angst.
Es ist ein tolles Spiel.



\section*{Credo I}
\addcontentsline{toc}{section}{Credo I}



Abends ist Johannas liebste Zeit auf der Schaukel. Wenn sie auf der Schaukel schwingt und die  Sonne langsam den Weg zum Horizont nimmt, der Himmel heller als am Tag weiss-dunstig sich bereit macht die Farben des Abendrotes auf zu nehmen. Die lang gezogenen Wolkenstreifen zartes Rot und Orange, zu Beginn hellstes Goldgelb, Farbklängen gleich, in die Weite musizieren. Mit dieser Musik vibriert das Herz, immer stärker und tiefer. Die Schaukel schwingt den Bauch, lockert ihn, öffnet jede Zelle, damit das Licht eindringen kann. Heben und senken, der Bauch, der Körper werden schwerer und ruhiger, bis sie selbst aufgelöst mit dem Licht, den Farben klingen. 
Nichts berührt den Kopf, der darf still sein, der Rest ein riesiges Lauschorgan, selbst Wellen der Zufriedenheit verströmend. Im Kern der Engelsmusik, sie spiralt gleichermassen darum, die Leere, all-liebende, alles und nichts seiende Leere.

Ein Auto kommt auf der Landstrasse hinter dem Jägerzaun vorbei gefahren, von der Kupplung gebogenes Motorengeräusch, das schnaufende Auto wird in seiner Fahrt gebremst, Ortseingang. Der höher sich verbreiternde Autoklang vervollkommnend die Musik um Endliches. Das Auto ist Bewegung -bewegen, Weg.
 


\section*{Er-inner-ung}
\addcontentsline{toc}{section}{Er-inner-ung}



„Mama, Mama, schau mal mein Bild an! Mama wach auf, schau!“ Johanna ist zu den schlafenden Eltern ins Schlafzimmer gestürmt und springt auf das grosse Bett. Die Mutter richtet sich verschlafen auf: „Was?“ „Schau doch, wie ich gemalt habe!“ Der Mutter wird ein Blatt mit kleinen Figuren unter die Augen gehalten. „Mmh? Ja, schön hast du das gemacht.“ „Ich hab die Haare anders gemacht und jetzt sehen sie wie richtige Menschen aus!“ „Prima, komm, geh leise wieder raus, bevor der Papa auch geweckt wird. Ich komm bald und mach Frühstück.“ Johanna geht leise, verzückt aus dem Schlafzimmer und malt gleich ein weiteres Bild auf dem die Menschen viel mehr wie Menschen aussehen als vorher. Johanna hat lange probiert und verschiedenes getestet, sie ist glücklich und stolz, dass sie eine neue Methode gefunden hat, Menschen gut zu zeichnen.

Wenn die Eltern am Wochenende länger schlafen, sitzt sie oft und malt, die Mutter hat ihr dann eine Schüssel mit Kellogs Smacks hingestellt, die knappert sie. Erst jedes einzelne, wie eine Kostbarkeit, dann mehrere, eine Handvoll und zum Schluss hält sie sich die Schüssel an den Mund und mampft die süssen Körner gierig in sich hinein. Sonst weckt sie die Eltern nicht, aber an diesem Tag gelingt ihr zum ersten mal ein kleines Wunder, sie schafft es Dinge zu zeichnen, wie sie sie sich vorstellt und sieht. Bei diesem ersten Mal-Erlebnis ist Johanna vier Jahre alt.



\section*{Paradies III}
\addcontentsline{toc}{section}{Paradies III}



„Komm, der Mist fährt auf den Misthaufen.“ Johanna und Conny stürmen aus dem Kuhstall. In der Rinne hinter den Kühen fahren die Schieber hin und her, klappen sich ein und aus und schieben die grünen, breiigen Haufen Stück um Stück auf das Förderband zu. Dieses steht hoch aufgerichtet vor dem Misthaufen und vergrössert diesen portionsweise um weiteres Grün.
Erwartungsvolle Spannung, wie gross wird der Haufen, der als nächstes herabstürzt, sein? Wie weit wird er spritzen?

Es ist eine Frage des Könnens, den Haufen, der hochgefahren kommt, einzuschätzen, im richtigen, aber letzten Moment vor den grünen Spritzern zu flüchten.

Im Inneren des Stalls fressen die Kühe gasend ihr Abendheu, während sie gemolken werden. Es muht von den Kühen, es klirrt von ihren Bügeln, durch die sie ihre Köpfe schieben, die sich dann um ihren Hals schliessen. Johanna findet es traurig, dass die Kühe sich, sind die Bügel geschlossen kaum bewegen können. Weiter surrt die Melkmaschinen und gibt gleichzeitig ein klackerndes Geräusch wieder, mit dem die Milch rhythmisch durch die durchsichtige, verstaubte Leitung pulsiert. In der Milchkammer steht eine von Connys Schwestern und giesst von der frischen Milch in Eimer. Die bringen Conny und Johanna den kleinen Kälbern. 

„Schau, meins ist schon ganz zahm!“ “Meins auch!“ Sie müssen die Eimer gut fest halten, denn die hungrigen Kälber stossen kräftig ihre Köpfchen in den Eimer um jeden Tropfen Milch zu schlecken, hinten wackeln und zittern die kleinen Schwänze. Nach der Trinkzeit lassen die Kinder die Kälbern eine Weile an ihren Fingern saugen. Die raue Zunge schmiegt sich an die Finger, die oben von dem zahnlosen Kiefer geknetet werden. Es schmerzt. Johanna freut sich aber, denn sie kann auf  das kleinen Kalb Acht geben. Mit dem Eimer in der Hand laufen die Kinder in die Tenne, jeder macht einen Griff in das Schrot und ab in den Mund damit. Im Winter läuft die Häckselmaschine, die aus den Futterrüben kuh- und kindgerechte Schnitze macht, die bei laufender Fahrt aus der Maschine geklaubt werden.



\section*{Tod II}
\addcontentsline{toc}{section}{Tod II}



Im Winter spielen Johanna und Conny in der Scheune verstecken. Sie ist bis unter das Dach mit Heu und Strohballen gefüllt. Die Kinder sollen das Stroh nicht durcheinander bringen und damit spielen- unmöglich. Für kleine Höhlen reicht das lose Stroh.

„Komm, wir springen von den Ballen dort oben!“ „ Ich traue mich von dem dort, der ist höher!“ Conny steigt höher und Johann will nicht nachstehen. Ein Moment hebt sich der Körper empor, schwingt mit der Luft und kehrt zurück auf die feste Erde. Sie können nicht genug bekommen und höher, höher\dots Der Aufprall staucht die Beine, der Schmerz wird grösser, die Luft wird aus den Lungen geschleudert, wenn sich der Leib fester und fester nach dem Höhenflug an den Boden presst. Es geht um die Ehre.

Conny springt. Bleibt liegen. Johanna klettert schnell herunter. Conny liegt da und rührt sich nicht mehr, die Augen sind geschlossen.

Johanna stürmt ins Haus und holt den Bauern und Bäuerin. Die Angst kommt mit, die Angst , die kribbelige, die erregende Angst vor dem Tod.

Conny ist wieder da. Die Eltern erleichtert, stürmen auf Johanna ein: „Conny hätte sterben können, du weisst, dass das Verboten ist! Das dürft ihr nicht\dots“ Johanna ist getroffen und verletzt, warum schreien der Bauer und die Bäuerin mit ihr herum, schliesslich war es Conny, der sich gestossen hat. Es ist nicht recht. Johanna kneift fest die Lippen zusammen und sagt nichts. Aber das Gewissen nimmt ein Stück Schuld zur Untermiete ins Herz.



\section*{Gnomus}
\addcontentsline{toc}{section}{Gnomus}



Am Abend fährt Johanna heim. Allein. Vorn, am Fahrradlenker, baumelt eine Tupperkanne mit frischer Milch. Der Weg besteht aus zwei Betonplattenspuren, getrennt von Gras. An den Seiten hat es Gras und an beiden Seiten Knicks mit Büschen und Sträuchern, kleinen Bäumen. Ein einfacher Weg, ein klarer Weg. Hinter der Kreuzung kommt das kurvige Stück. Johanna fürchtet sich vor dem kurvigen Stück, ihr ist jedes mal, als werde sie beobachtet, als wollte etwas hinter dem Erdwall hervorspringen und ihr den Weg versperren. Es ist keine besondere Ecke, die dunkler wäre oder enger, es ist nichts zu sehen und doch müssen unsichtbare, wütende und raubende Wesen hinter dem Wall sitzen, auf Beute lauernd. Dort fährt sie, so schnell sie kann. Einmal, als sie es besonders eilig hat, rutscht der Reifen über die Betonplattenkante auf die ausgewaschene Grasspur, bleibt widerspenstig, schliesslich bockt das Fahrrad und Johanna fällt.

Sofort taucht Johanna unter, nichts hat mehr Platz unter ihrer Seelenwasserkuppel ausser namenloser Schrecken\dots Nicht den kleinsten Schmerz spürt sie, während sie mit fliegenden Händen den Lenker greift und im Laufen auf das Fahrrad springt und wild bis zur letzten Kurve strampelt. Auf dem letzten, geraden Wegstück spürt sie ihr Herz im Mund klopfen und bemüht sich es wieder runter zu schlucken. Es dröhnt in den Ohren. Die Knie melden sich mit dem Schmerz vom Aufprall,  langsam geht die Fahrt nach Hause. Links der hohe, dicht bewachsene Knick und rechts eine Koppel, die die Blick auf den Abendhimmel öffnet. Über dem Kopf, hoch oben, die Strommasten und Leitungen, die den Strom fort bringen, den der Vater im nahen Kraftwerk gemacht hat. Es summt in der Luft.



\section*{Il uomo nero}
\addcontentsline{toc}{section}{Il uomo nero}



Conny und Johanna steigen auf die Fahrräder, sie wollen zu Johanna fahren. Jeder auf einer der von Gras getrennten Betonplattenspuren. Der klare Weg, ihr Weg, den kaum mal ein Spaziergänger betritt ist heute anders, still, kein Vogel ist zu hören, die Bäume auf den Knicks zu beiden Seiten wirken vor dem nebelig-trüben Himmel dunkel, die Luft legt sich dick, schwül auf die Augenlider.
 Ihnen entgegen kommt eine schwarze, in eine lange Kutte gehüllte Gestalt. Es ist ein Mann mit schwarzem, zerzaustem  Haar, einem langen Bart und bleichem, blassen Gesicht. Als er die Kinder sieht, bleibt er stehen, breitet er die Arme über den Kopf und beginnt mit lauter Stimme in einer fremden Sprache zu sprechen.
 
Sie bleiben stehen. Die Stimme saugt an den Ohren, sie will die beiden Kinder einfangen. Nach dem ersten lähmenden Schrecken, drehen Johanna und Conny um und fahren mit fliegenden Beinen zurück zum Hof.

„Da, da kommt ein schwarzer Mann. Er kommt, er kommt!“ „Ein ganz schwarzer Mann, der ruft.“ Schreiend und keuchend kommen die Kinder in die Küche gestürzt. Das rote Tier Angst löst sich von ihnen und springt\dots Die Erwachsenen springen auf. „Du bleibst drinne!“, der Bauer schaut Johanna streng und besorgt an, die anderen stürmen aus dem Haus. Conny darf mit raus. Der einsam gelegene Hof hat zwei Einfahrten und Johanna sieht den Bauern zu der einen Einfahrt gehen, zu der, die dem Wanderer am nächsten liegt, einen Besen mit kräftigen Stil in der Hand. Die Bäuerin nimmt eine Mistgabel. Connys grosse Schwestern gehen zu der anderen Einfahrt. Es kribbelt in Johannas Bauch. Wer ist der Mann? Was wird der Bauer mit ihm machen? Kann er ihn besiegen?

Wut und Angst wandern mit den Erwachsenen in den Toreinfahrten mit. Das rote Tier, das von einem zum anderen hetzt und sich die Leftzen leckt, auf Futter wartete. Es möchte toben und Blut, es drängt sich zu den Menschen, umspringt sie. Ganz plötzlich sieht Johanna es und es ist gruseliger als der schwarze Mann, der allein den Weg herauf kommt. Johannas Herz klopft heftig, wenn der Bauer den Mann schlägt? Mit dem Besenstiel. Sie kann die Einfahrt nicht sehen. 

Johanna kommt es vor wie eine Ewigkeit, die sie alleine am Küchenfenster verbringt. Conny läuft auf dem Hof hin und her. Die Erwachsenen reden aufgeregt miteinander und schauen immer wieder den Weg hinunter. Dann wird das  Tier kleiner, unruhig dreht es sich im Kreis, entfernt sich langsam, von den Menschen, wird von einer unsichtbaren Wand von ihnen weg geschoben.
Schliesslich sammeln sich alle in der Einfahrt, die dem Wanderer am nächsten liegt.  

Von den Mann ist keine Spur zu sehen! Er kommt nie an der Einfahrt an!
Wohin ist er auf dem schmalen, von Knicks und Feldern umgebenen Weg verschwunden\dots

Die Erwachsenen verstauen Besen und Mistgabel und lachen befreit, vorwurfsvoll, was Kinder für eine blühende Phantasie haben\dots Widerwillig verschwindet das rote Tier einen Rest Wut zurücklassend und Scham, die Erwachsenen können sich nicht in die Augen sehen.



\section*{Credo II (Oldendorf, Dithmarschen, 1976)}
\addcontentsline{toc}{section}{Credo II (Oldendorf, Dithmarschen, 1976)}



„Natürlich gibt es den lieben Gott!“ sagt Conny. 

„Bist du sicher?“

„Ja, der wohnt im Himmel und hat einen langen Bart. Der sitzt dort auf einer Wolke und wenn man betet, dann hört er das. Und er sieht alles, was du machst.“

„Und das Paradies?“  

„Das ist im Himmel, dort kommt man hin, wenn man stirbt. Oder in die Hölle, dort brennt es. Da kommen die Bösen hin.“ 
„Kann jeder mit dem lieben Gott reden?“ „Ja, man sagt, lieber Gott im Himmel und dann was man möchte\dots“


„Wenn ich Tod bin, bleibe ich nicht im Sarg liegen?“


„Nöh, dann kommst du zum lieben Gott\dots“

Ich liege nicht im Sarg als würde ich tief schlafen, denkt Johanna. Ich muss nicht im Sarg liegen und die Würmer kommen und fressen mich, während ich schlafe. Ich muss nicht im Sarg liegen, während die Würmer meinen Körper fressen und ich nicht träumen und nicht aufwachen kann\dots Ich kann nicht aus versehen aufwachen, wenn die Würmer meinen Körper zerfressen\dots

Wenn ich sterbe, darf ich weiterleben\dots bei dem lieben Gott im Himmel.

So glaubt Johanna still und heimlich, während ihr Kopf verzweifelt versucht den mütterlichen, kommunistisch-atheistischen Vorstellungen des Todseins zu folgen.


\section*{Prinzessin (Oldendorf, Dithmarschen, 1975-1977)}
\addcontentsline{toc}{section}{Prinzessin (Oldendorf, Dithmarschen, 1975-1977)}



Johanna greift in die Verkleidungskiste und holt ein glänzendes Kleid hervor und die schicken Sandalen mit Riemchen und kleinen Absätzen, solche, wie sie grosse Frauen tragen. Damit stolziert sie im Garten auf und ab.
 Wunderbar. 
 
 „Mama, darf  ich die Sandalen zum Einkaufen anziehen?“ „Nein, Johanna, das weisst du doch! Die Schuhe sind nicht gut für deine Füsse, weil die Einlagen nicht hineinpassen.“ “Aber Oma Frankfurt hat sie mir geschenkt.“ „ Johannaaa, nein!“ Johanna findet das ungerecht, warum darf sie keine richtigen Mädchenschuhe anziehen, nur diese hässlichen Sandalen, die hinten geschlossen sind, Babyschuhe.
 
Faschingszeit im Kindergarten. Plötzlich heisst es sich verkleiden. Johannas Mutter hat ihr ein Funkenmariechen Kostüm gekauft. Aber Johanna weiss nicht, was ein Funkenmariechen ist. Die rote Jacke mit den goldenen Knöpfen, sieht aus wie die eines Soldaten und das weisse Faltenröckchen sieht wie ein normaler Rock aus. Der roten Hut, ein Dreispitz, sieht auch aus wie für Soldaten. 

Aber Johanna darf die Sandalen mit Riemchen anziehen. 

Johanna bekommt einen Beutel mit Bonbons. Sie darf sie aber nicht essen. Ein Funkenmariechen wirft die Bonbons in die Luft, damit die anderen sie fangen und essen können. Das widerstrebt Johanna, warum soll sie all die Bonbons weggeben. Johanna versteht nichts, nicht, als was sie verkleidet ist und nicht, warum sie Bonbon zum Wegwerfen bekommt.

Aber, dass ihre Freundin Britta als Prinzessin zum Fasching darf, das versteht Johanna. Wie hat Britta das gemacht? Sie hat  nicht nur ein langen, goldgelben, glitzernden Rock und ein goldiges Oberkleid, nein, sie hat sogar ein kleines, glänzendes Krönchen mit einem Schleier auf dem Kopf. Und Britta schreitet umher und vergisst nicht, den Schleier, als ob er ihr langes Prinzessinenhaar wäre von der einen Schulter zu streichen und dann von der anderen zu streichen und den Kopf nach hinten zu werfen. Wie gern wäre Johanna eine Prinzessin\dots

Die Kinder tanzen und springen. „Johanna, du musst noch die Bonbons werfen und verteilen“, die Mutter hat den Beutel mit den Bonbons in der Hand. Johanna wird es heiss und dann läuft ein kalter Schauer den Rücken runter. Sie schämt sich und sie hat Angst. Wie soll sie das machen? Alle Kinder werden sie bedrängen, schreien und nach den Bonbons grabschen. Johanna möchte am liebsten den Sack, gefüllt wie er ist, mit nach Hause nehmen und das nicht, weil sie die Bonbons essen will. 

Die Mutter macht ihr Mut: “Los, Johanna, wirf` die Bonbons einfach zu den Kindern“. “Wie soll ich das den machen?“ Johanna ist verzweifelt.  „Du musst doch nur werfen!“, sagt die Mutter fassungslos, sie versteht nicht, warum Johanna es nicht macht. 

Die Bonbons fliegen, an die Wand und unter die Turnbänke, die rundherum aufgestellt sind. „Johanna, du sollst die Bonbons zu den Kindern werfen. Am Rand und unter den Bänken findet sie keiner“. Es dauert eine Zeit bis die anderen merken, dass Johanna mit Bonbons wirft. 

Dann kommt Silvia, die Tochter von der strengen Aushilfskindergärtnerin, die nur zu ihrer Tochter nett ist und alle anderen Kinder ausschimpft. „Darf ich auch Bonbons verteilen?“ Johanna kennt Silvia nicht gut, sie spielen an den zwei Tagen, an denen Johanna im Kindergarten ist, nie miteinander. Aber Silvia ist die Tochter von Frau Potthof und deshalb hat sie das Sagen bei den Kindern.  Silvia lächelt und wirft elegant die Bonbons. Die anderen Kinder drängen sich um sie und sie verteilt die Bonbons mit vollen Händen, als wäre sie das Funkenmariechen. Der Sack ist schnell leer, dass für Johanna kaum welche übrig bleiben. Silvia lacht und geht wieder Hüpfen und Tanzen, wie die übrigen Kinder. 

Und Johanna? Sie taucht in sich unter, ist froh, dass sie der Aufmerksamkeit entgangen ist und traurig, denn, die Mutter ist unzufrieden. Johannas Gewissen meldet sich, du hat es falsch gemacht und die Gier meldet sich, du hast nicht einen Bonbon abbekommen. 


\sterne


Es bietet sich eine neue Gelegenheit eine richtige Prinzessin zu werden. Eine Märchenprinzessin. Im Kindergarten werden beim Sommerfest Märchen aufgeführt. Johanna darf die Prinzessin im Froschkönig sein. Die Kindergärtnerin sitzt mit Johanna an dem Brunnen aus Pappe. „So, Johanna, jetzt stell dir vor, du bist die Prinzessin und dir ist deine liebstes Spielzeug in den Brunnen gefallen. Und nun weinst du!“ Die Kindergärtnerin sieht Johanna erwartungsvoll an. Johanna steht an dem Pappebrunnen, die anderen Kinder sehen zu. Johanna muss lachen und schaut die Kindergärtnerin hilflos an. „Ich zeig`dir wie es geht.“ Die Kindergärtnerin schlägt die Hände vor ihr Gesicht und gibt schluchzende Geräusche von sich:“Huuuuhuhuhuuuh!“ Am liebsten möchte Johanna wieder lachen, aber sie will sich Mühe geben, damit sie die Prinzessin vom Froschkönig sein kann. Johanna spürt wie sich ihr Körper verändert. Sie fühlt ihn stärker und stärker, gleichzeitig verschwimmt ihr die Kindergärtnerin. Sie versucht es:“Huhu, \dots“ Das Gesicht verrutscht und im Bauch ballt sich ein Glucksen zusammen. Johanna lacht, sie will nicht, aber sie muss lachen.
Nein, nein. Die Kindergärtnerin erklärt geduldig, Johanna wäre traurig, genau wie die Prinzessin und sie sollte es noch einmal versuchen.

Johanna wird es ungemütlich, sie möchte Prinzessin bleiben, aber tun, als ob sie weint, wenn sie nicht weinen muss, kann sie nicht. Warum, warum nur, kommt stattdessen ein Lachen aus ihr raus? Die Enttäuschung der Kindergärtnerin kribbeln in Johannas Bauch. Sie fühlt wie ein Stück von ihr tiefer  in sie hinein gesogen wird und sich in einen dunklen Winkel verkriecht. Johanna sitzt in der Taucherglocke. Johanna lacht mehr und mehr, sie kann nicht aufhören, als würde sie weinen. Dann ist Schluss mit dem Froschkönig. 

Johanna bekommt, als einzige, eine neue Rolle. Jetzt ist sie das Dornröschen. Sie weiss, dass die Kindergärtnerin sie zum Dornröschen bestimmt hat, weil sie dabei still mit geschlossenen Augen da liegen muss, das Einfachste auf der Welt\dots Aber, Johanna ist Prinzessin geblieben, eine mit einem unruhigen Gewissen, aber Prinzessin\dots

Zur Aufführung bekommt sie ein schönes Kleid und eine Krone. Die anderen Kinder singen das Dornröschenlied und tanzen um Johanna und ihre Kinderdornenhecke herum. Sie muss lachen, es kitzelt wieder im Bauch, diesmal, weil sie sich freut und dann weil es laut und lärmig ist und Johanna aufgeregt ist und sich wieder untergetaucht anfühlt. Sie kann nur noch ein Drehen im Bauch spüren, die anderen um sie herum, sieht und hört sie wie durch Wasser, weit weg und sie ist plötzlich ganz allein\dots Und deshalb lacht das Dornröschen statt zu schlafen und als der Prinz kommt, da springt es auf und hopst und springt wild mit ihm herum, dass ihm hören und sehen vergeht. 

Einige Wochen später merkt Johanna wieder wie schwer es ist, Prinzessin zu sein und: Johanna ist keine Prinzessin, nicht mal eine schlafende. Die Mutter hat die Fotos von der Aufführung abgeholt und lacht. Denn das Johannadornröschen liegt kichernd und mit aufgestellten Knien auf ihrem Prinzessinenbett, so dass jeder ihre Unterhose unter dem Kleid sehen kann\dots


\sterne


Jedenfalls will Johanna Klonschen haben. Die schicken Schuhe mit der Holzsohle, die alle tragen. Sie bittelt und bettelt und tatsächlich. Es sind nicht die, die sie sich gewünscht hat, sie darf die mit dem Riemen haben. Mit den neuen Schuhen geht sie hin und her, schaut auf ihre Füsse und spürt die harte Sohle am Fuss. Es fühlt sich grossartig, ungewohnt und ungemütlich an. Die Riemchen schiebt sie oben auf den Schuh und dann wieder hinter die Ferse, wieder auf den Spann, wieder an die Ferse, sie kann sich kaum satt sehen. Mit solchen schönen Schuhen muss sie in die Welt, Conny besuchen.

 „Aber nicht mit den Klonschen“, sagt die Mutter. „Bei Conny musst du andere Schuhe anziehen.“ Johanna kann es nicht fassen, sie hat endlich richtige Schuhe und darf damit nicht zu Conny gehen. „Ich pass` doch ganz doll auf.“ verspricht sie. „Aber dann musst du das Riemchen hinten tragen“. Johanna steigt auf das Fahrrad. Ein bisschen böse ist sie schon, wegen den Riemchen, die sind für Babys.  Dennoch,  sie kommt sich wie ein echtes Mädchen vor, eine echte Frau. 
 
Conny zeigt sich wenig beeindruckt über Johannas Schuhe, er hat andere Pläne und es geht über die Koppel im Galopp. „Warte“ schreit Johanna Conny nach, denn sie lernt, dass  Mädchenschuhe den Füssen einige Schwierigkeiten bereiten können, wenn es um Schnelligkeit geht. Conny ist beim Stall angelangt, Johanna hetzt hinter drein. Ein stechender Schmerz fährt ihr in Knöchel, der über einer alten Flasche, die gut versteckt im Gras die Sonne geniest, davon, zur Seite rollt und sie dabei zu Boden wirft. Johanna springt wieder auf, aber sie kann nur humpeln, der Knöchel ist dick geschwollen als sie heim kommt. „Siehst du, ich habe es dir ja gesagt!“ Ja, das hat die Mutter getan und Johanna hat bei der Strommassage, die der Knöchel über Wochen bekommt, Gelegenheit darüber nachzudenken, wie sinnvoll es ist auf die Erwachsenen zu hören. Schuldig, sagt das Gewissen. Warum kann Johanna kein richtiges Mädchen sein?


\sterne



Wenn Johanna mit den Freunden, den Jungen, unterwegs ist, dann fühlt sie sich wirklich als Prinzessin, oder als Mädchen. Und das ist gut. Denn Johanna ist ein starkes Mädchen, eines, das mit den Jungen mithält, keine Heulsuse. Die Jungen, das sind Sönke, Holger, Kai, Heiko und Conny.
Die Jungen sind alle miteinander verwandt und Johanna ist die „Henne“ im Korb. 

Der Sönke, der ist das Mädchen bei ihnen, denn er heult schnell und kann nicht schnell rennen. Und auch beim Weitpinkeln ist er der letzte.- Da kann Johanna nicht mitmachen, aber das macht nichts. Schliesslich braucht es einen verlässlichen Schiedsrichter, der schaut, wer es am weitesten geschafft hat.- Und weil Johanna genauso gut ist wie die Jungen und der Sönke das Mädchen ist, deshalb können auch Kai, der kleine Bruder vom Heiko und Holger, der kleine Bruder vom Sönke und vom Helge, der aber nicht mehr mit ihnen spielt und viel älter ist, nichts sagen. Denn den beiden ist es nicht immer recht ein Mädchen dabei zu haben.

Johanna ist gerne ein starkes Mädchen, sie misst sich mit den Jungen, will die kleinen Wettkämpfe gewinnen. Sie will den Mut und ihre Kraft trainieren. Ein kleines Stück über sich hinaus wagen, und schaffen! Instinktiv spürt sie dabei ihre unbeugsame Stärke, die eigene Verlässlichkeit und, dass beides lebenswichtig ist.

Es gibt viel zu erleben. Bei Conny auf dem Hof und auf den Koppeln, bei Johanna im Garten. Manchmal machen sie Fahrradtouren. Zu dem kleinen, viereckigen Tannenwald, in den sich aber nur die Mutigsten wagen, weil der Bauer, dem es gehört, dort mal ein totes Kalb mit einem Strick um den Hals liegen gelassen haben soll. Es heisst, der Bauer lauere mit der Schrottflinte und würde auf jeden schiessen, der seinen Wald betritt.

Einmal sind sie bei Kai und Heiko zu Hause. Dort sitzen sie ganz artig in der guten Stube und spielen Mensch-ärgere-dich-nicht. Es ist allen ungemütlich. Und sie wissen nicht richtig, wie sie sich benehmen sollen. Im Haus, am feinen Stubentisch fühlen sie sich erwachsen und versuchen sich zu benehmen, das ist langweilig. Aber keiner sagt es. 

Johanna lernt Neues kennen. Sie, die mit dem Conny verlobt ist, bekommt einen Verehrer. Heiko, der älteste von allen, er ist schon 7 Jahre alt, merkt, er ist in Johanna verliebt. Johanna findet das mal gut, nämlich dann, wenn einer von den kleinen Jungen über Mädchen schimpft und dann vom Heiko einen Rüffel kriegt manchmal ist es aber blöd, nämlich, wenn Heiko den Beschützer spielt und nicht will, das Johanna all die „gefährlichen“ Abenteuer mit macht.

Einmal, alle fahren rasend schnell mit ihren Fahrrädern auf dem Feldweg zu Johannas Haus, da fällt Johanna hin. Sie hat von allen das kleinste Fahrrad, die anderen sind weit vor raus, sie rutsch mit dem Vorderrad über die Kannte des Plattenweges, der Lenker fliegt herum, das Fahrrad springt, padauz. Johanna kriecht unter dem Fahrrad vor, sieht die anderen um die Ecke zu ihrem Haus abbiegen und ärgert sich. Oh, wie sie sich ärgert. Wütend steigt sie mit schmerzendem Knie auf ihr dummes, kleines Fahrrad und strampelt hinterher.

Als sie endlich zu hause ankommt, steht die Mutter in der offenen Tür, Heiko liegt vor ihr auf der Auffahrt am Boden: “So ist sie hin geflogen. So, so auf der Seite. Und das Bein war da unter dem Fahrrad.“ „Johanna, da bist du ja, du Arme, ist alles in Ordnung?“ fragt die Mutter besorgt.
 Johanna kocht, warum muss Heiko diesen peinlichen Sturz vor allen Jungen und der Mutter breit treten? Was geht ihn das an? Wieso nimmt er ihren Sturz um sich vor den anderen wichtig zu machen? Sie wollte die Sache lieber für sich behalten, sich wie ein Junge benehmen und tun als sei nichts passiert. Warum will Heiko Johanna wie ein Mädchen behandeln? Er schenkt der Prinzessinnenseite eine ungewohnte und unangenehme Aufmerksamkeit.
 
Als Heiko dann aber mit seinem kleinen Bruder an Johannas Geburtstag morgens früh vor der Tür steht, um ihr sein Geschenk zu bringen, in feinem Hemd und sauberer Hose, fein auf die Seite gekämmtem Haar, wie ein kleiner Herr, ist sie geschmeichelt. Sie will prinzessinengleich nach dem Päckchen greifen.  Aber die Mutter schickt die Jungen wieder nach Hause, sie sollen am Nachmittag kommen, wie alle Geburtstagsgäste.



\section*{Feigenbaum I 
Landesmuseum Schleswig-Holstein, Schloss Gottorf, Raum der Moorleichen, 1977)}
\addcontentsline{toc}{section}{Feigenbaum I (Landesmuseum Schleswig-Holstein, Schloss Gottorf, Raum der Moorleichen, 1977)}



{\em Ich liege still, begafft  mit angeekelten Blicken bemitleidet. Meine Haare haben sie genommen, damals, das Moor färbt Haare rot.
Mein Kopf ist schwarz und ich liege, schwarz an Leib und schwarz an Erinnerungen, schwarz in der Stille. Begierig kriechen ihre Blicke über mich, meinen Tod, meine Wiedergänger tanzen um sie.

Die bösen Geister trinken aus ihrem Mitleid und sie starren durch die Scheibe.

Ich springe sie an, das kleine Mädchen hinter der Scheibe, sieh her, was sie mit mir, mit uns getan haben. Ich kralle mich in ihre erstarrten Augen, die wiedererkennen, aber nicht begreifen, im Schock geweitet, mich verschlingen. Sieh mich an, meine vertrockneten, vom Strick zerstörten Stimmbänder knistern ihr ins Ohr:

"Ich war du und die, die du werden wolltest, wird ich werden, bis du mich in dir erlöst hast."

Sie trägt mich mit sich nach Hause und ich nehme sie mit jede Nacht\dots
Jede Nacht ertrinken wir im Moor und im Laufe der Jahre zieht sich der Strick, der meine Stimmbänder und meine Luftröhre zerbrach von Innen um ihren Hals.

Sie wird stumm, erstickt an der Stille, in der ich in ihr ertrinke im Moor, jede Nacht\dots

Nacht, um Nacht, um Nacht\dots

Ich liege, sie liegt und die Äste prasseln auf uns, verkrampft im Tod, in der Schande, verewigt am Pranger der Geschlechtlichkeit, ins Moor gegossen, im Schaukasten ausgestellt.

Ich liege in ihr, ich flüstere, ich knistere ihr den Strick um den Hals:
"Es ist das Herz! Törichtes Ding. Es muss schweigen, damit du leben kannst, dass es zwischen deinen Beinen still bleibt. Es ist das Herz."

Sieh mich, ich bin schwarz, schwarz wie die Nacht, wie die Sünde, wie die Offenbarung und die Verheissung. Schwarz und zerspielt\dots

Ich bin Hülle. Ich war Hülle. Aber das Moor, die schlüpfrige, alte Erdenmutter, sie hat mich zur Figur gemacht. Das Moor machte aus mir das Bild deines Schattens.

Ich bin schwarz. Ich bin tot und durchdrungen von der Erde zurückgekehrt, damit sie, die ihr Herz sucht, sich spiegelnd ins Licht wenden kann.}


\section*{Anderswelt I}
\addcontentsline{toc}{section}{Anderswelt I}



Johanna steigt mit der Mutter aus dem Auto. Heute ist ihr erster Schultag und sie ist aufgeregt. Die Mutter gibt ihr eine grosse Schultüte aus der oben der Kopf eines Barbiepferdes herausragt. Das hat sich Johanna lange gewünscht. Die Schule ist nicht die, an der sie früher immer vorbeigefahren sind, wenn sie in Itzehoe einkaufen gingen und wo Conny und Britta zur Schule gehen. Sie sind lange mit dem Auto gefahren bis zu der Stadt, wo Oma Ofeld, Mamas Mama im Nachbardorf wohnt.

Conny ist ein Jahr vor Johanna in die Schule gekommen und brachte von dort einen „Fu“ mit. Einen, aus einer alten Socke genähten, Fingerpuppenhund, der den Kindern im ersten Schulbuch erklärt, wie Lesen und Schreiben geht. Johannas Mutter hat ihr auch einen Fu genäht und die Kinder sitzen mit ihren Fus auf der Hand über das Schulbuch gebeugt und üben Buchstaben schreiben auf Blättern mit Linien.

Und heute darf Johanna selbst in die Schule\dots

Sie ist mit der Mutter in dieser Schule gewesen um ihren Lehrer kennen zu lernen, wie die Mutter sagt. Johanna findet alles merkwürdig. Den muffigen Geruch des Büros, indem sie und die Mutter von zwei Lehrern erwartet werden und deren prüfenden Blicke sind unangenehm. Der junge Lehrer will Johanna einen Ball zu werfen, den sie auffangen soll und Johanna bekommt Angst, denn fangen kann sie nicht gut, aber es geht. Und Rückwärtslaufen soll sie auf einem Strich, der auf den Boden gemalt ist. Johanna scheinen alle die Aufgaben, die ihr der Lehrer gibt, schwierig, dabei ist sie hundertmal rückwärts gelaufen und hat Dinge gefangen, aber, wenn der Lehrer es sagt und sie dabei streng beobachtet, dann fühlen sich ihre Arme und Beine anders an, als gehörten sie ihr nicht.

Zum Schluss darf Johanna ein Bild malen und sie entspannt sich. Wenn sie eine Sache kann und gern hat, dann ist es Malen. Sie malt ein Haus mit Garten und Himmel und Wolken und ist zufrieden. 

Aber der Lehrer ist es nicht. „Das Bild ist doch noch nicht fertig. Was fehlt noch, Johanna?“  Johanna sieht sich ihr Bild an und kann nichts Fehlendes entdecken. „Doch, das Bild ist fertig.“ „Nein, das Haus ist kein richtiges Haus, es hat nur einen Strich rundherum. So sieht kein Haus aus. Du musst es ausmalen.“ Johanna presst die Lippen zusammen. Sie malt nie ihre Häuser aus. Aber, wenn der Lehrer es will\dots Johanna nimmt den Buntstift und beginnt zu malen. Es ist mühsam und Johanna findet ihr Haus wird durch die vielen wirren Striche immer hässlicher. Sie legt den Stift weg. „Johanna, schau, das Dach braucht noch Dachziegel.“ Der Lehrer lässt nicht locker. Also malt Johanna das Dach an. Als sie fertig ist, sieht sie ein Bild, wie sie es noch nie von sich gesehen hat und sie mag es nicht leiden. Und das Gewissen sagt: „Der Lehrer sagt, du kannst nicht richtig malen.“


\sterne

Aber all das ist am ersten Schultag vergessen. Johanna wird heute Lesen, Schreiben und Rechnen lernen und sie geht selbst zur Schule.
Zuerst sammeln sich alle Kinder, die in  die Schule kommen in einem kleinen Haus, das sechs Ecken hat, der Eurythmiepavilion, aber das weiss Johanna ja nicht. Die anderen tragen auch Schultüten. Sie setzten sich vor eine kleine Bühne und die zweite Klasse führt ihnen ein Theaterstück vor, indem es um die Wochentage geht. Während der Aufführung lässt ein Junge seine Schultüte fallen und einige Sachen fallen heraus. Weintrauben hat er in seiner Schultüte. Der Junge tut Johanna Leid, denn sie wollte keine Früchte in ihrer Schultüte und sie freut sich auf ihr Barbiepferd.

Endlich ist die Aufführung vorbei und der Lehrer verspricht, dass sie in die Klasse gehen und ihre erste Unterrichtsstunde haben werden! Für jedes Kind liegt auf jedem Schulplatz ein Heft mit leeren, weissen Seiten und ein Kästchen mit Wachsmalkreide. Johanna wundert sich, sie findet Wachsmaler doof, zu hause malt sie lieber mit Filzstiften. 

Die Kinder dürfen nicht in ihr Schulheft malen, sondern sie bekommen ein Blatt. Der Lehrer, Herr Boss, erzählt den Kindern eine Geschichte. Dann zeigt er auf die Tafel. Auf der einen Tafel ist ein dicker, senkrechter Strich, der Lehrer nennt ihn „Grade“, gemalt, blau mit gelb drumherum, auf der anderen Tafel ist ein halber Kreis, der Lehrer nennt ihn eine „Krumme“ und dies beides sollen die Kinder auf ihr Blatt zeichnen.

Keine Buchstaben, keine richtigen Stifte, keine Bücher, kein Fu und keine Zahlen, keine Hefte mit Linien\dots?

Nach einer kurzen Stunde fahren Mama und Johanna wieder nach Hause\dots
Johanna freut sich als sie nach Hause fahren, sie will das Barbiepferd aus der Schultüte befreien und am Nachmittag kommen alle Freunde um mit ihr, wie am Geburtstag, ein Fest zu feiern, weil sie in die Schule geht. In eine komische Schule mit lauter fremden Menschen. Wo alles anders ist\dots


\sterne


Am Anfang ist es lustig. Die Schule ist kurz und Johanna und die Mutter fahren viel mit dem Auto. Vor allem wird plötzlich die Oma viel besucht, die im Nachbardorf neben der Schule wohnt. Dort kann Johanna am Bahndamm und im Garten sein und die Oma kocht das feine Gemüse aus dem Garten und freut sich, dass Johanna es gerne isst.

Ausserdem findet Johanna den Unterricht spannend und mag den Lehrer, Herrn Boss, sehr. Der kann tolle Geschichten erzählen und jeden Tag gibt es Neues zu erfahren. 

Johanna bemüht sich sehr, dass sie mit Herrn Boss reden kann und möchte ihm am liebsten alle Neuigkeiten mitteilen. Aber es ist schwierig Herrn Boss etwas zu sagen, denn es hat viele andere Kinder, vierzig Stück und alle wollen Herrn Boss erzählen. Einige von den Jungen wollen lieber raufen und jemanden verprügeln und um die muss sich Herr Boss vor allem kümmern.

Johanna hat die Schule gern, aber auf dem Stuhl sitzen, dem harten, aus Holz, ist schwer. Die Beine wollen sich bewegen und allzu schnell schlüpft das Bein unter den Po und dann rutscht sie, die Beine wechselnd hin und her, oder Johanna hängt eine Zeit über den Tisch. Die rhythmischen Übungen, bei denen die Kinder stehen, die sind gut, da darf sie stampfen und trappeln und trippeln. Klatschen und alle sprechen zusammen.

Im Unterricht ist Johanna still und lauscht den Geschichten, sind diese lustig muss sie lachen und dann lacht sie laut und lange. Herr Boss wird das im ersten Schulzeugnis erwähnen, dort klingt es tadelnd: „Johanna lacht ausdauernd und laut im Unterricht.“

„Herr Boss, Herr Boss, ich habe ihnen ein Bild gemalt!“ Johanna strahlt, was für eine Mühe hat sie sich gegeben ein besonders schönes Bild, schliesslich ist Herr Boss ihr Lieblingslehrer und sie möchte seine liebste Schülerin sein. Herr Boss schaut seltsam, er scheint sich nicht zu freuen, er scheint nicht zu wissen, warum Johanna ihm ein Bild schenken will. Das Gegenübergesicht des Lehrers wird hart. Johanna wird von einer Kraft, aus dem Blick zurückgestossen, die innen drin Weh macht.

Johanna bemüht sich alle Aufgaben richtig und gut zu machen, sich zu benehmen. Sie hofft, Herr Boss wird es bemerken und sie loben. Sie möchte wissen, ob sie ihre Sache gut macht. Aber es gelingt ihr nur selten, Herrn Boss` Aufmerksamkeit  zu erlangen. Beim Rechnen, da begreift sie schnell und Herr Boss lobt sie. Sie freut sich, hat aber das Gefühl, als müsste Herr Boss darüber staunen, dass da eine gute Johanna in seiner Klasse ist.
Für die Spielkameraden ist kaum Zeit. Aber Johanna ist in Oldendorf, sie schläft dort, sie sitzt abends auf ihrer Schaukel und schaukelt in die Sonne hinein und sieht den Engeln beim letzten Tänzchen zu. Papas Kraftwerk summt neben dem Haus.

Die Anderswelt der Schule kann Johanna verlassen, wenn sie zu Hause ist, vergessen und in die gewohnte, liebe Welt in Oldendorf  zurückkehren.


\sterne


Dann zieht die Mutter mit Johanna um. Nach Ofeld, das Dorf in dem die Oma wohnt und das die Stadt mit Johannas Schule zur Nachbarin hat. Auch Onkel, Tante und vier Kusinen wohnen in Ofeld. Elisabeth, die jüngste Tochter des Onkels ist drei Wochen älter als Johanna und sie kennen sich gut. An Familientreffen und wenn Johanna die Oma besucht, dann spielt sie mit Elisabeth. Elisabeth hatte Johanna in den Ferien in Oldendorf besucht. Aber sie hatte immer Heimweh und Angst vor Johannas Mutter und deshalb wollte sie nie richtig spielen und nichts essen.

Der Vater muss daheim bleiben, weil die neue Wohnung zu klein ist und weil er im Kraftwerk arbeiten muss\dots Johanna fühlt sich fremd in Ofeld.

In Oldendorf hatte sie das Gefühl, die Wiesen, die Wege, die geheimen Ecken und Höhlen in den Wäldchen und Knicks gehören ihr und Conny. In Ofeld gehören all diese Orte jemand anderem. Elisabeth kennt sich viel besser aus in Omas Garten und sie kennt die Oma und den Opa, der meist im Garten oder im Haus sitzt, besser. Johanna muss Oma und Opa teilen. Und jeden Platz den sie betritt, muss sie teilen oder bitten, ob sie ihn betreten darf.

Es ist gut, dass Elisabeth da ist und die geheimen Ecken und Spiele kennt, aber für Johanna ist es „geborgter“ Platz, sie sehnt sich zurück an ihren eigenen Ort.

Es ist eine komische Wohnung und eine komische Vermieterin, sie kommt Johanna wie eine Hexe vor, eine, die auf den ersten Blick ganz manierlich aussieht und dann von hinten über einen kommt. Es wohnen andere Leute mit in dem Haus, dass direkt an der Hauptstrasse an der ersten Kreuzung steht. Johanna wohnt nicht gerne mit all den Leuten unter einem Dach. Es hat diese fremden Geräusche und Gerüche. Dem Haus fehlt der Garten, wenn Johanna vor die Tür geht, steht sie auf der Strasse. Die Mutter ist unzufrieden und es gibt Schimpfe und Streit am Wochenende mit dem Vater. 

Aber, das spürt Johanna, die Mutter ist auch froh, dass sie in Ofeld wohnen. Sie müssen nicht weit fahren und sie findet Johanna könnte endlich mit ihren Schulkameradinnen spielen, von denen es welche im Dorf gibt. 


\section*{Andorra I}
\addcontentsline{toc}{section}{Andorra I}



Johanna verabredet sich oder wird verabredet, kein Entkommen mehr vor der Anderswelt, die gibt es hier scheinbar überall. Geheimnisvolle Gesetzte, Verbote und Dinge, die alle wissen, nur Johanna nicht und Menschen, die ohne ihr Herz lächeln.

Die Klassenkameradinnen kennen sich vom Kindergarten und Johanna fühlt sich wie ein Eindringling. Die Mädchen sind freundlich, aber Johanna versteht ihre Sprache nicht. Sie spielen lauter Dinge, die Johanna nicht kennt. In Oldendorf hat sie mit Britta gespielt, aber seltener und Britta und sie haben wilde Spiele gehabt.

Aber die Mädchen aus der Schule sind ruhig und leise, wie kleine Erwachsene. Sie sind verwirrt, wenn Johanna laut und wild und polternd umher flitzt. Die Eltern der Mädchen scheinen sich über Johanna zu wundern und zu ärgern, nämlich, wenn ihre Kinder wild werden wie Johanna. Johanna spürt eine Wand und dahinter Ablehnung. Diese Wand fühlt sich bedrohlich an, denn die Menschen, die Johanna kennenlernt, verstecken sich dahinter. Sie lächeln vor der Wand und in sich drinnen passiert anderes: Unverständnis, Ungeduld, Ärger. 

Johanna spürt die Dinge hinter der Wand genau, aber sie weiss nicht, wie sie damit umgehen soll. Bisher kennt sie nur Menschen, wie die Bauerneltern von Conny, die sagen, wenn sie ungeduldig oder wütend sind. Das findet Johanna nicht angenehm, aber sie weiss, was sie falsch gemacht hat.

Wissen denn die Erwachsenen nicht, wie schlimm es für Johanna ist, wenn Johanna hinter der Wand des Erwachsenen schutzlos seinen unangenehmen Gefühlen ausgeliefert ist? Johanna ertrinkt immer öfter in den freundlich lächelnden, schmutzigen Sümpfen.  Dabei entdeckt sie einen Raum in sich, in dem sie sicherer ist, dort machen die spitzen Bemerkungen, Unverständnis und Ärger halb so weh. Aber der Raum ist weit in ihr verborgen und es ist ein weiter Weg dahin und ein weiter zurück. So geschieht es: Johanna bleibt öfter für längere Zeit in dem Schutzraum. Die Seelenwasserglocke hält die Welt fern. Schliesslich sagt sie, wenn Erwachsene dabei sind, kaum ein Wort. Sie, die früher alle wilden Jungenabenteuer  vorn an der Spitze mitgemacht hat, wird ängstlich, lässt sich von Klassenkameraden puffen, schlagen und beschimpfen und, weil sie tief in sich verborgen ist, ist der Weg zum herausspringen und denjenigen packen, zu weit.

Herr Boss macht eine neue Sitzordnung und Johanna muss neben Michael sitzen, vorne ganz auf der Seite. Michael ist ein kleiner, teigig-dicker Jungen, mit blassem, greisenhaftem Gesicht, der schlägt, frech ist und von der Schule nichts wissen will. Selbst die Rowdys in der Klasse wollen nichts mit ihm zu tun haben. Johanna ekelt sich vor ihm. Lieber möchte sie neben einer Schlangengrube sitzen. Sie rutscht so weit sie kann an den Tischrand. Erst als Michael ihr in den Bauch tritt, traut sie sich Herrn Boss etwas zu sagen. Der wird wütend und schimpft mit Michel, aber dieser Triumph ist fade, denn es nutzt nicht viel und schliesslich muss Johanna wieder auf den Platz. Kein Mädchen sitzt in der Nähe, niemand, zu dem Johanna gehören mag. Die nächste neue Sitzordnung folgt, aber nicht für Johanna, sie wird die nächste Runde, neben Michael vorne an der Wandseite begraben.

Dabei tritt diese Kraft wieder und wieder in Erscheinung und bereitet Schmerz. Antipathie. Johanna versteht nicht, was sie falsch macht. Aber, wenn sie Herrn Boss begegnen will, steht da diese Kraft. Dabei gibt es Kinder, die scherzen und lachen mit Herrn Boss, die dürfen erzählen. Sie befinden sich in einem feinen Kokon, der sie schützt. Johanna ist nicht darin. Sie begreift, auch wenn sie es nicht verstehen kann, nicht alle Kinder in der Schule sind gleich\dots Es gibt Kinder in und Kinder ausserhalb des Bindungsgewebes. Johanna glaubt sich schuldig. Sie muss selber Schuld sein, wenn sie nicht zu den guten Kindern gehören darf. Sie lernt, am Besten ist, sie ist unsichtbar\dots Vielleicht kann sie, wenn sie sich versteckt hält, so Zuneigung erlangen.

Das verhängnisvolle Band, das sich bei der Begegnung mit der Frau aus dem Moor um ihren Hals gelegt hat, zieht sich zu.



\section*{Amici}
\addcontentsline{toc}{section}{Amici}




Etwas besser wird es, als die Mutter und sie wieder mit dem Vater und all ihren Sachen in eine neue Wohnung, diesmal neben dem Kraftwerk in Ofeld, ziehen. Johanna bekommt ein schönes, grosses Zimmer und es hat Felder, Wiesen, alte Bahndämme und Wald drumherum. Dort löst sich die Anderswelt wieder auf, da kann Johanna durch atmen. Nachbarjungen hat es dort, den stämmigen, ruhigeren Matthias und Torben. Torbens Beine wollen nicht gleich schnell wachsen. Er trägt an einem Schuh eine ganz dicke Sohle, aber das Bein darf frei sein, während das andere von der Hüfte bis zum Fuss zwischen zwei Metallstangen steckt, die mit breiten Lederriemen an Ober- und Unterschenkel befestigt sind, das Bein ist dadurch steif, es soll sich strecken. Wenn Torben laufen will muss er sein geschientes Bein, weil es sich nicht knicken lässt, in einem kleinen Halbkreis nach vorne bewegen, dabei schwangt er mit dem Oberkörper weit in die entgegengesetzte Richtung. Torben ist klein und schmal. Als sie sich kennenlernen, ärgern sie sich mit freundlichen Schimpfwörtern und Beleidigungen, schliesslich sind sie in einem Alter, da Jungen und Mädchen nicht selbstverständlich Freunde werden können. Aus den freundlichen Schimpfwörtern werden kleine Spasskämpfe. Beruhigt merken die Jungen, dass Johanna eine gute Kämpferin und niemals zimperlich ist, im Gegenteil, Matthias hat keine Chance sie zu besiegen. Nachdem diese wichtigen Dinge geklärt sind, spielen sie, dem Dorfkindertabu folgend, das auch hier gilt, draussen. 

Allerdings gibt es ein Problem, denn Johanna weiss nicht, wie sie sich Torben gegenüber verhalten soll. Am Anfang ist sie lieber mit Matthias allein, denn die Schiene an dem Bein findet sie schrecklich. Sie ist behutsam mit Torben und wenn er sie zum Kampf auffordern will, dann weicht sie aus und wenn Torben angreift, packt sie ganz sanft zu. Oder sie windet sich heraus, in dem sie Matthias packt. Dadurch ist Torben draussen vor.

Heute ist es anders, denn Matthias ist nicht da. Johanna trifft Torben allein, bei sich auf dem Rasen. Torben beginnt die gewohnten Rangeleien und Johanna kann nicht ausweichen. Ihr Herz klopft immer mehr. Wie soll sie sich raus winden? 

„He, jetzt kämpf` mal richtig.“ Zornig blitzt Torben Johanna an. „Du machst gar nicht Ernst!“ Johanna blickt ertappt zu Boden. Wie soll sie Ernst machen, wenn Torben behindert ist? „Ich will, dass du richtig kämpfst!“ In Torbens Augen sieht Johanna nicht nur Trotz, sondern auch Wut und Trauer. Sie begreift, sie fügt Torben mehr Schmerzen zu, wenn sie ihm ausweicht, als wenn sie ihn im Kampf Weh tut. „Okay, wenn du willst.“ der Kampf beginnt, Johanna ist erstaunt mit welcher Kraft Torben auf sie losgeht. Aber sie traut sich nicht an zu greifen. „Du kannst mir ruhig wehtun.“ Torben wirft ihr einen bittenden, erwartungsvollen Blick zu. Also, Johanna packt richtig zu. Ein Titanenkampf beginnt. Sie ringen und zerren aneinander, versuchen den anderen mit Beinstellen zu Fall zu bringen. Keiner von beiden gibt nach, je länger es geht, desto mehr steigt die Wut der letzten Jahre auf. Und befreit spüren beide, sie haben einen würdigen Gegner gefunden, der die Wut aushalten kann.

Der Kampf endet unentschieden. Von da an sind spielt Johanna genauso mit Torben, sogar lieber. Dank der Jungen kann Johanna sich wieder benehmen wie sie es gewohnt ist, die Freundschaft ist lockerer, als die zu Conny, aber sie erinnert an die Zeit, da Johanna mit den Jungen in Oldendorf durch die Marsch ziehen konnte.

 Und es ist ein Kraftwerk da, das summt und brummt, wo sie mit dem Vater am Wochenende mitgeht und „Büroarbeiten“ macht, zwischen den Rohren und Ventilen schleicht, genau gleich wie in Oldendorf, ein wenig Heimat unter den drei Schornsteinen.  Und der riesige Wald, der die ganze Seite der langen Strasse, die ins Dorf führt, bedeckt. Ein guter Wald zu Spielen, es hat eine Pferdekoppel die mitten im Wald, an der Strasse liegt.
Johanna bekommt ein neues, ein grosses Fahrrad, mit dem sie in die Schule fährt. Johanna kann nach der Schule zur Oma oder nach Hause fahren. 



\section*{Tod III (Oldendorf, 1982)}
\addcontentsline{toc}{section}{Tod III (Oldendorf, 1982)}




Johanna ist für einen Nachmittag mit den Eltern bei Brittas Familie in Oldendorf zu Besuch. Während die Eltern drinnen beim Kaffee erzählen, spielen die Mädchen und einige Kinder aus dem Dorf in Brittas Garten. Es gibt Streit mit Ramona, die Kinder mögen sie nicht und schnell werden  deftige Wörter ausgetauscht.  

„Ätsch, Conny ist tod!“

„Du lügst, du blöde Kuh!“

„Nein, ich lüge nicht, Conny ist tot!“ Johanna schluckt, sie weiss selbst, dass Conny lange im Krankenhaus war und viele Operationen bekam. Sie hat Conny mit der Mutter besucht, als er in Kiel in der Uniklinik liegt und ihm alle guten Asterix und Lucky Luck mit gebracht und  nicht wiederbekommen. Die Mutter hat mit Conny geredet, sie, die Krankenschwester, und er sprachen eine Sprache von Tod und Leid, die Johanna nicht verstehen konnte. Eine Sprache, der Johanna nicht folgen kann, nicht folgen will.

Aber warum weiss die blöde Ramona, dass Conny tot ist und Johanna seine beste Freundin nicht? Sie sagt das, um mich zu ärgern, versucht  Johanna zu denken, aber sie sieht in den erschrockenen Augen Ramonas den Spiegel ihres an der Wahrheit gefrorenen Herzens.

Später am Abend wird die Mutter auf dem Marktplatz in Tränen ausbrechen, unsicher und hilflos gehalten vom Vater, ohne ein Wort über Conny zu Johanna. Johanna steht allein\dots Warum weine ich nicht? Fragt sie sich.
Aber wohin mit  dem Schmerz, wenn die Erwachsenen weinen\dots

Keine Beerdigung, die Mutter geht hin, aber sie will Johanna den Schmerz ersparen und fährt allein. Keine Beerdigung, keine Abschied von Conny für Johanna. Ihr erster Mann verschwindet, bevor sie erwachsen geworden sind. Hat sich langsam und still aus ihrem Leben geschlichen. 

Ob sie ein Paar geblieben wären, später? Die letzten Male, wo Conny Johanna besucht hat, ist es anders gewesen. Aber -alles ist anders gewesen, denn Johanna wohnt nicht mehr in Oldendorf. 

Und wenn Conny zu Besuch kam, wollte er gleich wieder heim, weil er Heimweh hatte und Johanna war wütend, weil sie neidisch war, dass Conny zurückkehren konnte und sie nicht. 

Die Zeit des klaren, einfachen, von Knicks geschützten Weges ist vergangen\dots



\part*{Teil II}
\addcontentsline{toc}{part}{Teil II}

















 
 
 
 
 
 
 \end{document}