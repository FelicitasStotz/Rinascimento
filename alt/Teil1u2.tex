\documentclass[10pt,a5paper]{book}
 \usepackage[ngerman]{babel}
 \usepackage[utf8]{inputenc}
 \usepackage{graphicx}
 
 \pagestyle{plain}
 
 %\frenchspacing
 
\parindent0em \parskip1.0ex plus0.5ex minus 0.5ex

%\topmargin0pt
\headsep0pt
\headheight0pt
\textheight14cm

\footskip1cm




 \author{von\\Johanna Mollydottir}
 \title{Rinascimento}
 
\newcommand{\sterne}{\par{\centering ***\par}}

\newenvironment{tg}{\begin{quote}\em}{\end{quote}}
\newenvironment{dichter}{\begin{flushright}}{\end{flushright}}

 \begin{document}
 
 \maketitle
\tableofcontents
 
 

\section*{Epilog}
\addcontentsline{toc}{section}{Epilog}



„Eine Gabe ist eine Aufgabe.“
\begin{dichter} Käthe Kollwitz\end{dichter}

Sind Sie auf die Welt gekommen, vergessend, was vorher war?

Sie leben,  atmen, erfahren, begreifen Ihr Dasein als eine Abfolge des sich fortfahrenden „Jetzt“, beginnend mit den ersten Erinnerungen Ihres Lebens- einzigartig, neu und einmalig?

Sie leben und erinnern sich nicht? Welche Gnade.

Bei mir ist es anders, dieses  mal.

Denn ich erinnere mich.

Meine Erinnerung ist gross, sie quillt hervor aus allen Ritzen meines Seins.

Glauben Sie an Reinkarnation?

Stellen Sie sich vor, Sie haben das eine oder andere Leben gelebt, \dots und würden sich daran erinnern. Sie erinnern sich ein Feldherr zu sein, \dots Sie riechen den Schweiss und das Blut, \dots spüren den Triumph des Sieges, den Rausch der Macht\dots, denken an die Freuden, die die Weiber bereiten, \dots und finden Sie sich wieder in dem Körper einer Frau, mittleren Alters, verheiratet, zwei Kinder, Auto, Haus und Hund\dots

Wer sind Sie?

Wie viel des einen Lebens gehört zu dem anderen?

Wieso leben Sie dieses?

Ach, seliges Vergessen.

Sie „glauben“ nicht an Reinkarnation?
 
Na, und?

„Glauben“ Sie an den Eiffelturm?

Warum? Haben Sie ihn selbst gesehen? Wenn nicht, woher „wissen“ Sie von seiner Existenz? Von Bildern? Erzählungen anderer?

Dabei ist es nicht schwer sich zu erinnern. Wir erinnern uns ständig, nur, wir wissen nicht, dass wir uns an ein vergangenes Leben erinnern, glauben, es seien Motivationen, Konsequenzen des jetzigen Lebens, aus denen heraus wir handeln.

Was, wären wir ein Puzzle. Ein Puzzle unserer vergangenen Leben, das sich neu fügt, neue Teile einfügt, die das Bild erweitern. Erweitern für das nächste Puzzleteil, bis es vollkommen und harmonisch ist, sich alle Teile zu einer Einheit fügen.

Wo beginnt die Geschichte, die Verschlingungen und Lösungen? Wo beginnt es, wenn die Vergangenheit vorbei ist, sich selbst aber wieder und wieder neu gebärt ins Jetzt. Verändert sich die Vergangenheit nicht mit jedem Gedanken, der sie neu beschwört? Sie verändert mich im gegenwärdtlichen Sein, wann immer ich ihr Raum gebe und sie verändert die Räume in mir, die sie betritt. In diesen Gedanken öffnet und verschliesst sie Türen. Türen, die die eine Zukunft öffnen oder schliessen. Es gibt keine Vergangenheit und die Zukunft entspringt immer neu der Kreation der Gedanken.

Die Flut und Ebbe, der Sturm und die ruhige Regung von Gedanken und Emotionen erschaffen uns sekündlich neu. Es gibt für den Körper Zeit und Raum, den Anker, in der orientierungslosen Flut der Gedanken. Alles ist Maya! Maya, die grosse Illusion. Trug- und Wahrbild. Theater, Theater der Seele!
Vorhang auf für Maya.



\part*{Teil I}
\addcontentsline{toc}{part}{Teil I}


\section*{Paradies I (Oldendorf, Dithmarschen 1975-1977)}
\addcontentsline{toc}{section}{Paradies I(Oldendorf, Dithmarschen 1975-1977)}



„Du bist die Frau und ich bin der Mann! Und nun machen wir noch ein Kind.“ „Okay!“ Johanna legt sich auf die Matratze. Sie ist aufgeregt.

Es ist heiss auf dem Dachboden. Die Sonne scheint in das Fenster und taucht den Giebel mit dem dunklen Holz ins Licht. Und die Hitze, die sich dort zur Mittagsruhe gelegt hatte, hüllt die Haut mit zwickenden Fingern ein.

Gleich bekomme ich ein Kind, denkt Johanna und ist besorgt, denn sie weiss nicht, wie das Kind in ihren Bauch passen wird.

Conny legt sich ganz vorsichtig auf sie. Die Bäuche berühren sich und zwischen den Beinen spürt Johanna das weiche Glied des Freundes ihre Vagina berühren. Nein, nein, mit Kindern will Johanna es nicht versuchen.

Beide springen auf, auseinander mit klopfendem Herzen, ist es zu spät? Aber Johannas Bauch ist wie zuvor. Schwitzend sehen sie sich an. Und die Hitze zwickt mit langen Fingern, die nackte Haut und sie spüren sich nackend, sie spüren, der Mann und die Frau, sie warten.

„Aber wir heiraten später, wenn wir gross sind!“ „Ja, und ich werde Müller und habe eine eigene Mühle auf dem Hof und einen Esel!“ „Mit der Mühle ist in Ordnung, aber ich weiss nicht, ob sich meine Kühe mit dem Esel vertragen?“ „Aber Müller brauchen einen Esel!“ „Na, gut.“

Johanna und Conny steigen auf ihre Fahrräder und fahren zu Connys Bauernhof. Der Hof liegt einsam eingebettet in den satt grünen, von Knicks umrandeten Koppeln.

Ein kleiner Hof mit dreissig Kühen, Kälbern und einigen Feldern ringsum. Einer Grossmutter, die kaum ihren Altenteil verlässt und griesgrämig die dünnen Finger aneinander  reibt. Sprechend dicht mit ihrem dünnen, scharfkantigen Gesicht heranrückt und spitze Bemerkungen ausstossend wie ein Huhn zurück ruckt, das Gesicht verzogen.

Die Bäuerin kräftig mit geröteten Wangen, dunklen, kurzen Locken, ist grösser als ihr Mann Klaus. Sie liebt es Schauerliches herauf zu beschwören und dadurch auf Dinge hin zu weisen, die es zu erproben gälte, um für das Leben vorbereitet zu sein.“Aua, aua, mein Zahn, die Kuh hat mich wieder getreten“, jammert sie. Vormelken muss sie, damit die Pfropfen der Melkmaschine die Milch aus den Eutern saugen können.

Bei den gutmütigen Kühen üben sich die Kinder, aber Johanna schafft es nicht, der rosigen, rauen, wackeligen Zitze einen Tropfen Milch zu entlocken. „Drücken und ziehen musst du.“ Conny lacht, während zwischen seinen Fingern die Milch aus dem Euter spritzt. Wie gut, dass ich ja Müller werde, denkt sich Johanna, der Conny muss seine Kühe selbst melken.


\section*{Tod I}
\addcontentsline{toc}{section}{Tod I}


Vor dem Bauern hat Johanna Angst, der spricht selten und poltrig, laut und weiss von Schlimmeren zu berichten als seine Frau. Er verengt seine wässrigen, hellblauen Augen zu Schlitzen und sein Gesicht ist feuerrot, umrahmt von gelben, zerzausten Haaren.

Er lässt die Kinder auf dem Trecker mitfahren. Hinten angehängt eine Maschine um Löcher für neue Zaunpfähle zu bohren. „Wenn du da, an diese Stange mit dem Kopf stösst, gehst du tot.“ Der Bauer schaut Johanna scharf an. Zitternden Leibes sitzt Johanna auf ihrem metallischen, Kissen gepolsterten Sitz, die Finger um die Stange gekrallt, die sie mit dem Tod bedroht.

Der Trecker ruckt und kippt, während er über die Koppel fährt und der Bohrer sich in die Erde dreht.

Da! Der erste Schlag an den Kopf. Johanna blinzelt überrascht, sie ist nicht tot\dots Aber sie ist sicher, sie wird es bald sein. Der Trecker springt und bockt wie ein junges Fohlen, unvorhersehbar und der Kopf springt wieder und wieder gegen die Stange. Wie lange wird es dauern\dots Johanna taucht in sich selbst ein, so weit es geht. Jetzt ist eine Schutzschicht wie dickes, wässriges Glas zwischen ihr und der Welt, sie kann Conny und den Bauern nicht richtig sehen und hören, aber dort ist es sicherer.

Warum lacht Conny? Die Freude kann Johanna nicht verstehen. Der Freund ist ganz unbekümmert und scheint keine Gefahr zu spüren.

Am Ende der Fahrt ist Johanna enttäuscht, sie lebt und fragt sich, ob sie um eine Erfahrung betrogen wurde oder nur um den Glauben in die Worte des Bauern.



\section*{Paradies II}
\addcontentsline{toc}{section}{Paradies II}



Es ist ein beschauliches Dasein. Der Geruch der Marsch hüllt alles ein. Moorig, wässrig vom warmen Sonnenlicht ausgelöst, verwandelt sich die schwere Erde in kräftigen, säuerlichen-süssen, holzigen Geruch, der ummantelt, nicht durchdringt. Durchbrochen von den zahllosen Knicks, die mit Büschen, losem Strauchwerk und kleinen Bäumen die Wege und verschiedenen Koppeln säumen, gibt das flache Land grosszügig die Himmelskuppel frei. Eine grünes Brett unter einer hellblauen, dunstigen Käseglocke. 

Eine aquarellige Nass-in-Nass Landschaft, dominiert von den Himmelsvariationen und dem, weit in die Ferne in immer hellerem, verwischt Geschichtetem, sich verlierenden Blick. Mal zeigen sich weisslich, milchig, von einer verschleierten Sonne durch lichtete Wolken, der Maler hat das nasse Blau mit einem trockenen Pinsel durchfahren und die beiden unterschiedlichen Flächen machen sich miteinander vertraut. Mal strahlt Blau. Mal ziehen in endloser Formation, auf Schnüren hintereinander aufgeperlt, dicke, flauschige,  von unten platt gedrückt und grau, Cumuluswolken mit dem Wind aus dem Westen.

Die Kinder sind ganz für sich. Es braucht unter der Käseglocke keine Erwachsenen. 

Der kleine Fluss hinter dem Wäldchen bei Johannas Haus ist tabu. Und der Garten der einzigen Nachbarn. Vor dem Haus ist die Landstrasse, die nach Oldendorf führt, dahinter lehmig, gelb die Kiesgrube. Dort sind die Kinder selten. Neben dem Haus ist ein Wäldchen und auf der anderen Seite das gleiche Haus mit den Nachbarn, Auffahrt, Feldweg zu Connys Hof. An der Landstrasse dann ein kleines Stück weiter das Kraftwerk indem der Vater arbeitet.



\section*{Humunculus}
\addcontentsline{toc}{section}{Humunculus}



Gasturbinenkraftwerk. Zwei grosse Klötze nebeneinander mit drei grossen, silbrigmatten Schornsteinen, die weithin sichtbar nach jeder Ausfahrt das Zuhause ankündigen. Zwei Gastanks, rund sind sie und riesig. Es ist eine summende, brummende, vibrierende Welt. Sie riecht nach Öl und Metall, der Geruch beisst in der Nase wie der vom Kuhmist. Aber dieser Geruch fühlt sich an wie kleine, tote Nadeln, und der ranzigen Schwere von Erdöl, dem Leichensaft von Mutter Erde, er hüllt nicht ein wie ein müffelnder, warmer Pelz, sondern tastet vielmehr ab, er durchdringt und sucht,\dots Dann gibt es den Geruch vom Büro, von altem Papier, Arbeitsanzügen, Holzschränke, Kaffee und dem kalten Rauch von Papas Pfeife, der ist wie Heimat.

Wenn Johanna hinter den haushohen Turbinen in die hintersten Winkel der Werkhalle schleicht, es, trotz Beleuchtung, dunkler wird, die Füsse auf dem Laufgitter einen mehrstimmigen, federnden Klang erzeugen, die Rohre, gelb und rot lackiert, die Ventile, Rädchen und Räder zum drehen, aus ihren Winkeln vorrücken, Gesichter bekommen und sie anstarren, dann wird sie vom Grauen ergriffen und rennt zurück zum Eingangstor, das die ganze vordere Seite der Halle ausfüllt und offen und grosszügig Tageslicht herein lässt. Der Vater wäscht seelenruhig das Auto oder bastelt daran, denn Johanna ist nur im Kraftwerk, wenn es Wochenende ist und der Vater sich seinem Auto widmet. Der Vater weiss nichts, von den geheimnisvollen Wesen, die in  der hinteren Ecke der Halle lebendig werden. Und im Tageslicht betrachtet, findet Johanna selbst, dass dort ausser Maschinen und Rohren nichts sein kann,\dots Sie schleicht zurück, um sich zu beweisen, dass alle Dinge Dinge sind, steife, starre Dinge\dots

Fasziniert beobachtet Johanna, während sie sich erneut an schleicht, wie die Angst in ihrem Inneren wächst und wächst, mit den beruhigenden Gedanken und dem Vertrauen ringt und plötzlich vorschiesst und sie mit Entsetzten ausfüllt und dann rennt sie auch schon zurück. Je mehr sie sich zur Vernunft zwingt, je langsamer sie Schritt für Schritt ins Dunkle geht, sich genau umschaut und alles betrachtet, um so heftiger kommt die Angst.
Es ist ein tolles Spiel.



\section*{Credo I}
\addcontentsline{toc}{section}{Credo I}



Abends ist Johannas liebste Zeit auf der Schaukel. Wenn sie auf der Schaukel schwingt und die  Sonne langsam den Weg zum Horizont nimmt, der Himmel heller als am Tag weiss-dunstig sich bereit macht die Farben des Abendrotes auf zu nehmen. Die lang gezogenen Wolkenstreifen zartes Rot und Orange, zu Beginn hellstes Goldgelb, Farbklängen gleich, in die Weite musizieren. Mit dieser Musik vibriert das Herz, immer stärker und tiefer. Die Schaukel schwingt den Bauch, lockert ihn, öffnet jede Zelle, damit das Licht eindringen kann. Heben und senken, der Bauch, der Körper werden schwerer und ruhiger, bis sie selbst aufgelöst mit dem Licht, den Farben klingen. 
Nichts berührt den Kopf, der darf still sein, der Rest ein riesiges Lauschorgan, selbst Wellen der Zufriedenheit verströmend. Im Kern der Engelsmusik, sie spiralt gleichermassen darum, die Leere, all-liebende, alles und nichts seiende Leere.

Ein Auto kommt auf der Landstrasse hinter dem Jägerzaun vorbei gefahren, von der Kupplung gebogenes Motorengeräusch, das schnaufende Auto wird in seiner Fahrt gebremst, Ortseingang. Der höher sich verbreiternde Autoklang vervollkommnend die Musik um Endliches. Das Auto ist Bewegung -bewegen, Weg.
 


\section*{Er-inner-ung}
\addcontentsline{toc}{section}{Er-inner-ung}



„Mama, Mama, schau mal mein Bild an! Mama wach auf, schau!“ Johanna ist zu den schlafenden Eltern ins Schlafzimmer gestürmt und springt auf das grosse Bett. Die Mutter richtet sich verschlafen auf: „Was?“ „Schau doch, wie ich gemalt habe!“ Der Mutter wird ein Blatt mit kleinen Figuren unter die Augen gehalten. „Mmh? Ja, schön hast du das gemacht.“ „Ich hab die Haare anders gemacht und jetzt sehen sie wie richtige Menschen aus!“ „Prima, komm, geh leise wieder raus, bevor der Papa auch geweckt wird. Ich komm bald und mach Frühstück.“ Johanna geht leise, verzückt aus dem Schlafzimmer und malt gleich ein weiteres Bild auf dem die Menschen viel mehr wie Menschen aussehen als vorher. Johanna hat lange probiert und verschiedenes getestet, sie ist glücklich und stolz, dass sie eine neue Methode gefunden hat, Menschen gut zu zeichnen.

Wenn die Eltern am Wochenende länger schlafen, sitzt sie oft und malt, die Mutter hat ihr dann eine Schüssel mit Kellogs Smacks hingestellt, die knappert sie. Erst jedes einzelne, wie eine Kostbarkeit, dann mehrere, eine Handvoll und zum Schluss hält sie sich die Schüssel an den Mund und mampft die süssen Körner gierig in sich hinein. Sonst weckt sie die Eltern nicht, aber an diesem Tag gelingt ihr zum ersten mal ein kleines Wunder, sie schafft es Dinge zu zeichnen, wie sie sie sich vorstellt und sieht. Bei diesem ersten Mal-Erlebnis ist Johanna vier Jahre alt.



\section*{Paradies III}
\addcontentsline{toc}{section}{Paradies III}



„Komm, der Mist fährt auf den Misthaufen.“ Johanna und Conny stürmen aus dem Kuhstall. In der Rinne hinter den Kühen fahren die Schieber hin und her, klappen sich ein und aus und schieben die grünen, breiigen Haufen Stück um Stück auf das Förderband zu. Dieses steht hoch aufgerichtet vor dem Misthaufen und vergrössert diesen portionsweise um weiteres Grün.
Erwartungsvolle Spannung, wie gross wird der Haufen, der als nächstes herabstürzt, sein? Wie weit wird er spritzen?

Es ist eine Frage des Könnens, den Haufen, der hochgefahren kommt, einzuschätzen, im richtigen, aber letzten Moment vor den grünen Spritzern zu flüchten.

Im Inneren des Stalls fressen die Kühe gasend ihr Abendheu, während sie gemolken werden. Es muht von den Kühen, es klirrt von ihren Bügeln, durch die sie ihre Köpfe schieben, die sich dann um ihren Hals schliessen. Johanna findet es traurig, dass die Kühe sich, sind die Bügel geschlossen kaum bewegen können. Weiter surrt die Melkmaschinen und gibt gleichzeitig ein klackerndes Geräusch wieder, mit dem die Milch rhythmisch durch die durchsichtige, verstaubte Leitung pulsiert. In der Milchkammer steht eine von Connys Schwestern und giesst von der frischen Milch in Eimer. Die bringen Conny und Johanna den kleinen Kälbern. 

„Schau, meins ist schon ganz zahm!“ “Meins auch!“ Sie müssen die Eimer gut fest halten, denn die hungrigen Kälber stossen kräftig ihre Köpfchen in den Eimer um jeden Tropfen Milch zu schlecken, hinten wackeln und zittern die kleinen Schwänze. Nach der Trinkzeit lassen die Kinder die Kälbern eine Weile an ihren Fingern saugen. Die raue Zunge schmiegt sich an die Finger, die oben von dem zahnlosen Kiefer geknetet werden. Es schmerzt. Johanna freut sich aber, denn sie kann auf  das kleinen Kalb Acht geben. Mit dem Eimer in der Hand laufen die Kinder in die Tenne, jeder macht einen Griff in das Schrot und ab in den Mund damit. Im Winter läuft die Häckselmaschine, die aus den Futterrüben kuh- und kindgerechte Schnitze macht, die bei laufender Fahrt aus der Maschine geklaubt werden.



\section*{Tod II}
\addcontentsline{toc}{section}{Tod II}



Im Winter spielen Johanna und Conny in der Scheune verstecken. Sie ist bis unter das Dach mit Heu und Strohballen gefüllt. Die Kinder sollen das Stroh nicht durcheinander bringen und damit spielen- unmöglich. Für kleine Höhlen reicht das lose Stroh.

„Komm, wir springen von den Ballen dort oben!“ „ Ich traue mich von dem dort, der ist höher!“ Conny steigt höher und Johann will nicht nachstehen. Ein Moment hebt sich der Körper empor, schwingt mit der Luft und kehrt zurück auf die feste Erde. Sie können nicht genug bekommen und höher, höher\dots Der Aufprall staucht die Beine, der Schmerz wird grösser, die Luft wird aus den Lungen geschleudert, wenn sich der Leib fester und fester nach dem Höhenflug an den Boden presst. Es geht um die Ehre.

Conny springt. Bleibt liegen. Johanna klettert schnell herunter. Conny liegt da und rührt sich nicht mehr, die Augen sind geschlossen.

Johanna stürmt ins Haus und holt den Bauern und Bäuerin. Die Angst kommt mit, die Angst , die kribbelige, die erregende Angst vor dem Tod.

Conny ist wieder da. Die Eltern erleichtert, stürmen auf Johanna ein: „Conny hätte sterben können, du weisst, dass das Verboten ist! Das dürft ihr nicht\dots“ Johanna ist getroffen und verletzt, warum schreien der Bauer und die Bäuerin mit ihr herum, schliesslich war es Conny, der sich gestossen hat. Es ist nicht recht. Johanna kneift fest die Lippen zusammen und sagt nichts. Aber das Gewissen nimmt ein Stück Schuld zur Untermiete ins Herz.



\section*{Gnomus}
\addcontentsline{toc}{section}{Gnomus}



Am Abend fährt Johanna heim. Allein. Vorn, am Fahrradlenker, baumelt eine Tupperkanne mit frischer Milch. Der Weg besteht aus zwei Betonplattenspuren, getrennt von Gras. An den Seiten hat es Gras und an beiden Seiten Knicks mit Büschen und Sträuchern, kleinen Bäumen. Ein einfacher Weg, ein klarer Weg. Hinter der Kreuzung kommt das kurvige Stück. Johanna fürchtet sich vor dem kurvigen Stück, ihr ist jedes mal, als werde sie beobachtet, als wollte etwas hinter dem Erdwall hervorspringen und ihr den Weg versperren. Es ist keine besondere Ecke, die dunkler wäre oder enger, es ist nichts zu sehen und doch müssen unsichtbare, wütende und raubende Wesen hinter dem Wall sitzen, auf Beute lauernd. Dort fährt sie, so schnell sie kann. Einmal, als sie es besonders eilig hat, rutscht der Reifen über die Betonplattenkante auf die ausgewaschene Grasspur, bleibt widerspenstig, schliesslich bockt das Fahrrad und Johanna fällt.

Sofort taucht Johanna unter, nichts hat mehr Platz unter ihrer Seelenwasserkuppel ausser namenloser Schrecken\dots Nicht den kleinsten Schmerz spürt sie, während sie mit fliegenden Händen den Lenker greift und im Laufen auf das Fahrrad springt und wild bis zur letzten Kurve strampelt. Auf dem letzten, geraden Wegstück spürt sie ihr Herz im Mund klopfen und bemüht sich es wieder runter zu schlucken. Es dröhnt in den Ohren. Die Knie melden sich mit dem Schmerz vom Aufprall,  langsam geht die Fahrt nach Hause. Links der hohe, dicht bewachsene Knick und rechts eine Koppel, die die Blick auf den Abendhimmel öffnet. Über dem Kopf, hoch oben, die Strommasten und Leitungen, die den Strom fort bringen, den der Vater im nahen Kraftwerk gemacht hat. Es summt in der Luft.



\section*{Il uomo nero}
\addcontentsline{toc}{section}{Il uomo nero}



Conny und Johanna steigen auf die Fahrräder, sie wollen zu Johanna fahren. Jeder auf einer der von Gras getrennten Betonplattenspuren. Der klare Weg, ihr Weg, den kaum mal ein Spaziergänger betritt ist heute anders, still, kein Vogel ist zu hören, die Bäume auf den Knicks zu beiden Seiten wirken vor dem nebelig-trüben Himmel dunkel, die Luft legt sich dick, schwül auf die Augenlider.
 Ihnen entgegen kommt eine schwarze, in eine lange Kutte gehüllte Gestalt. Es ist ein Mann mit schwarzem, zerzaustem  Haar, einem langen Bart und bleichem, blassen Gesicht. Als er die Kinder sieht, bleibt er stehen, breitet er die Arme über den Kopf und beginnt mit lauter Stimme in einer fremden Sprache zu sprechen.
 
Sie bleiben stehen. Die Stimme saugt an den Ohren, sie will die beiden Kinder einfangen. Nach dem ersten lähmenden Schrecken, drehen Johanna und Conny um und fahren mit fliegenden Beinen zurück zum Hof.

„Da, da kommt ein schwarzer Mann. Er kommt, er kommt!“ „Ein ganz schwarzer Mann, der ruft.“ Schreiend und keuchend kommen die Kinder in die Küche gestürzt. Das rote Tier Angst löst sich von ihnen und springt\dots Die Erwachsenen springen auf. „Du bleibst drinne!“, der Bauer schaut Johanna streng und besorgt an, die anderen stürmen aus dem Haus. Conny darf mit raus. Der einsam gelegene Hof hat zwei Einfahrten und Johanna sieht den Bauern zu der einen Einfahrt gehen, zu der, die dem Wanderer am nächsten liegt, einen Besen mit kräftigen Stil in der Hand. Die Bäuerin nimmt eine Mistgabel. Connys grosse Schwestern gehen zu der anderen Einfahrt. Es kribbelt in Johannas Bauch. Wer ist der Mann? Was wird der Bauer mit ihm machen? Kann er ihn besiegen?

Wut und Angst wandern mit den Erwachsenen in den Toreinfahrten mit. Das rote Tier, das von einem zum anderen hetzt und sich die Leftzen leckt, auf Futter wartete. Es möchte toben und Blut, es drängt sich zu den Menschen, umspringt sie. Ganz plötzlich sieht Johanna es und es ist gruseliger als der schwarze Mann, der allein den Weg herauf kommt. Johannas Herz klopft heftig, wenn der Bauer den Mann schlägt? Mit dem Besenstiel. Sie kann die Einfahrt nicht sehen. 

Johanna kommt es vor wie eine Ewigkeit, die sie alleine am Küchenfenster verbringt. Conny läuft auf dem Hof hin und her. Die Erwachsenen reden aufgeregt miteinander und schauen immer wieder den Weg hinunter. Dann wird das  Tier kleiner, unruhig dreht es sich im Kreis, entfernt sich langsam, von den Menschen, wird von einer unsichtbaren Wand von ihnen weg geschoben.
Schliesslich sammeln sich alle in der Einfahrt, die dem Wanderer am nächsten liegt.  

Von den Mann ist keine Spur zu sehen! Er kommt nie an der Einfahrt an!
Wohin ist er auf dem schmalen, von Knicks und Feldern umgebenen Weg verschwunden\dots

Die Erwachsenen verstauen Besen und Mistgabel und lachen befreit, vorwurfsvoll, was Kinder für eine blühende Phantasie haben\dots Widerwillig verschwindet das rote Tier einen Rest Wut zurücklassend und Scham, die Erwachsenen können sich nicht in die Augen sehen.



\section*{Credo II (Oldendorf, Dithmarschen, 1976)}
\addcontentsline{toc}{section}{Credo II (Oldendorf, Dithmarschen, 1976)}



„Natürlich gibt es den lieben Gott!“ sagt Conny. 

„Bist du sicher?“

„Ja, der wohnt im Himmel und hat einen langen Bart. Der sitzt dort auf einer Wolke und wenn man betet, dann hört er das. Und er sieht alles, was du machst.“

„Und das Paradies?“  

„Das ist im Himmel, dort kommt man hin, wenn man stirbt. Oder in die Hölle, dort brennt es. Da kommen die Bösen hin.“ 
„Kann jeder mit dem lieben Gott reden?“ „Ja, man sagt, lieber Gott im Himmel und dann was man möchte\dots“


„Wenn ich Tod bin, bleibe ich nicht im Sarg liegen?“


„Nöh, dann kommst du zum lieben Gott\dots“

Ich liege nicht im Sarg als würde ich tief schlafen, denkt Johanna. Ich muss nicht im Sarg liegen und die Würmer kommen und fressen mich, während ich schlafe. Ich muss nicht im Sarg liegen, während die Würmer meinen Körper fressen und ich nicht träumen und nicht aufwachen kann\dots Ich kann nicht aus versehen aufwachen, wenn die Würmer meinen Körper zerfressen\dots

Wenn ich sterbe, darf ich weiterleben\dots bei dem lieben Gott im Himmel.

So glaubt Johanna still und heimlich, während ihr Kopf verzweifelt versucht den mütterlichen, kommunistisch-atheistischen Vorstellungen des Todseins zu folgen.


\section*{Prinzessin (Oldendorf, Dithmarschen, 1975-1977)}
\addcontentsline{toc}{section}{Prinzessin (Oldendorf, Dithmarschen, 1975-1977)}



Johanna greift in die Verkleidungskiste und holt ein glänzendes Kleid hervor und die schicken Sandalen mit Riemchen und kleinen Absätzen, solche, wie sie grosse Frauen tragen. Damit stolziert sie im Garten auf und ab.
 Wunderbar. 
 
 „Mama, darf  ich die Sandalen zum Einkaufen anziehen?“ „Nein, Johanna, das weisst du doch! Die Schuhe sind nicht gut für deine Füsse, weil die Einlagen nicht hineinpassen.“ “Aber Oma Frankfurt hat sie mir geschenkt.“ „ Johannaaa, nein!“ Johanna findet das ungerecht, warum darf sie keine richtigen Mädchenschuhe anziehen, nur diese hässlichen Sandalen, die hinten geschlossen sind, Babyschuhe.
 
Faschingszeit im Kindergarten. Plötzlich heisst es sich verkleiden. Johannas Mutter hat ihr ein Funkenmariechen Kostüm gekauft. Aber Johanna weiss nicht, was ein Funkenmariechen ist. Die rote Jacke mit den goldenen Knöpfen, sieht aus wie die eines Soldaten und das weisse Faltenröckchen sieht wie ein normaler Rock aus. Der roten Hut, ein Dreispitz, sieht auch aus wie für Soldaten. 

Aber Johanna darf die Sandalen mit Riemchen anziehen. 

Johanna bekommt einen Beutel mit Bonbons. Sie darf sie aber nicht essen. Ein Funkenmariechen wirft die Bonbons in die Luft, damit die anderen sie fangen und essen können. Das widerstrebt Johanna, warum soll sie all die Bonbons weggeben. Johanna versteht nichts, nicht, als was sie verkleidet ist und nicht, warum sie Bonbon zum Wegwerfen bekommt.

Aber, dass ihre Freundin Britta als Prinzessin zum Fasching darf, das versteht Johanna. Wie hat Britta das gemacht? Sie hat  nicht nur ein langen, goldgelben, glitzernden Rock und ein goldiges Oberkleid, nein, sie hat sogar ein kleines, glänzendes Krönchen mit einem Schleier auf dem Kopf. Und Britta schreitet umher und vergisst nicht, den Schleier, als ob er ihr langes Prinzessinenhaar wäre von der einen Schulter zu streichen und dann von der anderen zu streichen und den Kopf nach hinten zu werfen. Wie gern wäre Johanna eine Prinzessin\dots

Die Kinder tanzen und springen. „Johanna, du musst noch die Bonbons werfen und verteilen“, die Mutter hat den Beutel mit den Bonbons in der Hand. Johanna wird es heiss und dann läuft ein kalter Schauer den Rücken runter. Sie schämt sich und sie hat Angst. Wie soll sie das machen? Alle Kinder werden sie bedrängen, schreien und nach den Bonbons grabschen. Johanna möchte am liebsten den Sack, gefüllt wie er ist, mit nach Hause nehmen und das nicht, weil sie die Bonbons essen will. 

Die Mutter macht ihr Mut: “Los, Johanna, wirf` die Bonbons einfach zu den Kindern“. “Wie soll ich das den machen?“ Johanna ist verzweifelt.  „Du musst doch nur werfen!“, sagt die Mutter fassungslos, sie versteht nicht, warum Johanna es nicht macht. 

Die Bonbons fliegen, an die Wand und unter die Turnbänke, die rundherum aufgestellt sind. „Johanna, du sollst die Bonbons zu den Kindern werfen. Am Rand und unter den Bänken findet sie keiner“. Es dauert eine Zeit bis die anderen merken, dass Johanna mit Bonbons wirft. 

Dann kommt Silvia, die Tochter von der strengen Aushilfskindergärtnerin, die nur zu ihrer Tochter nett ist und alle anderen Kinder ausschimpft. „Darf ich auch Bonbons verteilen?“ Johanna kennt Silvia nicht gut, sie spielen an den zwei Tagen, an denen Johanna im Kindergarten ist, nie miteinander. Aber Silvia ist die Tochter von Frau Potthof und deshalb hat sie das Sagen bei den Kindern.  Silvia lächelt und wirft elegant die Bonbons. Die anderen Kinder drängen sich um sie und sie verteilt die Bonbons mit vollen Händen, als wäre sie das Funkenmariechen. Der Sack ist schnell leer, dass für Johanna kaum welche übrig bleiben. Silvia lacht und geht wieder Hüpfen und Tanzen, wie die übrigen Kinder. 

Und Johanna? Sie taucht in sich unter, ist froh, dass sie der Aufmerksamkeit entgangen ist und traurig, denn, die Mutter ist unzufrieden. Johannas Gewissen meldet sich, du hat es falsch gemacht und die Gier meldet sich, du hast nicht einen Bonbon abbekommen. 


\sterne


Es bietet sich eine neue Gelegenheit eine richtige Prinzessin zu werden. Eine Märchenprinzessin. Im Kindergarten werden beim Sommerfest Märchen aufgeführt. Johanna darf die Prinzessin im Froschkönig sein. Die Kindergärtnerin sitzt mit Johanna an dem Brunnen aus Pappe. „So, Johanna, jetzt stell dir vor, du bist die Prinzessin und dir ist deine liebstes Spielzeug in den Brunnen gefallen. Und nun weinst du!“ Die Kindergärtnerin sieht Johanna erwartungsvoll an. Johanna steht an dem Pappebrunnen, die anderen Kinder sehen zu. Johanna muss lachen und schaut die Kindergärtnerin hilflos an. „Ich zeig`dir wie es geht.“ Die Kindergärtnerin schlägt die Hände vor ihr Gesicht und gibt schluchzende Geräusche von sich:“Huuuuhuhuhuuuh!“ Am liebsten möchte Johanna wieder lachen, aber sie will sich Mühe geben, damit sie die Prinzessin vom Froschkönig sein kann. Johanna spürt wie sich ihr Körper verändert. Sie fühlt ihn stärker und stärker, gleichzeitig verschwimmt ihr die Kindergärtnerin. Sie versucht es:“Huhu, \dots“ Das Gesicht verrutscht und im Bauch ballt sich ein Glucksen zusammen. Johanna lacht, sie will nicht, aber sie muss lachen.
Nein, nein. Die Kindergärtnerin erklärt geduldig, Johanna wäre traurig, genau wie die Prinzessin und sie sollte es noch einmal versuchen.

Johanna wird es ungemütlich, sie möchte Prinzessin bleiben, aber tun, als ob sie weint, wenn sie nicht weinen muss, kann sie nicht. Warum, warum nur, kommt stattdessen ein Lachen aus ihr raus? Die Enttäuschung der Kindergärtnerin kribbeln in Johannas Bauch. Sie fühlt wie ein Stück von ihr tiefer  in sie hinein gesogen wird und sich in einen dunklen Winkel verkriecht. Johanna sitzt in der Taucherglocke. Johanna lacht mehr und mehr, sie kann nicht aufhören, als würde sie weinen. Dann ist Schluss mit dem Froschkönig. 

Johanna bekommt, als einzige, eine neue Rolle. Jetzt ist sie das Dornröschen. Sie weiss, dass die Kindergärtnerin sie zum Dornröschen bestimmt hat, weil sie dabei still mit geschlossenen Augen da liegen muss, das Einfachste auf der Welt\dots Aber, Johanna ist Prinzessin geblieben, eine mit einem unruhigen Gewissen, aber Prinzessin\dots

Zur Aufführung bekommt sie ein schönes Kleid und eine Krone. Die anderen Kinder singen das Dornröschenlied und tanzen um Johanna und ihre Kinderdornenhecke herum. Sie muss lachen, es kitzelt wieder im Bauch, diesmal, weil sie sich freut und dann weil es laut und lärmig ist und Johanna aufgeregt ist und sich wieder untergetaucht anfühlt. Sie kann nur noch ein Drehen im Bauch spüren, die anderen um sie herum, sieht und hört sie wie durch Wasser, weit weg und sie ist plötzlich ganz allein\dots Und deshalb lacht das Dornröschen statt zu schlafen und als der Prinz kommt, da springt es auf und hopst und springt wild mit ihm herum, dass ihm hören und sehen vergeht. 

Einige Wochen später merkt Johanna wieder wie schwer es ist, Prinzessin zu sein und: Johanna ist keine Prinzessin, nicht mal eine schlafende. Die Mutter hat die Fotos von der Aufführung abgeholt und lacht. Denn das Johannadornröschen liegt kichernd und mit aufgestellten Knien auf ihrem Prinzessinenbett, so dass jeder ihre Unterhose unter dem Kleid sehen kann\dots


\sterne


Jedenfalls will Johanna Klonschen haben. Die schicken Schuhe mit der Holzsohle, die alle tragen. Sie bittelt und bettelt und tatsächlich. Es sind nicht die, die sie sich gewünscht hat, sie darf die mit dem Riemen haben. Mit den neuen Schuhen geht sie hin und her, schaut auf ihre Füsse und spürt die harte Sohle am Fuss. Es fühlt sich grossartig, ungewohnt und ungemütlich an. Die Riemchen schiebt sie oben auf den Schuh und dann wieder hinter die Ferse, wieder auf den Spann, wieder an die Ferse, sie kann sich kaum satt sehen. Mit solchen schönen Schuhen muss sie in die Welt, Conny besuchen.

 „Aber nicht mit den Klonschen“, sagt die Mutter. „Bei Conny musst du andere Schuhe anziehen.“ Johanna kann es nicht fassen, sie hat endlich richtige Schuhe und darf damit nicht zu Conny gehen. „Ich pass` doch ganz doll auf.“ verspricht sie. „Aber dann musst du das Riemchen hinten tragen“. Johanna steigt auf das Fahrrad. Ein bisschen böse ist sie schon, wegen den Riemchen, die sind für Babys.  Dennoch,  sie kommt sich wie ein echtes Mädchen vor, eine echte Frau. 
 
Conny zeigt sich wenig beeindruckt über Johannas Schuhe, er hat andere Pläne und es geht über die Koppel im Galopp. „Warte“ schreit Johanna Conny nach, denn sie lernt, dass  Mädchenschuhe den Füssen einige Schwierigkeiten bereiten können, wenn es um Schnelligkeit geht. Conny ist beim Stall angelangt, Johanna hetzt hinter drein. Ein stechender Schmerz fährt ihr in Knöchel, der über einer alten Flasche, die gut versteckt im Gras die Sonne geniest, davon, zur Seite rollt und sie dabei zu Boden wirft. Johanna springt wieder auf, aber sie kann nur humpeln, der Knöchel ist dick geschwollen als sie heim kommt. „Siehst du, ich habe es dir ja gesagt!“ Ja, das hat die Mutter getan und Johanna hat bei der Strommassage, die der Knöchel über Wochen bekommt, Gelegenheit darüber nachzudenken, wie sinnvoll es ist auf die Erwachsenen zu hören. Schuldig, sagt das Gewissen. Warum kann Johanna kein richtiges Mädchen sein?


\sterne



Wenn Johanna mit den Freunden, den Jungen, unterwegs ist, dann fühlt sie sich wirklich als Prinzessin, oder als Mädchen. Und das ist gut. Denn Johanna ist ein starkes Mädchen, eines, das mit den Jungen mithält, keine Heulsuse. Die Jungen, das sind Sönke, Holger, Kai, Heiko und Conny.
Die Jungen sind alle miteinander verwandt und Johanna ist die „Henne“ im Korb. 

Der Sönke, der ist das Mädchen bei ihnen, denn er heult schnell und kann nicht schnell rennen. Und auch beim Weitpinkeln ist er der letzte.- Da kann Johanna nicht mitmachen, aber das macht nichts. Schliesslich braucht es einen verlässlichen Schiedsrichter, der schaut, wer es am weitesten geschafft hat.- Und weil Johanna genauso gut ist wie die Jungen und der Sönke das Mädchen ist, deshalb können auch Kai, der kleine Bruder vom Heiko und Holger, der kleine Bruder vom Sönke und vom Helge, der aber nicht mehr mit ihnen spielt und viel älter ist, nichts sagen. Denn den beiden ist es nicht immer recht ein Mädchen dabei zu haben.

Johanna ist gerne ein starkes Mädchen, sie misst sich mit den Jungen, will die kleinen Wettkämpfe gewinnen. Sie will den Mut und ihre Kraft trainieren. Ein kleines Stück über sich hinaus wagen, und schaffen! Instinktiv spürt sie dabei ihre unbeugsame Stärke, die eigene Verlässlichkeit und, dass beides lebenswichtig ist.

Es gibt viel zu erleben. Bei Conny auf dem Hof und auf den Koppeln, bei Johanna im Garten. Manchmal machen sie Fahrradtouren. Zu dem kleinen, viereckigen Tannenwald, in den sich aber nur die Mutigsten wagen, weil der Bauer, dem es gehört, dort mal ein totes Kalb mit einem Strick um den Hals liegen gelassen haben soll. Es heisst, der Bauer lauere mit der Schrottflinte und würde auf jeden schiessen, der seinen Wald betritt.

Einmal sind sie bei Kai und Heiko zu Hause. Dort sitzen sie ganz artig in der guten Stube und spielen Mensch-ärgere-dich-nicht. Es ist allen ungemütlich. Und sie wissen nicht richtig, wie sie sich benehmen sollen. Im Haus, am feinen Stubentisch fühlen sie sich erwachsen und versuchen sich zu benehmen, das ist langweilig. Aber keiner sagt es. 

Johanna lernt Neues kennen. Sie, die mit dem Conny verlobt ist, bekommt einen Verehrer. Heiko, der älteste von allen, er ist schon 7 Jahre alt, merkt, er ist in Johanna verliebt. Johanna findet das mal gut, nämlich dann, wenn einer von den kleinen Jungen über Mädchen schimpft und dann vom Heiko einen Rüffel kriegt manchmal ist es aber blöd, nämlich, wenn Heiko den Beschützer spielt und nicht will, das Johanna all die „gefährlichen“ Abenteuer mit macht.

Einmal, alle fahren rasend schnell mit ihren Fahrrädern auf dem Feldweg zu Johannas Haus, da fällt Johanna hin. Sie hat von allen das kleinste Fahrrad, die anderen sind weit vor raus, sie rutsch mit dem Vorderrad über die Kannte des Plattenweges, der Lenker fliegt herum, das Fahrrad springt, padauz. Johanna kriecht unter dem Fahrrad vor, sieht die anderen um die Ecke zu ihrem Haus abbiegen und ärgert sich. Oh, wie sie sich ärgert. Wütend steigt sie mit schmerzendem Knie auf ihr dummes, kleines Fahrrad und strampelt hinterher.

Als sie endlich zu hause ankommt, steht die Mutter in der offenen Tür, Heiko liegt vor ihr auf der Auffahrt am Boden: “So ist sie hin geflogen. So, so auf der Seite. Und das Bein war da unter dem Fahrrad.“ „Johanna, da bist du ja, du Arme, ist alles in Ordnung?“ fragt die Mutter besorgt.
 Johanna kocht, warum muss Heiko diesen peinlichen Sturz vor allen Jungen und der Mutter breit treten? Was geht ihn das an? Wieso nimmt er ihren Sturz um sich vor den anderen wichtig zu machen? Sie wollte die Sache lieber für sich behalten, sich wie ein Junge benehmen und tun als sei nichts passiert. Warum will Heiko Johanna wie ein Mädchen behandeln? Er schenkt der Prinzessinnenseite eine ungewohnte und unangenehme Aufmerksamkeit.
 
Als Heiko dann aber mit seinem kleinen Bruder an Johannas Geburtstag morgens früh vor der Tür steht, um ihr sein Geschenk zu bringen, in feinem Hemd und sauberer Hose, fein auf die Seite gekämmtem Haar, wie ein kleiner Herr, ist sie geschmeichelt. Sie will prinzessinengleich nach dem Päckchen greifen.  Aber die Mutter schickt die Jungen wieder nach Hause, sie sollen am Nachmittag kommen, wie alle Geburtstagsgäste.



\section*{Feigenbaum I 
Landesmuseum Schleswig-Holstein, Schloss Gottorf, Raum der Moorleichen, 1977)}
\addcontentsline{toc}{section}{Feigenbaum I (Landesmuseum Schleswig-Holstein, Schloss Gottorf, Raum der Moorleichen, 1977)}



{\em Ich liege still, begafft  mit angeekelten Blicken bemitleidet. Meine Haare haben sie genommen, damals, das Moor färbt Haare rot.
Mein Kopf ist schwarz und ich liege, schwarz an Leib und schwarz an Erinnerungen, schwarz in der Stille. Begierig kriechen ihre Blicke über mich, meinen Tod, meine Wiedergänger tanzen um sie.

Die bösen Geister trinken aus ihrem Mitleid und sie starren durch die Scheibe.

Ich springe sie an, das kleine Mädchen hinter der Scheibe, sieh her, was sie mit mir, mit uns getan haben. Ich kralle mich in ihre erstarrten Augen, die wiedererkennen, aber nicht begreifen, im Schock geweitet, mich verschlingen. Sieh mich an, meine vertrockneten, vom Strick zerstörten Stimmbänder knistern ihr ins Ohr:

"Ich war du und die, die du werden wolltest, wird ich werden, bis du mich in dir erlöst hast."

Sie trägt mich mit sich nach Hause und ich nehme sie mit jede Nacht\dots
Jede Nacht ertrinken wir im Moor und im Laufe der Jahre zieht sich der Strick, der meine Stimmbänder und meine Luftröhre zerbrach von Innen um ihren Hals.

Sie wird stumm, erstickt an der Stille, in der ich in ihr ertrinke im Moor, jede Nacht\dots

Nacht, um Nacht, um Nacht\dots

Ich liege, sie liegt und die Äste prasseln auf uns, verkrampft im Tod, in der Schande, verewigt am Pranger der Geschlechtlichkeit, ins Moor gegossen, im Schaukasten ausgestellt.

Ich liege in ihr, ich flüstere, ich knistere ihr den Strick um den Hals:
"Es ist das Herz! Törichtes Ding. Es muss schweigen, damit du leben kannst, dass es zwischen deinen Beinen still bleibt. Es ist das Herz."

Sieh mich, ich bin schwarz, schwarz wie die Nacht, wie die Sünde, wie die Offenbarung und die Verheissung. Schwarz und zerspielt\dots

Ich bin Hülle. Ich war Hülle. Aber das Moor, die schlüpfrige, alte Erdenmutter, sie hat mich zur Figur gemacht. Das Moor machte aus mir das Bild deines Schattens.

Ich bin schwarz. Ich bin tot und durchdrungen von der Erde zurückgekehrt, damit sie, die ihr Herz sucht, sich spiegelnd ins Licht wenden kann.}


\section*{Anderswelt I}
\addcontentsline{toc}{section}{Anderswelt I}



Johanna steigt mit der Mutter aus dem Auto. Heute ist ihr erster Schultag und sie ist aufgeregt. Die Mutter gibt ihr eine grosse Schultüte aus der oben der Kopf eines Barbiepferdes herausragt. Das hat sich Johanna lange gewünscht. Die Schule ist nicht die, an der sie früher immer vorbeigefahren sind, wenn sie in Itzehoe einkaufen gingen und wo Conny und Britta zur Schule gehen. Sie sind lange mit dem Auto gefahren bis zu der Stadt, wo Oma Ofeld, Mamas Mama im Nachbardorf wohnt.

Conny ist ein Jahr vor Johanna in die Schule gekommen und brachte von dort einen „Fu“ mit. Einen, aus einer alten Socke genähten, Fingerpuppenhund, der den Kindern im ersten Schulbuch erklärt, wie Lesen und Schreiben geht. Johannas Mutter hat ihr auch einen Fu genäht und die Kinder sitzen mit ihren Fus auf der Hand über das Schulbuch gebeugt und üben Buchstaben schreiben auf Blättern mit Linien.

Und heute darf Johanna selbst in die Schule\dots

Sie ist mit der Mutter in dieser Schule gewesen um ihren Lehrer kennen zu lernen, wie die Mutter sagt. Johanna findet alles merkwürdig. Den muffigen Geruch des Büros, indem sie und die Mutter von zwei Lehrern erwartet werden und deren prüfenden Blicke sind unangenehm. Der junge Lehrer will Johanna einen Ball zu werfen, den sie auffangen soll und Johanna bekommt Angst, denn fangen kann sie nicht gut, aber es geht. Und Rückwärtslaufen soll sie auf einem Strich, der auf den Boden gemalt ist. Johanna scheinen alle die Aufgaben, die ihr der Lehrer gibt, schwierig, dabei ist sie hundertmal rückwärts gelaufen und hat Dinge gefangen, aber, wenn der Lehrer es sagt und sie dabei streng beobachtet, dann fühlen sich ihre Arme und Beine anders an, als gehörten sie ihr nicht.

Zum Schluss darf Johanna ein Bild malen und sie entspannt sich. Wenn sie eine Sache kann und gern hat, dann ist es Malen. Sie malt ein Haus mit Garten und Himmel und Wolken und ist zufrieden. 

Aber der Lehrer ist es nicht. „Das Bild ist doch noch nicht fertig. Was fehlt noch, Johanna?“  Johanna sieht sich ihr Bild an und kann nichts Fehlendes entdecken. „Doch, das Bild ist fertig.“ „Nein, das Haus ist kein richtiges Haus, es hat nur einen Strich rundherum. So sieht kein Haus aus. Du musst es ausmalen.“ Johanna presst die Lippen zusammen. Sie malt nie ihre Häuser aus. Aber, wenn der Lehrer es will\dots Johanna nimmt den Buntstift und beginnt zu malen. Es ist mühsam und Johanna findet ihr Haus wird durch die vielen wirren Striche immer hässlicher. Sie legt den Stift weg. „Johanna, schau, das Dach braucht noch Dachziegel.“ Der Lehrer lässt nicht locker. Also malt Johanna das Dach an. Als sie fertig ist, sieht sie ein Bild, wie sie es noch nie von sich gesehen hat und sie mag es nicht leiden. Und das Gewissen sagt: „Der Lehrer sagt, du kannst nicht richtig malen.“


\sterne

Aber all das ist am ersten Schultag vergessen. Johanna wird heute Lesen, Schreiben und Rechnen lernen und sie geht selbst zur Schule.
Zuerst sammeln sich alle Kinder, die in  die Schule kommen in einem kleinen Haus, das sechs Ecken hat, der Eurythmiepavilion, aber das weiss Johanna ja nicht. Die anderen tragen auch Schultüten. Sie setzten sich vor eine kleine Bühne und die zweite Klasse führt ihnen ein Theaterstück vor, indem es um die Wochentage geht. Während der Aufführung lässt ein Junge seine Schultüte fallen und einige Sachen fallen heraus. Weintrauben hat er in seiner Schultüte. Der Junge tut Johanna Leid, denn sie wollte keine Früchte in ihrer Schultüte und sie freut sich auf ihr Barbiepferd.

Endlich ist die Aufführung vorbei und der Lehrer verspricht, dass sie in die Klasse gehen und ihre erste Unterrichtsstunde haben werden! Für jedes Kind liegt auf jedem Schulplatz ein Heft mit leeren, weissen Seiten und ein Kästchen mit Wachsmalkreide. Johanna wundert sich, sie findet Wachsmaler doof, zu hause malt sie lieber mit Filzstiften. 

Die Kinder dürfen nicht in ihr Schulheft malen, sondern sie bekommen ein Blatt. Der Lehrer, Herr Boss, erzählt den Kindern eine Geschichte. Dann zeigt er auf die Tafel. Auf der einen Tafel ist ein dicker, senkrechter Strich, der Lehrer nennt ihn „Grade“, gemalt, blau mit gelb drumherum, auf der anderen Tafel ist ein halber Kreis, der Lehrer nennt ihn eine „Krumme“ und dies beides sollen die Kinder auf ihr Blatt zeichnen.

Keine Buchstaben, keine richtigen Stifte, keine Bücher, kein Fu und keine Zahlen, keine Hefte mit Linien\dots?

Nach einer kurzen Stunde fahren Mama und Johanna wieder nach Hause\dots
Johanna freut sich als sie nach Hause fahren, sie will das Barbiepferd aus der Schultüte befreien und am Nachmittag kommen alle Freunde um mit ihr, wie am Geburtstag, ein Fest zu feiern, weil sie in die Schule geht. In eine komische Schule mit lauter fremden Menschen. Wo alles anders ist\dots


\sterne


Am Anfang ist es lustig. Die Schule ist kurz und Johanna und die Mutter fahren viel mit dem Auto. Vor allem wird plötzlich die Oma viel besucht, die im Nachbardorf neben der Schule wohnt. Dort kann Johanna am Bahndamm und im Garten sein und die Oma kocht das feine Gemüse aus dem Garten und freut sich, dass Johanna es gerne isst.

Ausserdem findet Johanna den Unterricht spannend und mag den Lehrer, Herrn Boss, sehr. Der kann tolle Geschichten erzählen und jeden Tag gibt es Neues zu erfahren. 

Johanna bemüht sich sehr, dass sie mit Herrn Boss reden kann und möchte ihm am liebsten alle Neuigkeiten mitteilen. Aber es ist schwierig Herrn Boss etwas zu sagen, denn es hat viele andere Kinder, vierzig Stück und alle wollen Herrn Boss erzählen. Einige von den Jungen wollen lieber raufen und jemanden verprügeln und um die muss sich Herr Boss vor allem kümmern.

Johanna hat die Schule gern, aber auf dem Stuhl sitzen, dem harten, aus Holz, ist schwer. Die Beine wollen sich bewegen und allzu schnell schlüpft das Bein unter den Po und dann rutscht sie, die Beine wechselnd hin und her, oder Johanna hängt eine Zeit über den Tisch. Die rhythmischen Übungen, bei denen die Kinder stehen, die sind gut, da darf sie stampfen und trappeln und trippeln. Klatschen und alle sprechen zusammen.

Im Unterricht ist Johanna still und lauscht den Geschichten, sind diese lustig muss sie lachen und dann lacht sie laut und lange. Herr Boss wird das im ersten Schulzeugnis erwähnen, dort klingt es tadelnd: „Johanna lacht ausdauernd und laut im Unterricht.“

„Herr Boss, Herr Boss, ich habe ihnen ein Bild gemalt!“ Johanna strahlt, was für eine Mühe hat sie sich gegeben ein besonders schönes Bild, schliesslich ist Herr Boss ihr Lieblingslehrer und sie möchte seine liebste Schülerin sein. Herr Boss schaut seltsam, er scheint sich nicht zu freuen, er scheint nicht zu wissen, warum Johanna ihm ein Bild schenken will. Das Gegenübergesicht des Lehrers wird hart. Johanna wird von einer Kraft, aus dem Blick zurückgestossen, die innen drin Weh macht.

Johanna bemüht sich alle Aufgaben richtig und gut zu machen, sich zu benehmen. Sie hofft, Herr Boss wird es bemerken und sie loben. Sie möchte wissen, ob sie ihre Sache gut macht. Aber es gelingt ihr nur selten, Herrn Boss` Aufmerksamkeit  zu erlangen. Beim Rechnen, da begreift sie schnell und Herr Boss lobt sie. Sie freut sich, hat aber das Gefühl, als müsste Herr Boss darüber staunen, dass da eine gute Johanna in seiner Klasse ist.
Für die Spielkameraden ist kaum Zeit. Aber Johanna ist in Oldendorf, sie schläft dort, sie sitzt abends auf ihrer Schaukel und schaukelt in die Sonne hinein und sieht den Engeln beim letzten Tänzchen zu. Papas Kraftwerk summt neben dem Haus.

Die Anderswelt der Schule kann Johanna verlassen, wenn sie zu Hause ist, vergessen und in die gewohnte, liebe Welt in Oldendorf  zurückkehren.


\sterne


Dann zieht die Mutter mit Johanna um. Nach Ofeld, das Dorf in dem die Oma wohnt und das die Stadt mit Johannas Schule zur Nachbarin hat. Auch Onkel, Tante und vier Kusinen wohnen in Ofeld. Elisabeth, die jüngste Tochter des Onkels ist drei Wochen älter als Johanna und sie kennen sich gut. An Familientreffen und wenn Johanna die Oma besucht, dann spielt sie mit Elisabeth. Elisabeth hatte Johanna in den Ferien in Oldendorf besucht. Aber sie hatte immer Heimweh und Angst vor Johannas Mutter und deshalb wollte sie nie richtig spielen und nichts essen.

Der Vater muss daheim bleiben, weil die neue Wohnung zu klein ist und weil er im Kraftwerk arbeiten muss\dots Johanna fühlt sich fremd in Ofeld.

In Oldendorf hatte sie das Gefühl, die Wiesen, die Wege, die geheimen Ecken und Höhlen in den Wäldchen und Knicks gehören ihr und Conny. In Ofeld gehören all diese Orte jemand anderem. Elisabeth kennt sich viel besser aus in Omas Garten und sie kennt die Oma und den Opa, der meist im Garten oder im Haus sitzt, besser. Johanna muss Oma und Opa teilen. Und jeden Platz den sie betritt, muss sie teilen oder bitten, ob sie ihn betreten darf.

Es ist gut, dass Elisabeth da ist und die geheimen Ecken und Spiele kennt, aber für Johanna ist es „geborgter“ Platz, sie sehnt sich zurück an ihren eigenen Ort.

Es ist eine komische Wohnung und eine komische Vermieterin, sie kommt Johanna wie eine Hexe vor, eine, die auf den ersten Blick ganz manierlich aussieht und dann von hinten über einen kommt. Es wohnen andere Leute mit in dem Haus, dass direkt an der Hauptstrasse an der ersten Kreuzung steht. Johanna wohnt nicht gerne mit all den Leuten unter einem Dach. Es hat diese fremden Geräusche und Gerüche. Dem Haus fehlt der Garten, wenn Johanna vor die Tür geht, steht sie auf der Strasse. Die Mutter ist unzufrieden und es gibt Schimpfe und Streit am Wochenende mit dem Vater. 

Aber, das spürt Johanna, die Mutter ist auch froh, dass sie in Ofeld wohnen. Sie müssen nicht weit fahren und sie findet Johanna könnte endlich mit ihren Schulkameradinnen spielen, von denen es welche im Dorf gibt. 


\section*{Andorra I}
\addcontentsline{toc}{section}{Andorra I}



Johanna verabredet sich oder wird verabredet, kein Entkommen mehr vor der Anderswelt, die gibt es hier scheinbar überall. Geheimnisvolle Gesetzte, Verbote und Dinge, die alle wissen, nur Johanna nicht und Menschen, die ohne ihr Herz lächeln.

Die Klassenkameradinnen kennen sich vom Kindergarten und Johanna fühlt sich wie ein Eindringling. Die Mädchen sind freundlich, aber Johanna versteht ihre Sprache nicht. Sie spielen lauter Dinge, die Johanna nicht kennt. In Oldendorf hat sie mit Britta gespielt, aber seltener und Britta und sie haben wilde Spiele gehabt.

Aber die Mädchen aus der Schule sind ruhig und leise, wie kleine Erwachsene. Sie sind verwirrt, wenn Johanna laut und wild und polternd umher flitzt. Die Eltern der Mädchen scheinen sich über Johanna zu wundern und zu ärgern, nämlich, wenn ihre Kinder wild werden wie Johanna. Johanna spürt eine Wand und dahinter Ablehnung. Diese Wand fühlt sich bedrohlich an, denn die Menschen, die Johanna kennenlernt, verstecken sich dahinter. Sie lächeln vor der Wand und in sich drinnen passiert anderes: Unverständnis, Ungeduld, Ärger. 

Johanna spürt die Dinge hinter der Wand genau, aber sie weiss nicht, wie sie damit umgehen soll. Bisher kennt sie nur Menschen, wie die Bauerneltern von Conny, die sagen, wenn sie ungeduldig oder wütend sind. Das findet Johanna nicht angenehm, aber sie weiss, was sie falsch gemacht hat.

Wissen denn die Erwachsenen nicht, wie schlimm es für Johanna ist, wenn Johanna hinter der Wand des Erwachsenen schutzlos seinen unangenehmen Gefühlen ausgeliefert ist? Johanna ertrinkt immer öfter in den freundlich lächelnden, schmutzigen Sümpfen.  Dabei entdeckt sie einen Raum in sich, in dem sie sicherer ist, dort machen die spitzen Bemerkungen, Unverständnis und Ärger halb so weh. Aber der Raum ist weit in ihr verborgen und es ist ein weiter Weg dahin und ein weiter zurück. So geschieht es: Johanna bleibt öfter für längere Zeit in dem Schutzraum. Die Seelenwasserglocke hält die Welt fern. Schliesslich sagt sie, wenn Erwachsene dabei sind, kaum ein Wort. Sie, die früher alle wilden Jungenabenteuer  vorn an der Spitze mitgemacht hat, wird ängstlich, lässt sich von Klassenkameraden puffen, schlagen und beschimpfen und, weil sie tief in sich verborgen ist, ist der Weg zum herausspringen und denjenigen packen, zu weit.

Herr Boss macht eine neue Sitzordnung und Johanna muss neben Michael sitzen, vorne ganz auf der Seite. Michael ist ein kleiner, teigig-dicker Jungen, mit blassem, greisenhaftem Gesicht, der schlägt, frech ist und von der Schule nichts wissen will. Selbst die Rowdys in der Klasse wollen nichts mit ihm zu tun haben. Johanna ekelt sich vor ihm. Lieber möchte sie neben einer Schlangengrube sitzen. Sie rutscht so weit sie kann an den Tischrand. Erst als Michael ihr in den Bauch tritt, traut sie sich Herrn Boss etwas zu sagen. Der wird wütend und schimpft mit Michel, aber dieser Triumph ist fade, denn es nutzt nicht viel und schliesslich muss Johanna wieder auf den Platz. Kein Mädchen sitzt in der Nähe, niemand, zu dem Johanna gehören mag. Die nächste neue Sitzordnung folgt, aber nicht für Johanna, sie wird die nächste Runde, neben Michael vorne an der Wandseite begraben.

Dabei tritt diese Kraft wieder und wieder in Erscheinung und bereitet Schmerz. Antipathie. Johanna versteht nicht, was sie falsch macht. Aber, wenn sie Herrn Boss begegnen will, steht da diese Kraft. Dabei gibt es Kinder, die scherzen und lachen mit Herrn Boss, die dürfen erzählen. Sie befinden sich in einem feinen Kokon, der sie schützt. Johanna ist nicht darin. Sie begreift, auch wenn sie es nicht verstehen kann, nicht alle Kinder in der Schule sind gleich\dots Es gibt Kinder in und Kinder ausserhalb des Bindungsgewebes. Johanna glaubt sich schuldig. Sie muss selber Schuld sein, wenn sie nicht zu den guten Kindern gehören darf. Sie lernt, am Besten ist, sie ist unsichtbar\dots Vielleicht kann sie, wenn sie sich versteckt hält, so Zuneigung erlangen.

Das verhängnisvolle Band, das sich bei der Begegnung mit der Frau aus dem Moor um ihren Hals gelegt hat, zieht sich zu.



\section*{Amici}
\addcontentsline{toc}{section}{Amici}




Etwas besser wird es, als die Mutter und sie wieder mit dem Vater und all ihren Sachen in eine neue Wohnung, diesmal neben dem Kraftwerk in Ofeld, ziehen. Johanna bekommt ein schönes, grosses Zimmer und es hat Felder, Wiesen, alte Bahndämme und Wald drumherum. Dort löst sich die Anderswelt wieder auf, da kann Johanna durch atmen. Nachbarjungen hat es dort, den stämmigen, ruhigeren Matthias und Torben. Torbens Beine wollen nicht gleich schnell wachsen. Er trägt an einem Schuh eine ganz dicke Sohle, aber das Bein darf frei sein, während das andere von der Hüfte bis zum Fuss zwischen zwei Metallstangen steckt, die mit breiten Lederriemen an Ober- und Unterschenkel befestigt sind, das Bein ist dadurch steif, es soll sich strecken. Wenn Torben laufen will muss er sein geschientes Bein, weil es sich nicht knicken lässt, in einem kleinen Halbkreis nach vorne bewegen, dabei schwangt er mit dem Oberkörper weit in die entgegengesetzte Richtung. Torben ist klein und schmal. Als sie sich kennenlernen, ärgern sie sich mit freundlichen Schimpfwörtern und Beleidigungen, schliesslich sind sie in einem Alter, da Jungen und Mädchen nicht selbstverständlich Freunde werden können. Aus den freundlichen Schimpfwörtern werden kleine Spasskämpfe. Beruhigt merken die Jungen, dass Johanna eine gute Kämpferin und niemals zimperlich ist, im Gegenteil, Matthias hat keine Chance sie zu besiegen. Nachdem diese wichtigen Dinge geklärt sind, spielen sie, dem Dorfkindertabu folgend, das auch hier gilt, draussen. 

Allerdings gibt es ein Problem, denn Johanna weiss nicht, wie sie sich Torben gegenüber verhalten soll. Am Anfang ist sie lieber mit Matthias allein, denn die Schiene an dem Bein findet sie schrecklich. Sie ist behutsam mit Torben und wenn er sie zum Kampf auffordern will, dann weicht sie aus und wenn Torben angreift, packt sie ganz sanft zu. Oder sie windet sich heraus, in dem sie Matthias packt. Dadurch ist Torben draussen vor.

Heute ist es anders, denn Matthias ist nicht da. Johanna trifft Torben allein, bei sich auf dem Rasen. Torben beginnt die gewohnten Rangeleien und Johanna kann nicht ausweichen. Ihr Herz klopft immer mehr. Wie soll sie sich raus winden? 

„He, jetzt kämpf` mal richtig.“ Zornig blitzt Torben Johanna an. „Du machst gar nicht Ernst!“ Johanna blickt ertappt zu Boden. Wie soll sie Ernst machen, wenn Torben behindert ist? „Ich will, dass du richtig kämpfst!“ In Torbens Augen sieht Johanna nicht nur Trotz, sondern auch Wut und Trauer. Sie begreift, sie fügt Torben mehr Schmerzen zu, wenn sie ihm ausweicht, als wenn sie ihn im Kampf Weh tut. „Okay, wenn du willst.“ der Kampf beginnt, Johanna ist erstaunt mit welcher Kraft Torben auf sie losgeht. Aber sie traut sich nicht an zu greifen. „Du kannst mir ruhig wehtun.“ Torben wirft ihr einen bittenden, erwartungsvollen Blick zu. Also, Johanna packt richtig zu. Ein Titanenkampf beginnt. Sie ringen und zerren aneinander, versuchen den anderen mit Beinstellen zu Fall zu bringen. Keiner von beiden gibt nach, je länger es geht, desto mehr steigt die Wut der letzten Jahre auf. Und befreit spüren beide, sie haben einen würdigen Gegner gefunden, der die Wut aushalten kann.

Der Kampf endet unentschieden. Von da an sind spielt Johanna genauso mit Torben, sogar lieber. Dank der Jungen kann Johanna sich wieder benehmen wie sie es gewohnt ist, die Freundschaft ist lockerer, als die zu Conny, aber sie erinnert an die Zeit, da Johanna mit den Jungen in Oldendorf durch die Marsch ziehen konnte.

 Und es ist ein Kraftwerk da, das summt und brummt, wo sie mit dem Vater am Wochenende mitgeht und „Büroarbeiten“ macht, zwischen den Rohren und Ventilen schleicht, genau gleich wie in Oldendorf, ein wenig Heimat unter den drei Schornsteinen.  Und der riesige Wald, der die ganze Seite der langen Strasse, die ins Dorf führt, bedeckt. Ein guter Wald zu Spielen, es hat eine Pferdekoppel die mitten im Wald, an der Strasse liegt.
Johanna bekommt ein neues, ein grosses Fahrrad, mit dem sie in die Schule fährt. Johanna kann nach der Schule zur Oma oder nach Hause fahren. 



\section*{Tod III (Oldendorf, 1982)}
\addcontentsline{toc}{section}{Tod III (Oldendorf, 1982)}




Johanna ist für einen Nachmittag mit den Eltern bei Brittas Familie in Oldendorf zu Besuch. Während die Eltern drinnen beim Kaffee erzählen, spielen die Mädchen und einige Kinder aus dem Dorf in Brittas Garten. Es gibt Streit mit Ramona, die Kinder mögen sie nicht und schnell werden  deftige Wörter ausgetauscht.  

„Ätsch, Conny ist tod!“

„Du lügst, du blöde Kuh!“

„Nein, ich lüge nicht, Conny ist tot!“ Johanna schluckt, sie weiss selbst, dass Conny lange im Krankenhaus war und viele Operationen bekam. Sie hat Conny mit der Mutter besucht, als er in Kiel in der Uniklinik liegt und ihm alle guten Asterix und Lucky Luck mit gebracht und  nicht wiederbekommen. Die Mutter hat mit Conny geredet, sie, die Krankenschwester, und er sprachen eine Sprache von Tod und Leid, die Johanna nicht verstehen konnte. Eine Sprache, der Johanna nicht folgen kann, nicht folgen will.

Aber warum weiss die blöde Ramona, dass Conny tot ist und Johanna seine beste Freundin nicht? Sie sagt das, um mich zu ärgern, versucht  Johanna zu denken, aber sie sieht in den erschrockenen Augen Ramonas den Spiegel ihres an der Wahrheit gefrorenen Herzens.

Später am Abend wird die Mutter auf dem Marktplatz in Tränen ausbrechen, unsicher und hilflos gehalten vom Vater, ohne ein Wort über Conny zu Johanna. Johanna steht allein\dots Warum weine ich nicht? Fragt sie sich.
Aber wohin mit  dem Schmerz, wenn die Erwachsenen weinen\dots

Keine Beerdigung, die Mutter geht hin, aber sie will Johanna den Schmerz ersparen und fährt allein. Keine Beerdigung, keine Abschied von Conny für Johanna. Ihr erster Mann verschwindet, bevor sie erwachsen geworden sind. Hat sich langsam und still aus ihrem Leben geschlichen. 

Ob sie ein Paar geblieben wären, später? Die letzten Male, wo Conny Johanna besucht hat, ist es anders gewesen. Aber -alles ist anders gewesen, denn Johanna wohnt nicht mehr in Oldendorf. 

Und wenn Conny zu Besuch kam, wollte er gleich wieder heim, weil er Heimweh hatte und Johanna war wütend, weil sie neidisch war, dass Conny zurückkehren konnte und sie nicht. 

Die Zeit des klaren, einfachen, von Knicks geschützten Weges ist vergangen\dots


\part*{Teil II}
\addcontentsline{toc}{part}{Teil II}


\section*{Spruch}
\addcontentsline{toc}{section}{Spruch}



\begin{verse}
Ich schaue in die Welt,\\
in der der die Sonne leuchtet,\\
in der die Sterne funkeln,\\
in der die Steine lagern,\\
die Pflanzen lebend wachsen, \\
die Tiere fühlend leben, \\
in der der Mensch beseelt\\
dem Geiste Wohnung gibt.\\
Ich schaue in die Seele,\\
 die mir im Innern lebet, \\
der Gottes Geist erwebt\\
im Sonn- und Seelenlicht.\\
Im Weltenraum da darausen,\\
in Seelentiefen drinnen, \\
zu dir, oh, Gottesgeist,\\
will ich bittend mich wenden,\\
dass Kraft und Segen mir\\
zum Lernen und zur Arbeit\\
in meinem Innern wachse.  
\end{verse}
\begin{dichter}Rudolf Steiner\end{dichter}


\section*{Liebeslied}
\addcontentsline{toc}{section}{Liebeslied}



\begin{verse}
Wie soll ich meine Seele halten, dass\\
sie nicht an deine rührt? Wie soll ich sie\\
hinheben über dich zu anderen Dingen?\\
Ach, gerne möcht ich sie bei irgendwas\\
Verlorenem im Dunklen unterbringen\\
an einer fremden, stillen Stelle, die\\
nicht weiterschwingt, wenn deine Tiefen schwingen.\\
Doch alles, was uns anrührt, dich und mich,\\
nimmt uns zusammen wie ein Bogenstrich,\\
der aus zwei Seiten eine Stimme zieht.\\
Auf welches Instrument sind wir gespannt?\\
Und welcher Spieler hat uns in der Hand?\\
Oh, süsses Lied
\end{verse}
\begin{dichter}Rainer Maria Rilke\end{dichter}


\section*{Enee, menee Miste, es\dots (Rendsburg 1985-1986)}
\addcontentsline{toc}{section}{Enee, menee Miste, es\dots (Rendsburg 1985-1986)}




„Mira und du, ihr seit die einzigen von der ganzen Schule, die ich leiden kann.“ Es schmeichelte Johanna als Antje das sagt, obwohl sich auch ihr schlechtes Gewissen regt, mit dem Hinweis, dass Antje und ihre Meinung ihr prinzipiell egal sind und ihr Stolz über dieselbe daher  nicht zusteht. Es ist früh morgens und die beiden gehen der Schulfreundin Mira entgegen, die vom Bahnhof her gelaufen kommt. Sie legen beide keinen grossen Wert auf die Gesellschaft der übrigen Klassenkameraden. „Und den, den finde ich soo toll!“ Anja stupst Johanna mit dem Ellenbogen in die Seite, gleichzeitig versteckt sie sich hinter Johannas Rücken, als Thomas an den beiden vorbei geht, unnahbar, männlich-melancholisch, ganz in schwarz, ohne sie zu sehen. Er besucht die Klasse über den beiden. „Ha, das ist doch Thomas, mit dem bin ich im Flötenkreis in der Musikschule“, antwortet Johanna. Oh, jeh, denkt Johanna, wie kitschig und schmalzig, sich in einen arroganten, Grufftischnösel verlieben, das kann mir nicht passieren. Sie fühlt Scham für die Kameradin, die zitternd und mit schmachtendem Blick, in hoffnungsloser, ineffizienter Romantik dem Objekt ihres Herzschmerzes hinterher schaut. Aber Johanna kommt eine Idee, die ihr einige Unterhaltung zu bieten scheint. „Komm doch nächsten Dienstag mit in den Flötenkreis.“ „Meinst du das geht?“ „Na, klar, du kannst zur Probe mitkommen.“ „Dann sehe ich ihn aus der Nähe, cool!“ Antjes Augen leuchten, Johanna denkt an den Flötenkreis, den ihre kleine, strenge, grauhaarige Flötenlehrerin leitet. Über eine andere Klassenkameradin war sie in den Flötenkreis gekommen und recht erstaunt gewesen, weil zwischen all den Mädchen neben Friedhelm mit der Hasenscharte, der obercoole Thomas gelangweilt mit der Bassflöte auf dem Stuhl sass. Was macht der hier? Die Antwort  ist einfach, aber bezwingend, denn eines Tages sagte Thomas „Mutti“ zu der resoluten Flötenlehrerin Frau K.!

Es ist eine Katastrophe. Antje sitzt, da sie Sopran spielt, neben Thomas, allerdings sitzt Frau K. zwischen den beiden und weil Anja vor lauter Aufregung kaum spielen kann, wird sie gehörig zusammen gestaucht, während Thomas genervt die Augen verdreht. Sie verlässt den Flötenkreis mit hochrotem Kopf und kommt nicht wieder. 

Frau K., die Johanna auch Einzelstunden im Flöten gibt, fragt an, ob Johanna nicht zur Verstärkung in die Bassstimme kommen kann. Johanna bekommt eine eigene Bassflöte von den Eltern. Und nun sitzt sie im Flötenkreis neben dem arroganten Thomas. Sie ignoriert ihn. Auch wenn sie Antje mitgenommen hat, gleichwohl sich einen Scherz erlaubend, hat Thomas kein Recht sich derart unfreundlich auf zu führen. Johanna kann gut schweigen, es ist das einzige, was sie gut beherrscht und die Bassstimme des Flötenkreises  wird zum kältesten Punkt im Umkreis von der kleinen Stadt. Friedhelm sitzt unbeholfen dazwischen. Bei jedem Versuch von Thomas seine schlechte Laune, die er scheinbar im Flötenkreis hat, an Johanna aus zu lassen, faucht sie ihn an wie eine Wildkatze. Es soll ihm klar werden, dass sie nicht zu seinen Anhängerinnen gehört, die allein bei seinem Blick davon schweben. 

Die Resistenz, die Johanna zeigt, beeindruckt Thomas. Er ist es nicht gewöhnt von Mädchen angefahren zu werden. Das stimmt nicht ganz, denn Thomas hat vier grosse Schwestern zu Hause, bei denen er allenfalls als Nesthäckchen zählt\dots  Mit der Zeit kommt es zum Waffenstillstand. Sie sind höflich, aber distanziert miteinander, dabei bemerkt Johanna, dass Thomas, wenn er will auf eine unbeholfene Art nett sein kann.

Mit der Zeit merkt Johanna, dass der Dienstag ein besonderer Tag wird, weil sie dann Thomas sieht. Zwar machen sie die getauschten Freundlichkeiten unsicher, aber es macht ihr mehr Freude nett zu Thomas zu sein, statt als Racheengel neben ihm zu sitzen. Kein Blitz trifft sie, kein Donner erschallt in ihr, der ihr ankündigen würde, was in ihrem Herzen passiert. Friedlich und leise gebärt sich Thomas in ihre Seele. Eine Ruhe, auch wenn das Herz schneller schlägt, wenn er kommt, breitet sich in Johanna aus. Dabei lauscht sie im Inneren den Tönen und Farben beider, die sich mischen und miteinander spielen. Kein Blitzschlag, sanft und leise wächst ein Gefühl, das in Muskeln, Herz, Blutgefässe und in die Knochen dringt.

Johanna sitzt im Flur der Musikschule und liesst. Viktor Hugo ist der Auserwählte, „die Elenden“. Ein Lehrer hat im freien Religionsunterricht davon erzählt. Das Buch ist alt und dick, Johanna fühlt sich wohl darin. Es gibt ihr Gewicht, Nahrung und Worte.

 „Was liesst du denn da?“ Thomas schaut erst auf das Buch und sieht Johanna an, als hätte er einen Schatz entdeckt. In seinem Gesicht sieht Johanna Faszination und Erstaunen. „Och, das ist Viktor Hugo.“ Gibt sich Johanna betont langweilig. „ Liesst du öfter solche Bücher?“  „Ja, klar, warum denn nicht?“ Johanna fragt sich im Geheimen, was „solche Bücher“ sind. 
 
Am nächsten Dienstag sitzt Johanna und liesst. Thomas kommt. Johanna liesst. Thomas reisst ihr das Heft aus der Hand. „Oh, Luuki Luuk“, er schaut enttäuscht\dots  Johanna liesst wieder \dots „solche Bücher“. Sie möchte sich und Thomas nicht wieder in Verlegenheit bringen. 
Blicke werde ausgetauscht und immer öfter musste Frau K. daran erinnern, dass die beiden sich im Flötenkreis befinden, was beide beschämt, weil es ihnen geheim ist, dass sie gut miteinander auskommen.

Sie packen die Noten zusammen ein, jeder hält einen Zipfel fest, Johanna bekommt eine Gänsehaut, als sie die schwarzen Haare auf Thomas Arm entdeckt. Ein zarter, dunkler Flaum. Sie kichern zusammen und verpassen eins ums andere mal den Einsatz. Thomas zupft ganz vorsichtig an Johannas Pullover, wir müssen spielen! Und Johanna wird es heiss und kalt dabei. Frau K. schaut sich das ganze an und Johanna hat das Gefühl, sie würde sich eins schmustern, wenn ihre Bassspieler mal wieder geträumt haben. 


\sterne

In der Anderswelt der Schule versteckt Johanna sich. Thomas ist eine Klasse höher, gehört zu den Grossen, vor denen hat sie Respekt. Thomas Klassenkameraden sind viel fröhlicher und freundlicher im Umgang miteinander, das irritiert Johanna. Jungen und Mädchen reden und lachen zusammen, unmöglich denkbar in Johannas Klasse. Da gibt es die grosse Gruppe Jungen, die alles dominieren, terrorisieren, gehässige Bemerkungen im Unterricht machen, wenn sich ein unwürdiges Wesen gemeldet hat. Einer kleine Gruppe Mädchen ist reden erlaubt, der Rest ist still und leise und öffnet den Mund in der Pause, wenn die hohen Herren weit weg sind. Johanna hat sich raus gezogen. Sie hat Mira. Sie haben ihre kleine Welt und wenn Mira versucht bei den guten Mädchen mitzuhalten und in ihren Kreis drängt, betrachtet Johanna das mit Argwohn, kümmert sich aber nicht darum, versteht den Sinn nicht. Warum soll sie um Aufmerksamkeit betteln, bei Menschen, die sie nicht achten?

 Dass in Thomas Schulclique Mädchen sind, findet Johanna gefährlich und bei jeder Gelegenheit, in den Pausen, beobachtet sie misstrauisch, die Mädchen, die um Thomas herumschwirren und viel zu laut und unnütz lachen. Allerdings stellt sie erleichtert fest, dass Thomas sich vor zu viel Nähe abwendet. 
 
Die Oberstufentage rücken heran. Drei wunderbare Tage in denen die Oberstüfenschüler zwei Gruppen wählen dürfen und bunt gemischt diskutieren und handwerklich tätig sein können. Drei Tage Pause von der Anderswelt. Johanna hat mutig einen Tag gewartet, bevor sie sich in ihre Kurse einschreibt, jetzt kann sie sich im Zeichnen eintragen, da wo Thomas sich eingeschrieben hat. Zwei Fliegen mit einer Klappe geschlagen: Drei Tage Lieblingsfach Malen und drei Tage Thomas.

Der andere Kurs ist nach längerer Überlegung thomasfrei. Denn Hans ein ehemaliger Schüler, der Schauspieler geworden ist, gibt einen Improvisationskurs. Thomas ist in einer Diskussionsrunde eingetragen. Johanna will sich nicht die Blösse geben und Thomas alles nachmachen. 
Mit dem Hans ist es besonders, denn er hat eine Zeit bei Johannas Lieblingslehrer gewohnt. Johanna weiss es, weil Hans aus O`feld kommt, die Mutter hat es erzählt. Eines Tages ist er bei dem Lehrer vor der Tür gestanden und hat gefragt, ob er bei ihm wohnen darf. Welcher Mut, Mut zu dem eigenen Bedürfnis und der eigenen Person. Der Lehrer hat ihn aufgenommen. 

Er ist Dichter und somit Deutsch- und Poetiklehrer. Alt ist er, der Rücken von der Last des Ertragens gebeugt, der aufgeblähte Brustkorb, zu gross, und doch nach Innen eingefallen, umklammert die Lungenflügel, die flatternden Vögeln gleich, die dünnen Atemzüge raus pfeifen. Mit welch einer Liebe zum Wort er die Ge-dichte ausspricht. Jeder Buchstabe, jedes Wort auf einem goldenen Tablett tiefen Mitgefühls vorgetragen, mit leiser, pfeifender melodiöser Stimme. Johanna sitzt mit offenem, staunenden Mund da und lauscht der Musik von Gryhpius, Brentano, Eichendorff. Der Rest der Klasse tut es nicht, ist empört über die gefühlsduselige Störung in dem wichtigen Geplapper mit dem Nachbar. Kaum ein Tag  vergeht an dem der Lehrer nicht um neun Uhr aufspringt, seinen Stock nimmt und die Klasse verlässt, weil er es nicht länger erträgt. Die Vogelstimmen sind zu schwach für das Dickhäutergebrüll. Johanna möchte dann vor Wut weinen, weil sie sich schämt für die Rohheit und der Verlust der gedichteten Musik die weit geöffneten Ohren hart triff.

Und am nächsten Morgen steht er wieder da. Mit der Liebe zur Poesie und der Liebe zu seinem Beruf. Johanna spürt eine tiefe Verbundenheit zu dieser Hingabe und diesem Menschen, der bereit ist, sich jeden Tag neu zu opfern.

Hans wiederum hat sie auf der Bühne gesehen, in der Schule hat er ein Abendprogramm gemacht, das Johanna mit den Eltern besucht. Hans spielt die Sprache, die Gedanken und Poesie seines Lehrmeisters. Johanna lauscht den Worten, die ihr Lehrer aus sich geschöpft hat, gierig. Es scheint mehr zu geben, mehr Menschen, die unterscheiden zwischen sich, der Liebe- und der Anderswelt. Ich bin nicht allein, wispert ihr Herz entzückt.

Hans kommt Johanna vor, wie ihr Geheimnis, er kann Türen öffnen, die Thomas nicht öffnet, Türen hinter denen eine wundervolle Nahrungsquelle ist, die Johanna wie ein Spatzenkind im Frühling rund und glücklich macht. Ich kann mich ausfüllen, lernt Johanna, mit der richtigen Umgebung, kann ich, die Höhlenhockerin, mein ganzes Bewusstsein ausfüllen und -sein. Johanna möchte dieses Geheimnis nicht gleich mit Thomas teilen, vielmehr, will sie diese Schatztruhe alleine erkunden, geh` du nur zu deiner Diskussionsrunde\dots 

Am ersten Tag gibt es eine Einführung. Alle versammeln sich im Eurythmiepavillion: „Kunst, Wissenschaft, Religion“ ist das übergeordnete Thema. Drei Disziplinen, die, würden sie miteinander schaffen, so die These des vortragenden Lehrers, Neues, Grosses hervorbringen könnten. Johanna ist von der Idee begeistert, sie entspricht ihrem Grundgefühl,  die Dinge aus mehreren Perspektiven umfassend zu betrachten.

Mit Elan betritt sie den Malraum, der, mit vielen Oberlichtern bestückt, im Dachstuhl der Schule ein warmes Nest bietet. Zu ihrem Leid sieht sie Thomas mit dem Mops hereinspazieren. Der Mops, ein selbstbewusstes, mit hübschem, runden Gesicht ausgestattetes Mädchen aus Thomas Klasse, hat aus Eifersucht diesen Namen bekommen. Er ist offensichtlich an Thomas interessiert und weicht nicht von seiner Seite. Johannas Minderwertigkeitsgefühl flucht und tobt.

Die letzte Bankreihe, hinter den schweren Dachstuhlbalken, wählen alle drei aus. Johanna, züchtig und Desinteresse heuchelnd, lässt einen Tisch Abstand zu Thomas. Der Kunstlehrer hat ein Thema gewählt, sie sollen einen Gegenstand in die Abstraktion führen in mehreren Schritten auf mehreren Blättern.

Johanna hat Mühe. Sie findet keinen Gegenstand. Zeichnet und kritzelt, verwirft wieder. Thomas ist viel zu Nah! Sie spürt Herzklopfen und gleichzeitig die tiefe Ruhe, die Thomas für sie bedeutet, wird herausgerissen, weil der Mops versucht mit Thomas zu kichern und zu schwätzen, was dieser brummig, ganz in seine Zeichnungen vertieft, höflich, aber bestimmt und konzentriert abwehrt. Der Lehrer macht eine Runde, aber bei Johanna gibt es nichts zu sehen. Sie selbst macht im sicheren Abstand eine Runde, staunt über das eine, beneidet das andere, belächelt und wünschte sich, selbst eine Idee zu haben.

Langsam füllt sie drei  Näpfe mit Farbe, lichter Aquarellfarbe, Zitronengelb, Karminrot, Preussischblau. Ein Blatt zieht sie auf. Klebt es fest, beginnt zu malen und dann kommt das Füchslein zu ihr. Eingerollt in grünes Gras, liegt es, mit rotbraunem Fell, hellem Bäuchlein, dunkler, bläulicher Schnauze und dunklen Läufen und schläft friedlich. Johanna ist überrascht. Ein Fuchs. Thomas an ihrer Seite verschwimmt in den Hintergrund. Denn da liegt das Füchslein und beginnt in seiner  Form und mit seinen Farben zu tanzen.

Die Schleuse, gewahr werdend der inneren Ruhe und Konzentration, öffnet sich; langsam, Farben, Formen strömen durch die Schädeldecke aus einem von ihr ausserhalb liegenden, unendlichen, luftig durchlässigen Raum direkt über dem Kopf, im Hirn einen farbig-rasanten Film erzeugend, gleich in die Hände, die über das Papier fliegen, um keinen der wundersam hervor perlenden Töne zu verpassen. Einem Orchesterstück gleich, einer Symphonie, entstehen die Bilder und Musik ähnlich, kommen und gehen sie, wieder kehren die Themen in neuer Variation.

Ein neues Blatt, ein neuer Fuchs, dieser ist aufgewacht und schaut erstaunt auf seine verlängerten Läufe, seinen zu grossen Schwanz und die Farben seines Fells, die sich den Grundfarben, Rot, Gelb und Blau annähern. Johanna holt sich gleich drei weitere Bretter um Papier aufzuziehen. Froh ist sie über ihre Ecke, ganz hinten, im Dunklen, aber mit genug Platz. Sie legt alle Bretter um sich auf den Boden, dem freien Tisch neben Thomas und schwelgt in den Klängen füchsischer Musik.

Der Lehrer macht seine Runde. Er bleibt bei Thomas stehen und fragt, was er malt. „Ich will die unterschiedlichen Reaktionen in den Gesichtern bei einer bestimmten Situation festhalten.“ Johanna wundert sich, was hat das mit dem Thema zu tun. Thomas hat mehrere Bilder gemalt, die alle erschreckte, gequälte blasse, grüne Gesichter zeigen. Der Lehrer nickt stumm. 

Erstaunt schaut er auf die vielen Malbretter, die um Johanna gewachsen sind. Da sie im Schichtverfahren malt, arbeitet sie an fünf Bildern gleichzeitig. „Sie macht echt gute Bilder!“ Thomas steht plötzlich neben dem Lehrer. „Die sind genial!“ Setzt er nach und strahlt, schaut aber den Lehrer an, der verwirrt schaut. „Ja, die sind wirklich schön.“ Der Lehrer gibt Johanna ein, zwei Ratschläge und geht weiter. Alles ist wie zuvor. Johanna steht wie ein begossener Pudel da, wie ein begossener Fuchs.

Der Mops ist Johanna in den Kurs von Hans gefolgt. Erst ist sie wütend.  Aber, der Mops ist nett und der einzige, den sie im Kurs kennt. Es sind viele Neuntklässler dabei und sowohl der Mops, als auch Johanna finden, sie sind bei den Paarübungen das beste Team. Es ist merkwürdig, wenn Thomas nicht da ist, verstehen sie sich auf unverbindliche Weise gut, sobald er auftaucht, sprechen sie nicht mehr miteinander. Der Mops legt kräftig zu und fällt Thomas in der Pause von hinten um den Hals, aber er hat es nicht gern und schiebt ihn wieder weg. Johanna ist froh und gleichzeitig ängstlich, würde er es mit ihr auch so machen. Sie will auf der Hut bleiben, sich und ihr Herz in sicherer Distanz wissen.

Der zweite Tag. Mit dem letzten Bild wartet Johanna, muss sie warten. Der symphonische Bilderraum über ihrem Kopf spielt auf allen Farben und Formen gleichzeitig, es gibt kein klares Bild, den Händen fehlt der Auftrag. In der Mittagspause ist Johanna allein im Malraum, sie lauscht den Tönen, kann sie ein Muster erkennen? Sie wandert im Raum umher. Betrachtet den präzise abgezeichneten Kopf des Davids von Michelangelo, den eine Zwölftklässlerin gekonnt in viele Dreiecke zersplitter hat. Mit Grausen schaut sie auf Thomas Werk. Schreckensgestalten, gut gemalt, aber schockierend in der Wirkung. Thomas hat nichts abstraktes gemalt, oder doch: Gespenster.

Sie hebt den eigenen Pinsel, die Tür klappt, polternde Schritte auf der Treppe. Thomas zögert ein winzigen Moment. Dann kommt er herauf. „Ich will mir mal meine Bilder in Ruhe ansehen“. Johanna sagt nichts, vielleicht ein kleines Lächeln auf den Lippen. Thomas nimmt seine Bilder und stellt sie am hellsten Fleck am anderen Ende des Raumes auf eine Staffelei. Mit geübtem Malerauge tritt er einige Schritte zurück und  schaut auf seine Gespenstergalerie.

SAG` WAS! Die Stille dehnt sich wellenförmig aus, setzt sich in den Kehlen fest und wird dort zu Blei. Sag` was, schreit die klare, liebevolle Stimme in Johanna, geh` zu ihm, das ist deine Stunde, deine Gelegenheit ihm gleiche Referenz zu erweisen, wie er sie dir aus voller Begeisterung gab.
Ein anderer Teil bockt und in dem Kampf, ist plötzlich die knisternde, schwarze Stimme, die flüstert: „Er ist ein Mann, er wird dich richten!“ Es ruckt in Johannas Kehlkopf, die schwarze, vertrocknete Schnur reisst dem Lebendigen die Worte weg. Der Schmerz wird unerträglich, Johanna ist froh, Thomas geht. Tiefe, nagende Angst bleibt. Und unendliche Trauer über die eigene Unfähigkeit, dem liebsten, was sie kennt, ein Hauch Miteinander zu gönnen.

In der Pause gibt es ein Kuchenbüfett. Alle drängeln und schieben, Essen, Essen,\dots 

Johanna läuft das Wasser im Mund zusammen, sie schlängelt sich vorsichtig voran und wird von einem fröhlichen Thomas zur Seite gedrängt: „Ich hab` noch keinen Kuchen!“ He, denkt Johanna: „Ich auch nicht! Bring mir mal welchen mit!“ „Ja, klar!“ Heldenhaft  bahnt sich Thomas seinen Weg zur vordersten Kuchenfront und reicht Johanna  zwei Stücke, bevor er sich selbst bedient. „Danke schön“ grinst Johanna und setzt sich in eine ruhige Ecke, um den Thomaskuchen zu geniessen. Schwerfällig rutscht der  Kuchen an dem Herz vorbei, es klopft im Hals. Der Druck lässt nach, ich kann mit Thomas in der Anderswelt reden.

Am dritten Tag gibt es ein Abschlussfest am Abend und alle Kunstwerke werden in der Aula aufgehängt. Johanna hat es, der letzte Fuchs ist rechtzeitig in ihre Hände gebeamt worden. Er besteht aus wenigen Strichen in den drei Grundfarben. Sie hat eine Bilderreihe geschaffen, die als Ganzes betrachtet werden kann, aber ebenso jedes einzelne Bild für sich steht.

Sie umrundet die Stellwände, neugierig, was es alles gibt. Thomas steht mit Maria aus seiner Klasse vor Johannas Bildern. Leise schiebt sich Johanna von hinten näher heran. „Die sind so cool geworden, die Bilder.“ Thomas schwärmt als sei es sein Verdienst, sein Werk. Maria schaut: „Von wem sind die?“ „Johanna!“ „Johanna? Welche Johanna? Aus der Neunten?“ „Nein!“ empört sich Thomas: „Zehnte!“ Thomas dreht sich um, sieht sich in Johannas Augen und stürmt davon, Röte auf den Wangen.


\sterne


Johanna atmet auf, Thomas war von dem Anfänger- in den Fortgeschrittenen Flötenkreis gewandert und sie darf jetzt auch. Unerträglich ist es ihr geworden. An den geliebten Dienstagen Thomas nur beim Verlassen des Raumes zu sehen, wenn die Fortgeschrittenen kommen und ihre Gruppe geht.
In der neuen Gruppe sind alle älter, gute Spielerinnen, die Musik ist anspruchsvoller macht mehr Freude. Die Mädchen kleben wieder wie Fliegen an Thomas. Sabine, deren Ausschnitt mehr zeigt als verbirgt, lacht und scherzt am lautesten. Keine Gelegenheit lässt sie aus in jeder kurzen Spielpause mit Thomas zu flirten. Johanna bleibt stoisch und ruhig, denn Thomas bleibt unverbindlich. Stattdessen schaut er Johanna tief in die Augen, während er mit Sabine plaudert, die misstrauisch zwischen Johanna und Thomas hin und her schaut und ihr Lachen wird schriller. Thomas lacht auch, aber er wirft Johanna neckisch den Bleistift zu, mit dem die Noten umkringelt und beschrieben werden. Johanna staunt, Thomas ist ausgelassen, streckt ihr spielerisch die Zunge raus.



\sterne


Da Frau K. einen Pastoren zum Gemahl  hat, häufen sich die kleinen Vorspiele des Flötenkreises  in der Novemberzeit um die Alten bei Kaffee und Musik vom Herbst abzulenken. Thomas macht ein ganz entsetztes Gesicht, als Johanna meint, sie könne wegen einem Termin nicht zum Vorspiel kommen. Aber schliesslich kann sie doch. Aber Thomas ist krank, an diesem Tag. Und Johanna sitzt alleine da mit Friedhelm, dem anderen Bass. Und ausgerechnet in dieser Woche sind zwei Vorspiele.

 Zweites Vorspiel im Gemeindehaus der Kirche. Missmutig fährt Johanna durch den Regen nach Rendsburg. Als sie ankommt ist niemand da, die Tür ist verschlossen und sie wartet und tröstet sich damit, dass das Gemeindehaus jedenfalls einen überdachten, grossen Eingang hat. Die anderen Mitspieler sammeln sich langsam und da kommt Frau K. beladen mit  Flöten, Taschen und Noten und Thomas folgt ihr und schaut mit breitem Grinsen in Johannas entgeistertes Gesicht. „ Thomas ist mir zuliebe mitgekommen, obwohl er krank ist,“ sagt Frau K., aber Johanna fragt sich, warum Frau K.  ihr dabei in die Augen schaut und grinst wie Thomas. Alle zusammen drängen sich ins trockene Warm und beginnen ihre Instrumente auszupacken. „Du und du kommt mit die Notenständer aufbauen.“ Frau K. geht mit den Auserwählten und zu ihrem Schrecken stellt Johanna fest, sie sitzt mit Thomas alleine da! Eine weitere Welle Schamröte ergiesst sich über Johanna, sie verwünscht Frau K., die alte Kupplerin und wünscht sich weit fort um ihr Herzklopfen auf ein erträgliches Mass unter Kontrolle zu kriegen. Thomas ist dafür um so munterer und Johannas Ohren bemühen sich ihr zu übermitteln, dass er sie nicht nur unentwegt anschaut, sondern mit ihr plappert. „Wie findest du denn den dicken Müller?“ fragt Thomas. „Bäh, der ist eklig!“ entfährt es Johanna, während sie an den dicken Lehrer denkt, den sie mit ihrem Gelächter über seine Leibesfülle zur Weissglut getrieben und sich zum Feind gemacht hat, was eine erhebliche Leistung ist, weil den nichts aus der Ruhe bringen kann. „Ach, Mülli ist doch süüsss!“ schwärmt Thomas. Johanna fragt sich, warum sie es verdient hat, dass ausgerechnet der unerotische „Mülli“ sich in das Gespräch geschlichen hat, ihre erste Lektion in männlicher Romantik? Johanna scheint, sie hat mit ihrer heftigen Antwort das Geplapper mit erschlagen, denn Thomas sagt nichts mehr  und sie sind froh, als sie zum Vorspielen gerufen werden. Thomas verscheucht Friedhelm von dem mittleren Stuhl, damit er neben Johanna sitzen kann.
 
Der grosse Kirchenchor gibt ein Konzert, Thomas ist auch dort Mitglied. Johanna ist mit dem Flötenkreis dabei. Nach der Aufführung treffen sich alle im Gemeindehaus zum Büfett.

 Elisabeth ist dabei,  Johannas Cousine, drei Wochen älter als sie, Schwesterersatz und Freundin. Frau K. erlaubt Johanna und Elisabeth dort zu bleiben. Und prompt sitzt Elisabeth neben Thomas. Johanna auf ihrer anderen Seite. Johanna hat weiche Knie und ist hin und her gerissen, mit Elisabeth wichtige, geheime Mädchentuschelgespräche zu führen, oder zu Thomas hinüber zu schauen, der ebenfalls abgelenkt ist, weil er Besuch aus England hat und sich, staunt Johanna, weltmännisch auf Englisch unterhält. 
Thomas Blick kann Johanna kaum aushalten, schnell muss sie wegsehen, denn sonst müsste alles in ihr übersprudeln. Es wirbelt in ihr. Obwohl Elisabeth dazwischen sitzt ist Thomas ständig in und um sie herum. Er schwatzt mal mit ihnen, dann wieder mit seinem englischen Freund, dann kann Johanna verschnaufen. 

Elisabeth ist danach fasziniert und aufgeregt wie Johanna: „Eh, Thomas hat dich die ganze Zeit angestarrt.“ Johanna steht unter Strom, was passiert da? Wie kann ein Mensch tief in ihr Inneres eintauchen, leise und geschmeidig und von dort ihren Körper mit intensivem Licht füllen? Johanna wundert sich, sie wollte sich nicht einfangen lassen, nicht von dem, was Thomas ihr als Bild spiegelte. 

Sie darf in ihn hinein spazieren und findet eine gemütvolle, warme Stille, jedes mal. Sie versteckt sich, lauert auf ein Funken Ablehnung, aber, wenn sie ihm begegnet findet sie Platz für sich. 

Ganz traut sie dem nicht, sein Blick, der sie in sich ein saugt, macht ihr Angst. Und gleichzeitig sehnt sie diesen ruhigen, für sie allein bestimmten Platz herbei, in den sie die Augen einladen. Schliesslich wird die Spannung Johanna zu viel. Wenn es diesen wunderbaren Ort für sie geben sollte, dann will sie ihn, jetzt.


\section*{Karmaloka I (Rendsburg, 18.11.1986)}
\addcontentsline{toc}{section}{Karmaloka I (Rendsburg, 18.11.1986)}


Heute, nach dem Flötenkreis, da will sie Thomas ansprechen, ob sie sich treffen können, weil sie ihm eine Geschichte erzählen will. Sie hat alles lang überlegt und morgen wäre doch ein guter Tag, denn Morgen ist Buss-und Bettag, da haben alle frei.

Aber, als der Flötenkreis zu ende ist, wird Thomas von seinem Schulfreund Dani abgeholt. Dani ist ein lieber Chaot, aber Johanna möchte Thomas nicht vor seiner Nase um eine Verabredung bitten. Ob das ein Zeichen, ein schlechtes Omen ist? Thomas wird sonst nie abgeholt, weil er nach dem Flötenkreis in der Volkshochschule zum Zeichnen geht. Nachdenklich fährt sie nach Hause, soll sie den Plan fallen lassen? Aber die Unruhe innen drinnen ist riesig. Nach dem Abendbrot ruft Johanna Elisabeth an. „Soll ich Thomas anrufen?“ Elisabeths Stimme klingt interessiert: „Ja, mach doch!“ „Was soll ich denn sagen? Und wenn Frau K. am Telefon ist?“ „Ist doch nicht so schlimm, ruf einfach an und frag, ob er Zeit hat.“ „Okay, dann Tschüss!“ Wirklich beruhigt ist Johanna nicht, aber Elisabeth kann nicht wissen, wie es sich anfühlt, das Gefühl als müsste der Bauch platzen, vor Unruhe und Aufregung.

Elisabeth hat gesagt: mach`s. Johanna überlegt, wann er wieder zu hause sein könnte von seinem Kurs. Ist es richtig? Ist es nicht besser ihn direkt anzusprechen? Dann kann sie in seine wunderschönen Augen sehen und ablesen, was die dazu meinen. Sie weiss, wenn sie vor ihm steht und ihn anschaut, dann kann er nicht nein sagen. Aber morgen ist der freie Tag. Johanna  sieht sich mit Thomas in der Stadt umherziehen. Sie verbringen den ganzen Nachmittag miteinander, plaudern, lachen und freuen sich an einander\dots 

Während Johanna träumt, fühlt sie das rosa, orange starke Band, das aus ihrem Nabel in den von Thomas führt, stark, es kommt ihr vor, als hätte sie tausende male mit Thomas gesprochen, tausende male eine Verabredung mit ihm abgemacht. Die Bilder sind real vor den inneren Augen, es kann daher morgen nur ein wunderbarer Tag werden\dots 

Um halb neun geht Johanna in die Wohnstube der Grossmutter und holt sich das Telefon in die Küche. Sorgfältig verschliesst sie die Türen, da kommt die Mutter rein gestürmt: „Wen willst du so spät noch anrufen?“ „K.s“ „So spät?“ Die Mutter runzelt die Stirn. Johanna blickt trotzig auf den Boden, ihr bleibt nichts erspart: „Thomas ist nicht früher zu Hause.“ Die Mutter stutzt und schweigt und Johanna darf sich mit dem Telefon in ihre Wohnstube setzten, Türen zu.

Johanna geht alles durch was sie sagen will und was Thomas antworten wird. Ihre Hände zittern, das Herz, vorher standhaft ruhig, schlägt bis zum Hals und ihr Ringfinger beginnt zu kribbeln. Der Finger kribbelt, wenn Thomas in der Nähe ist, oder wenn kurze Zeit später ein Unglück passiert. Was bedeutet das? Auch der Mittelfinger beginnt sich auf zu führen  als sei eine Herde Ameisen darin unterwegs. Johanna bekommt Furcht, doppeltes Unglück? Aber, morgen ist der freie Tag, ihr Tag, den sie für sich und Thomas auserkoren hat\dots 

7\dots , 2\dots , 7\dots, die Wählscheibe macht behäbig ihre Runden\dots , 6\dots, 6\dots : „Tuut! Johanna wird bleich, der Telefonhörer schlottert an ihrem Ohr\dots  „Tuut!“ Ich leg auf, denkt Johanna, ich kann nicht\dots  „Tuut!“

Auflegen, schnell, so leg` doch auf: „K.!“ Scheisse!

„Hier ist Johanna, kann ich bitte Thomas sprechen?“ Johanna staunt, ihre Stimme ist ruhig und Frau K.s Stimme findet es scheinbar normal, dass Johanna Thomas anruft\dots  „Ja! Moment\dots “ Johanna hört wie sie den Hörer weglegt und weggeht. Er ist da, Oh, nein, auflegen, leg` auf\dots  „Hallo?“ Wie hört sich seine Stimme an, Johanna erschreckt, die Stimme ist fremd.

„Hier ist Johanna!“ „Ja, Johanna?“ er spricht zu mir, wie zu einer Idiotin, durchfährt es Johanna, was habe ich bloss angerichtet. Johanna wird es Angst, dem Menschen, dem sie jetzt begegnet, ist nicht der, den sie anrufen wollte. Wo ist der freundliche, tollpatschige, schöne Mann geblieben, die verschlingenden braunen Augen\dots  Alles ist aus, aber sie kann nicht stoppen, zu oft haben die Worte, die sie sagen wollte in ihrem Kopf getanzt, zu oft, sich Bilder ins Hirn gebrannt, von einem Thomas, der ihr zuhört, der ihr antwortet auf alle die vielen Fragen, die zwischen ihnen sind.

„Ich wollte dir eine Geschichte erzählen und du sollst mir sagen, wie sie zu Ende geht.“ „Eine Geschichte?“ Die Gegenüberstimme ist verwirrt: „Ähm, die Geschichte hat ein schlechtes Ende!“ „Du kennst sie doch noch gar nicht.“ „Ja, also, die Geschichte hat keine Ende! Sie hat kein Ende.“ Ein kleiner Triumph hat sich in die Stimme geschlichen, aus getrickst, sagt sie, sie weiss nicht, dass sie Wahrworte gesprochen hat, die seit 500 Jahren gültig sind\dots 

„Kein Ende?“ Johanna ist verblüfft, was redet er da? Schweigen.
Er denkt, ich erzähle, Johanna besinnt sich, warum sie angerufen hat: „Am Telefon wollte ich nicht erzählen, ich wollte dich fragen, ob du morgen Zeit hast?“ „Da muss ich auf den Stundenplan gucken.“ Der Hörer wird zur Seite gelegt, Johanna hört Schritte. Stundenplan? Morgen? Übrigens, denkt sie, morgen hättest du sechs Stunden. „ Morgen ist schlecht, da hab` ich sechs Stunden.“ Johanna wird es peinlich, höflich und ruhig, wie zu einem kleinen Kind sagt sie: „Morgen, haben wir schulfrei.“ „Scheisse, Scheisse, ähm,\dots ähm, morgen ist schlecht, gaanz schlecht\dots “ Es geht noch schimmer, denkt Johanna und sie spürt Wut- und Schamröte in ihrem Gesicht Ringkämpfe austragen.

Johanna will dieses blöde Spiel beenden, aber sie will die Antwort auf ihre Frage und dabei in seine Augen sehen und zwar vor dem nächsten Dienstag, wenn sie wieder im Flötenkreis hocken. „Und sonst, Donnerstag?“ „ Muss ich auf den Stundenplan gucken.“ Der Telefonhörer wird wieder weggelegt. Mein Gott, denkt sie, was für ein Theater, vier Stunden\dots , beide! „Donnerstag geht, da hab ich vier Stunden.“ „Ich auch, und wo?“ „In deiner Klasse?“ Was, wenn welche da sind, denkt Johanna: „Nein, nachher sind da welche.“ „In der Raucherecke?“ Johannas Geduld ist zu ende, die Bilder, die sie sich gemacht hat, zerbröckeln, Thomas gibt es nur in der Raucherecke, in der Schule, in der verhassten Anderswelt\dots  Die Seelenwasserglocke, bisher einen kleinen Spalt geöffnet, vibriert, mit unhörbarem Getöse schliesst sie sich langsam, sinkt herab. Johanna will es hinter sich haben, nicht bis Donnerstag warten, nicht in der Raucherecke, nicht in der Andersweltschule mit Thomas reden, auf dem Weg in ihre hinterste Höhle, dreht sie sich noch mal um: „Weisst du was? Du kannst es mir jetzt gleich sagen.“ „Aber, ich weiss ja nicht.“ „Sag` einfach, was du denkst.“ „Kann es sein, dass du in mich verknallt bist?“ Nein, denkt Johanna, das kann nicht sein. Was ich fühle, ist nicht „verknallt sein“, bisschen verliebt sein,\dots 

Johanna spürt den letzten Luftzug, das Geräusch, der sich schliessenden Seelenwasserglocke, übertönt alle Empfindungen, dann ist Stille. Johanna geht weiter auf die Höhle zu, sie schaut sich nicht um, lässt ihre Hülle das Gespräch beenden.

„So Ähnlich.“ „So ähnlich?\dots  Die Gefühle werden nicht erwidert.“ 
Kein Wort durchdringt das Glas. „Ich bin das gewöhnt.“ Die Scham traut ihren Ohren nicht, warum hat Johanna das gesagt? Aber die ist fort. „Okay.“ „Gut.“ „Tschüss!“ Jedenfalls hat sie als erste auf gelegt, schnieft das Selbstbewusstsein.
Die Mutter kommt rein: „Möchtest du lachen oder weinen.“ Mit glasgetrübtem Blick schaut  Johanna sie an, eine lächerliche Vorstellung hat sie sich mit Thomas geliefert: „Beides.“ Die Mutter schaut irritiert. Johanna erzählt von dem Gespräch. Das letzte, das mit den Gefühlen, hat Johanna vergessen, jedes Wort weiss sie auswendig, aber diese? Der Sinn ist geblieben.
Die Mutter erlaubt Johanna mit der Grossmutter fern zu sehen, sonst gibt es wegen dem Fernseher lange Diskussionen, es läuft ein Heimatfilm, mit grossen Gefühlen. Johanna rutsch tief in den Sessel, lässt die Augen und Ohren und das Gehirn den Film folgen, während der Rest in der Stille  der Höhle ringt. Stolz und Selbstbewusstsein sind empört, sie fordern den eigenen Tod oder Rache, Minderwertigkeitskomplexe zischeln, es sei klar gewesen, wie es enden würde. Sie alle zerren und reissen an Johanna, die leer und erschöpft sich abwendet.

Johanna hat sich weit verkrochen und spürt nichts von den nächsten Tagen. Das Gespräch folgt ihr, die einzelnen Worte laufen in einer Endlosschlaufe in ihrem Kopf, bis zu dem entscheidenden Satz, dort folgt die Sendepause, Wasserglockenrauschen\dots 


\sterne


Die Schule, eh schon eine Qual, weil die einzige Freundin Mira für ein Jahr nach England gegangen ist, ist unerträglich. Der Stolz bekniet Johanna eine Auszeit auf dem Mond zu nehmen\dots 

Am Freitag trifft sie im Gang erstmals auf Thomas, unvermeidlich und vorhersehbar, aber ein Rest Hoffnung, bildete sich ein, Thomas könnte verschwunden sein. Sie sieht ihn von Ferne, auf alles ist sie gefasst, aber nicht auf die geballte Wut und Enttäuschung, die sie wie ein brüllender, trampelnder Stier nieder rennen. 

Aber, warum Wut? Der Gang liegt im Dunkeln, quadratisch erhellt, dort, wo eine Türe offen steht, es  sind nicht viele und in der letzten Tür steht Thomas und entfesselt den Stier. Warum Wut? Johanna begreift es nicht, sie, sie müsste wütend sein. Warum, vorwurfsvolle, gewaltige Wut? „Du hast alles kaputt gemacht!“ kreischt seine Stimme in ihrem Kopf. Aber, wenn er sie nicht will, er sich nichts aus ihr macht, was hat sie kaputt gemacht? Woher die Wut? Enttäuschung? Von was ist Thomas enttäuscht?

Eine Woche später vertraut ihr Elisabeth an, Johanna sei nicht die einzige, die Thomas angerufen hätte, nein, sie hätte gehört, es gäbe einige Mädchen, die ihn toll fänden und von denen hätten welche bei ihm angerufen und ganz ähnliche Gespräche gewünscht. Johanna ist entsetzt, warum hat Elisabeth das nicht vorher gesagt? Sie, eine von vielen Telephonterrormädchen, die Thomas bis in sein Zuhause plagen? Die Scham flucht, weitere Überstunden, findet, es reicht. Aber, kann Thomas nicht unterscheiden? Johanna ist sich sicher, sie hat es falsch angefangen, einen riesigen Fehler gemacht, aber sie ist keine von „denen“, sie hat auf eine offene Tür in Thomas Inneren reagiert, die jetzt verschlossen ist. Kann Thomas nicht unterscheiden?

Johanna verbietet sich jede Träumerei von Thomas, jeden Gedanken an ihn, er will sie nicht? Gut, es ist zu Ende.

Kein Ende. Es nimmt kein Ende, Thomas hat es gesagt und Johanna fühlt unter der Leere glüht das liebe, lichte Gefühl weiter, als sei nichts gewesen. Johanna bleibt dabei, Gedanken erlaubt sie sich, aber nicht die kleinste Träumerei. Das ist unendlich schwer, Ihr Herz sehnt sich nach dem Kopfkino, Bildern auf denen sie nach Thomas Hand greift, die Härchen auf seinem Arm zählt, seine vollen Lippen küsst\dots Verboten! Aller strengstens verboten. Sonst, werde ich völlig verrückt, denkt sie.
Was bleibt sind harte, kalte Gedanken. Gedanken, die kreisen und kreisen um den einen unverständlichen, vergessenen Satz, der einem bissigen Hund gleich, vor Johannas Höhle hockt und sie blutrünstig knurrend begrüsst, wenn sie die Nase vor streckt. Sie kann nicht hinaus, der letzte Rest Wärme, den sie in der verbliebenen Liebe spürt, könnte verloren gehen und sie müsste vertrocknen.



\section*{Nachklänge Beethovenscher Musik}
\addcontentsline{toc}{section}{Nachklänge Beethovenscher Musik}



\begin{verse}
1.\\
Einsamkeit , du stummer Bronnen,\\
Heil`ge Mutter  tiefer Quellen\\
Zauberspiegel innrer Sonnen, \\
Die vor Tönen überschwellen:\\
Seit ich durft in deine Wonnen\\
Das betörte leben stellen,\\
Seit du ganz mich überronnen\\
Mit den dunklen Wunderwellen,\\
Hab zu funkeln ich begonnen.\\
Und nun klingen all die hellen\\
Sternensphären meiner Seele,\\
Deren Takt ein Gott mir zähle.\\
Alle Sonnen meines Herzens,\\
Die Planeten meiner Lust,\\
Die Kometen meines Schmerzens\\
Tönen laut in meiner Brust.\\
In dem Monde meiner Wehmut\\
Alles Glanzes unbewusst\\
Muss ich singen und in Demut\\
Vor den Schätzen meines Innern,\\
Vor der Armut meines Lebens,\\
Vor den Gipfeln meines Strebens\\
Ew`ger Gott, mich dein erinnern.\\
Alles andre ist vergebens.

2.\\
Gott, dein Himmel fasst mich in den Haaren,\\
Deine Erde reisst mich in die Hölle,\\
Herr, wo soll ich doch mein Herz bewahren,\\
Dass ich deine Schwelle sicher stelle?\\
Also floh ich durch die Nacht, da fliessen\\
Meine Klagen hin wie Feuerbronnen,\\
Die mit glüh`nden Meeren mich umschliessen,\\
Doch inmitten hab ich Grund gewonnen,\\
Rage auf gleich rätselhaften Riesen,\\
Memmnon`s Bild, des Morgens erste Sonnen\\
Fragend ihren Strahl zur Stirn mir schiessen\\
Und den Traum, den Mitternacht gesponnen,\\
Üb ich tönend, um den Tag zu grüssen.

3.\\
Selig, wer ohne Sinne\\
schwebt, wie ein Geist auf dem Wasser,\\
Nicht wie ein Schiff die Flaggen\\
Wechselnd der Zeit und Segel\\
Blähend, wie heute der Wind weht.\\
Nein, ohne Sinne, dem Gott gleich,\\
Selbst sich nur wissend und dichtend,\\
Schafft er die Welt, die er selbst ist,\\
Und es sündigt der Mensch drauf\\
und es war nicht sein Wille!\\
Aber geteilt ist alles,\\
Keinem ward alles, denn jedes\\
Hat einen Herrn, nur der Herr nicht;\\
Einsam ist er und dient nicht.\\
So auch der Sänger.\\
\end{verse}

\begin{dichter}Clemens Brentano\end{dichter}

\section*{Einsamkeit (Rendsburg 1986-1991)}
\addcontentsline{toc}{section}{Einsamkeit (Rendsburg 1986-1991)}




Johanna ist verschwunden. Sie geht zur Schule. Sie ist zu Hause, aber in Wahrheit ist sie von der Welt draussen abgeschnitten. Der Lärm der anderen Seelen, die sich suchen und ausprobieren, ihre Rollen finden wollen und sich im „So-Sein“ üben, kann sie nicht ertragen. Warum?

Johanna hört, wie ein schlecht eingestellter Radiosender die Zwischentöne, die Untertöne. Ängste, Aggressionen und Lügen, Süchte und Lüste, sie alle prasseln ungefiltert auf sie ein. Die der Lehrer, die der ungeliebten Klassenkameraden, die der Eltern und die von Thomas. Das Radio abstellen, was wäre das schön, aber sie weiss nicht wie das geht.

Wenn sie mit den Menschen versucht zu sprechen, durchdringen sie die unbewussten Informationen des Gegenüber und sie reagiert darauf. Die anderen finden das abstossend, wer möchte mit jemandem sprechen, der die Leichen im Keller aufzählt, während man selbst sich im besten, stärksten Licht präsentieren will?

Johanna selbst wird zusätzlich von den eigenen Ängsten geplagt. Sie sieht die Mitschüler, die Masken, die sie anprobieren und empfindet die Unehrlichkeit und Herzlosigkeit und will es selbst anders machen. Dabei stösst sie an die eigene Glaskugel. Sieht sich aus drei Metern Entfernung, kann weder Stimme, noch Handlung ergreifen und findet sich so selbst als Maske, Fratze, unecht und herzlos.

Dazwischen Thomas, seine ruhige, bedächtige Art, der warme Seelengrund, beginnt wie alle anderen hinter einer Maske zu verschwinden. Dort erlebt sie wie von der Innenansicht die Verwandlung. Sie spürt die Wärme in der Seele und sieht, wie Thomas sich verschliesst und betont cooles und arrogantes Benehmen pflegt. 

Johanna ist hin und her gerissen, mal erlebt sie den äusseren Schein und ist empört über das Schauspiel, dann, in einem unbedachten Augenblick, einem kurzen unabsichtlichen Moment des Blickes erstrahlt wieder das buntschillernde, geliebte Seelenpanorama. 

Was passiert da?

Tagebuchseiten werden gefüllt, eine um die andere. Wie kann ich jemanden so intensiv wahrnehmen, wenn nichts zwischen uns ist? In wiederkehrenden Rhythmen wird dies zur zentralen Frage. 

Die wiederkehrenden Kreisläufe führen immer an einen Punkt. Johanna liebt Thomas. Es geschieht mit ihr. Sie kann es nicht beeinflussen. Sie kann sich für eine gewisse Zeit einreden, sie fände ihn schrecklich, kann sich eine Weile in der Gedankenwelt  aufhalten, urteilen, verurteilen, analysieren. Aber die Gefühle schleichen sich langsam ein, Empfindungen, sie wachsen und brechen sich  eine Bahn hin zu dem überwältigenden Gefühl von „Ich liebe Dich!“.  Fertig. Punkt.

Aber Johanna hat sich träumen verboten. Kein Kopfkino, dadurch erlebt sie ihr Liebegefühl losgelöst, allgemeiner, eingekuschelt in die Zweisamkeit, wandert die Liebe umher und bestaunt die Natur, Pflanzen, Tiere, den Himmel. In diesen glücklichen Ruhezeiten fühlt sich Johanna getragen und als Einheit nicht mit Thomas allein, sondern mit sich und der Welt. Festhalten? Geht nicht. All zu schnell folgt der bissige Wörterhund, kläfft: „ Du darfst nicht glücklich sein mit dem Gefühl, denn Thomas hat gesagt\dots “ Die nächste Runde beginnt.

Halb wahnsinnig wird Johanna von dem Verbundenheitsgefühl. Sie beobachtet jeden Mucks den Thomas macht, wenn sie ihn sieht. Jede Bewegung, jeden Wimpernschlag und gleichzeitig beobachtet sie sich, wie sie Thomas beobachtet. Beobachtet wie sie geierartig auf jede Bewegung lauscht und giert, wie Thomas flieht vor der saugenden Kraft. Johanna weiss, sie erzwingt auf diese Weise Reaktionen, selbst, wenn sie aus den Augenwinkel schaut und macht als sähe sie Thomas nicht. Enttäuscht und traurig ist sie, schämt sich, läuft Thomas davon. Trauer und Scham, weil sie sich ihres Übergriffs bewusst ist. 

Und dann, wenn sie mit bitteren Selbstvorwürfen Abstand nimmt, sich von Thomas zurückzieht, ihm aus dem Weg geht, dann kommt der ersehnte Blick, die erhoffte Geste. Die innere Seelenregung von Thomas hin zu ihr. Und die Einsicht, Thomas lässt mich ebenso wenig los.

Aber, ist das Liebe? Ist die Verbundenheit, die sich Jahrzehnte in wiederkehrenden Träumen weiterhin zeigen wird, Liebe? Und wenn es keine Liebe ist, was ist es dann?

Mit dieser Frage richtet sich Johanna schliesslich an den Meister, an Gott. Was hast Du Dir dabei gedacht, Gott? Die Antwort ist ein beruhigendes Brummen, wie eine Katze schnurrt Gott und verrät nichts. Flehen, betteln, beleidigt sein, nichts hilft eine Antwort zu bekommen. Gott ist da, weit entfernt, aber da und schnurrt. Kein ersehntes Zeichen fällt vom Himmel, keine geheimnisvollen Zufälle wollen sich einstellen. Nur eine Einsicht: Alles hat einen Sinn.

Johanna beginnt den Sinn zu erforschen. Dabei hat sie Glück, denn die Eltern, die hilflos den schweren Allergieerkrankungen ihres Zwölf Jahre jüngeren Bruders gegenüberstehen, sind wie sie auf der Suche. Esoterische Bücher wandern ins Haus, werden studiert, diskutiert. Das Gesetz von Ursache und Wirkung, Makro- und Mikrokosmos, Reinkarnationslehren und der Glaube finden einen zentralen Platz in der Familie.



\section*{Träume}
\addcontentsline{toc}{section}{Träume}



Johanna schläft gerne und viel. Und sie träumt. Träume sind ihr eine verständliche Sprache. Eine vertraute Welt. Sie ist selbst im Wachen dicht vor dem Schleier, der die Träume verbirgt. C.G. Jung hilft ihr, die Sprache besser zu übersetzten.

Sie träumt viel von Thomas, aber er weicht in den Träumen aus, genauso wie im echten Leben. Er ist freundlich und lieb, aber er muss schnell fort, oder sie verliert ihn aus den Augen. Jeder Traum wird notiert, denn er bedeutet eine weitere, winzige Brücke, zu Thomas und hoffentlich, zum Verständnis, was gespielt wird. Die Träume in denen Thomas vorkommt sind anders, sie fühlen sich echter, dichter an. Die Bilder sind klarer, es wird mehr gesprochen und am nächsten morgen, ist das Empfinden der Stimmung, die der Traum hinterlässt stärker. Sie begleitet den Tag.
Einige wenige Träume, in denen Thomas nicht vor kommt, haben die gleiche Qualität.

Johanna träumt:

Ich gehe durch die Strassen einer Stadt. Die Häuser sind grau, die Fensterscheiben mit grünen Holzrahmen versehen. Ich bin in einer anderen Zeit. Die Menschen sind unheimlich. Sie haben eine schwarz-braune Gesichtsfarbe. Dort wo sich die Augen befinden, haben sie Papiermasken, wie Brillen an, auf die die Augen gemalt sind. Durch die Papieraugen sind sie blind. Ich gehe durch die Menschen,  die mich mit ihren Papieraugen ansehen. Und sie sind stumm. Ich habe es eilig, ich muss nach Hause zu meiner Familie, ich habe Angst.

Schneller und schneller laufe ich, während die dunklen Menschen mich an rempeln. Ich komme zu dem Haus meiner Eltern. Es ist von einer hohen Mauer umgeben. Ein grünes Tor aus Holz, gibt Einlass, es ist massiv und stark. Über dem Tor ist ein dreieckiger Giebel aus Stein. 

Nachdem ich durch das Tor geschlüpft bin, verschliesse ich es fest von Innen. 

Meine Familie finde ich im ersten Stock des Hauses. Sie sitzen und stehen in der Wohnstube zusammen. Es sind mehrere Personen. Die Frauen weinen und die Männer gehen unruhig auf und ab. Da weiss ich, was kommen muss. Wir sind Juden. Sie werden uns holen. Die Pappaugenmenschen, die Nazis, sie werden uns holen, jeden Moment\dots  ich versuche zu Trösten, obwohl ich selbst jung bin, ein junges Mädchen.

Es kracht an der schweren Holztür, die ich vor kurzer Zeit zugeschlossen habe. Im Raum beginnt ein Heulen und Fluchen, es gibt keinen Ausweg, keine Fluchtmöglichkeit, nur die Gewissheit, jetzt, jetzt kommen sie und holen uns\dots 

Das ohrenbetäubende Krachen wird zum Bersten, die Tür ist zerbrochen. Sie johlen, sie brüllen, sie kommen, die schweren Stiefel trampeln auf der Treppe, wir, wir sterben, in dem Moment, da die Schritte auf der Treppe zu hören sind. Wenn sie uns ermordet haben werden, sind wir schon längst tot gewesen, in diesem Augenblick\dots , sterben wir, die Schritte auf der Treppe\dots 

Mit klopfendem Herzen reisst es Johanna aus dem Schlaf. Das war kein Traum! Das fühlt sich anders an, das war eine Erinnerung! Sie zittert, sie kommen, ich sterbe, ich werde sterben, sie werden mich ermorden, ich will sterben, oh, Gott, lass mich jetzt sterben und verschone mich vor dem, was sie mir antun werden\dots 

Ich, eine Jüdin? Aber die schwarzen Menschen mit den Papieraugen? Die sind Traumgebilde\dots , perfekt Nazideutschland zeigende Traumgebilde, schwarze Menschen, die nicht sehen, weil sie ein Papierstück, statt Augen im Gesicht tragen, die stumm sind\dots 

Ich werde geholt, sie werden mich ermorden und dies, dies ist der letzte Augenblick Leben. Freiheit. Wenn sie kommen werden sie aus mir eine Tier machen, mich des Menschlichen berauben, ich werde Ameise und zertreten.
Dieses überwältigende Gefühl von wahr sein, sich erinnern, die Familie, die Angst.

Kann ich deshalb im Fernseher keine Leichen sehen? Fühle ich mich deshalb in jüdischen Geschichten, Gesichtern, Leben geborgen?  Wittere ich deshalb in den anderen Gefahr, die sich wie Herrenmenschen aufführen? Habe ich deshalb Angst, viel zu viel Angst? 

Heisst das Reinkarnation? Sich erinnern und Angst haben?

Mit dem Traum öffnen sich die Türen ein Spalt mehr, die Sinn und Unsinn zeigen. Und die verwirren, wie sollen mehrere Leben in einem Platz haben, wenn das eine kompliziert genug für mehrere ist\dots 

Neben diesem wird ein weiterer Traum Johanna den Rest ihres Lebens begleiten, weil er sie verblüfft und Fragen stellt. Während die säuberlich aufgeschriebenen, sich wiederholenden Träume von Thomas versinken.

Johanna träumt:

Ich komme in ein Gebäude, das violett ist. Die Wände sind violett. Der Fussboden ist seltsam geformt. Zwei schiefe Flächen treffen sich vor meinen Füssen sie werden zu den Seiten hin höher, so entsteht eine Rinne, ein Winkel, an ihrer Schnittgraden. Die Flächen treffen unten auf eine fünf-eckige Fläche, die zu den beiden hin schräg nach unten geneigt ist. In dem Raum befinden sich mehrere Menschen,  die winken mir, ich soll kommen. Ich rutsche plötzlich die Rinne zwischen den beiden Flächen runter und lande auf dem Fünfeck. Die Menschen heissen mich willkommen, ich sehe von unten, dass sich der Raum an der Seite zu einem weiteren öffnet, der strahlend Orange ist.

 Kein Spektakulum, aber ein Traum, der wiederum wie echt ist, wie sich erinnern. Was Johanna verblüfft, ist die Form des Raumes, völlig anders, als alles, was sie an Räumen kennt. Sie versucht den Raum zu zeichnen, was schwierig ist, wegen all der vielen schrägen Flächen.
 
Sie wird den Raum schliesslich finden! Mit einer Chorgruppe, die aus freiwilligen Mädchen des Schulchores besteht, studiert Johanna das „Stabat Mater“ von Pergolesi ein. 

Die Aufführungen führen sie in die Christengemeinschaft. Das Gebäude ist klein, grau aus Beton, rosig angemalt, übergehend in die Wohnung des Priesters.

Die Mädchen stellen sich, den schweren Altar aus Beton im Rücken, auf. Der Raum ist dürftig besucht. Er ist violett. Der Gesang beginnt. Während der Soli hat Johanna Zeit sich um zu sehen. Ihr Blick schweift zur Decke. Über dem Eingang treffen sich zwei schräge Flächen, und, über einen kleinen Absatz, enden sie in einem Fünfeck, das über ihr den Altarraum umfasst.
Der Raum, erfüllt von der Musik, kippt. Und Johanna rutscht, die beiden schrägen Flächen hinunter  in den Altarraum. Das plötzliche Bild verschwindet wieder. Johanna versucht es wieder zu holen, und vor allem, ohne den Kopf zu verdrehen sich die Decke des Raumes als Fussboden vor zu stellen. Tatsächlich, der merkwürdige Raum, den sie gezeichnet hat, ist die auf den Kopf gestellte Christengemeinschaft. 

Was?

 
 \section*{Andorra II(Rendsburg 1988}
\addcontentsline{toc}{section}{Andorra II(Rendsburg 1988}
 

 
„Du sollst dir kein Bildnis machen von Gott\dots “

11.Klasse, Klassenspiel. Ein Stück gesucht, gefunden, Max Frischs Andorra. Die Klassengemeinschaft, spielt ein Stück über Ausgrenzung, Vorurteile, Lügen glauben wollen, um das eigene Gesicht zu wahren, Dummheit der Masse und sich dahinter verstecken des Einzelnen. Mit dem üblichen Fleiss, Anbiedern, andere weg schubsen, um selbst den besten Brocken Selbstdarstellung zu erhalten, beginnt die Arbeit daran. Inhalt reflektieren? Nicht nötig, nicht erwünscht, wir sind nicht zum Nachdenken in der Schule, sondern um fürs spätere Leben den richtigen Zettel in der Hand zu haben, Freifahrschein zum elitären Egoputzen.

Jeder darf einen Rollenwunsch abgeben. Aus Trotz entscheidet sich Johanna für die weibliche Hauptrolle der Barblin auch wenn sie sie niemals bekommen wird und vernünftig, wie sie ist, für die des Priesters. Priester = Pastor = Thomas. Eine einfache Rechnung, schliesslich ist Thomas Vater Pastor.

Der Lehrer liesst jeden Wunschzettel vor. Er bemüht sich um Sachlichkeit, als er Johanna den Hauptrollenwunsch abschlägt, was sie ihm hoch anrechnet, denn die Empörung und Reaktion der Klasse, ist, obwohl sie es voraus geahnt hat, gewohnt schmerzlich.

Gegen den zweiten Wunsch hat niemand Einwände, wer will einen Pfaffen spielen? Obwohl die Rolle eine der grösseren ist. Katrin spielt die selben Rollen wie Johanna. Idiot und Pfarrer.

Obwohl der Idiot nichts sagt, er wankt einige male über die Bühne, ist die Rolle schwer zu lernen. Idiotisch laufen und authentisch wirken, ist schwer. Die Schauspiellehrerin gibt sich alle Mühe. Bei dieser Rolle kann sich Johanna nicht hinter Worten verstecken, allein der Körper darf sprechen. Johanna, die sich jedes Wort im Miteinander erkämpfen muss, bemerkt, wie schwer die Sprache des Körpers zu beherrschen ist. Will sie die Rolle spielen und fühlt sich in ihren Körper hinein, bewegt sie sich wie ein Wattemännchen. Die Ohren rauschen laut und die Augen, die von Aussen auf die Bewegung schauen wollen, sehen verschwommen. 


„Du sollst dir kein Bildnis machen\dots “ Mache ich mir Bilder? Fragt Johanna sich. Die Mechanismen, die Frisch in seinem Stück bis ins Detail bloss legt, kennt sie genügend. Aber macht sie dies mit anderen auch? Welches Bild habe ich von Thomas? Sie findet keines, das Bestand hat, es sind zu viele Puzzleteile, die sie hintereinander, gleichzeitig und durcheinander wahrnimmt. Bilder von anderen? Johanna fühlt sich wie ein Spielball. All die Emotionen, die die vielen Menschenradios ihr den ganzen Tag zu senden, vor allem die vielen, massiv negativen, lassen keine Bilder zu. So wenig, wie es möglich ist, sich von ständig laufender und viel zu lauter Radiomusik ein Bild zu machen, so wenig kann Johanna den anderen Bilder über stülpen. Sie sieht jeden doppelt, was er darstellen will nach Aussen und die Ängste dahinter. 

Verwirrt von den schizophrenen Eindrücken, hat sie genug damit zu tun, auf die widersprüchlichen „Äusserungen“ zu reagieren und sich selbst weit davon entfernt zu halten, um sich selbst nicht zu verlieren.

Der Lehrer lässt alle zu Hause ein Plakat zeichnen für das Stück. Wenige Striche reichen Johanna aus, um sowohl das Stück, als auch die eigene Situation auf beängstigende Weise lebendig werden zu lassen. Neben dem Stolz auf die Zeichnung, schluckt sie die Betroffenheit hinunter. Auch der Lehrer schluckt. Die Mitschüler schlucken nicht, sie ignorieren den Entwurf und streiten sich, wessen Bild am besten wäre. Diesmal greift der Lehrer ein, zwei Entwürfe dürfen ausgewählt werden und er bestimmt, einer davon ist der von Johanna.

Aufführung. Johanna, mit stoppelkurzem Haar und Brille, wird ihrer Männerrolle mehr als gerecht. 

Johanna fährt mit Elisabeth im Auto von Onkel und Tante mit, die eigens mit ihrer Pflegetochter aus Hamburg gekommen sind, um Johanna spielen zu sehen. 

Auf der Schwebefähre können Johanna und Elisabeth nicht mehr an sich halten, die Schlagermusik gruselt sie zu sehr. Sie tuscheln und kichern. 
„Was gibts da zu lachen?“ Die Tante wird böse. „Ihr hört wohl nur solche Rämmidämmimusik?“ „Nein“ sagt Johanna spitz, „ich höre am liebsten Mozart.“ Oje. Eisiges Schweigen, schlimmer, als jedes böse Wort. Johanna fällt wie in einem schlechten Traum und schlägt hart auf das schlechte Gewissen. Die Tante ist tödlich beleidigt. „Du glaubst wohl, du bist was besseres, weil du auf diese Schule da gehst!“ Quetscht sie schliesslich hervor. 

Elisabeth begleitet Johanna hinter die Bühne. Die kalten und abwertenden Blicke von Johannas Klassenkameraden lassen sie schnell in den Zuschauerraum flüchten. 

Johanna steht auf der Bühne, hinter dem Vorhang, aufgeregt ist sie. Sie muss gleich in der ersten Szene als zweite Rolle auf die Bühne und noch ein Fahrrad neben sich her schieben\dots  Es geht los. Johanna läuft los und hört im gleichen Moment ein genervtes Stöhnen von den Kameraden am Bühnenrand. Das Fahrradpedal hat sich am Vorhang verfangen und zieht ihn hinterher. 

Johanna  überkommt eine tiefe Ruhe. Kurz blitzt es in ihr auf:“ Ihr Arschlöcher, könnt ihr mich nicht mal auf der Bühne, mit eurer Missgunst verschonen\dots “ Sie nimmt den Vorhang vom Pedal, als  ob sie nie etwas anderes gemacht hätte und spielt. Leicht kommen die Worte und Gesten, fast, will es ihr scheinen, könnte sie die Freude auf der Bühne zu stehen und sich zu zeigen geniessen.

Das schlimme Erwachen kommt nach dem Spiel zu Hause. Kein Wort über Johannas Aufführung, stattdessen bitterste Vorwürfe von der Mutter, wegen der Tante. Johanna sei schrecklich und arrogant und es nicht wert, dass die Tante und der Onkel extra aus Hamburg angereist sind. 
Bei der zweiten Aufführung ist Johanna der Idiot. Thomas sitzt im Publikum\dots 

Da beide in der ersten Szene auftreten sitzt sie mit Katrin hinter der Bühne zusammen. „Ich würde am liebsten die Kostüme tauschen“, sagt Katrin. „Ich auch!“ Johannas Antwort kommt von Herzen. Sie gibt sich ganz in die Rolle, wenn ich schon vor Thomas den Idioten spiele, dann richtig\dots 
Die Proben und Aufführungen gehen weiter. Aber Johanna wird krank. Das Band im Hals hat sich eng zugezogen, sie kann kaum sprechen. Deshalb spielt ihre Besetzung ohne sie, Katrin springt ein.

Johanna soll in der anderen Besetzung die zweite auswärtige Aufführung mitmachen. Bei der Generalprobe ist sie schwach. Der Hals ist eng und der Schweiss läuft unter dem Kostüm in Bächen. Mit krächzender, stockender Stimme beginnen die Proben. Jede Rolle ist doppelt besetzt und die Mitspieler reagieren anders als gewohnt.

Szene im Zimmer des Pfarrers. Statt dem gewohnten, kameradschaftlichen Spiel mit dem Hauptdarsteller Mischa, reagiert der zweite „Andri“ Patrik aggressiv. 

„Du machst das nicht richtig, du musst da stehen.“ „Du bist zu langsam.“ „Ich kann so nicht spielen!“ Ruft er schliesslich theatralisch. Mit Wut verzerrtem, roten Gesicht funkelt Patrik  Johanna an. Johanna steht bleich und mit hängenden Schultern, die vielen Wiederholungen, haben den kargen Rest der Stimme verbraucht, es flüstert aus ihr heraus. Patrik springt auf den Lehrer los. „Ich kann so nicht spielen, mit der geht es nicht!“

Es wird still. Johanna geht von der Bühne. Die Kehle ist zu trocken, sonst würde sie weinen, sie ist froh: kein Gefühl, keine Stimme, die Schnur erwürgt jede Regung. 

Sogar einige Mitschüler empören sich und stellen sich auf Johannas Seite. „Das kannst du nicht sagen.“ Johanna fährt heim. Langsam kehrt die Stimme flüsterig zurück. Der Lehrer ruft an. „Sie müssen bitte mitmachen. So darf es doch nicht gehen. Bitte kommen Sie.“ „Nein, ich kann nicht und ich will auch nicht mehr.“ Der Lehrer schweigt.

An der letzten Aufführung nimmt Johanna nicht mehr Teil, auch nicht am Klassengeschehen. Im Unterricht rollt sie sich zusammen und hört halb träumend dem Lehrer zu. Sie saugt Wissen auf, kann die Ohren in diesem gedämpften Zustand, scharf stellen und jedes Wort speichern. 



\section*{Spiele}
\addcontentsline{toc}{section}{Spiele}


Johanna geht aus einem Grund dennoch gerne in die Schule, sie will wissen. Wie ein Schwamm saugt sie auf, was die Lehrer ihr erzählen können. Bei dem einen mehr, dem anderen weniger. Vermutlich ist sie deshalb weiterhin in der Abiturgruppe. Sie, die keinen Ton von sich gibt, macht ihre Aufgaben und liefert perfekte Haupthefte ab. Sie kann nicht anders, denn Wissen will wieder raus und sich weitertragen. Schwach ist sie nur in einem Fach, Englisch, weil, sprechen, kann sie nicht. Die Lehrer setzen ihr zu, sie muss sich mehr beteiligen, sie zuckt mit den Schultern, beteiligt sich an den Gesprächen, die erfreulicherweise nach dem Unterricht entstehen können, wenn es spannend gewesen ist im Unterricht. Sie beweist ihr Können, dann, wenn es inoffiziell ist. Sie spürt die Unsicherheit der Lehrer, sie wollen mich so gerne aussortieren und es geht nicht\dots 

Im Unterricht, längst hat sie heraus gefunden, der beste Platz, um unsichtbar zu sein, ist vorne und an der Seite, bettet sie ihren Kopf in die Arme und ruht. Dem Klang des Wissens lauschend, es breitet sich wie ein Panorama aus, bildet Bilder, nicht Worte und jeder Satz fügt ein Stück dazu. Ein guter Lehrer kann die Bilder lebendig machen, sie erzeugen einen Fluss, dem Johanna in ihren Heften an den Ursprung folgen kann, hier und da einen Seitenarm beschreibend. Sie muss nicht wie die anderen lernen, üben, sie kann spielerisch, die Bilder memorieren und diese beschreiben. Sagen, was sie sieht. Dabei spielt es keine Rolle, ob es sich um Zahlen, chemische Formeln oder Gedichte handelt, für Johanna ist Wissen und Lernen ein sinnlicher Prozess, die Sinne dazu, befinden sich innen drinnen, dort gibt es Augen, Ohren, Arme, Beine, Haut und die nehmen Wissen als Farbe wahr, als Bild  und als Musik.

Wenn es ihr langweilig wird, hat sie ein Spiel erfunden, sie verwandelt sich in eine kleine Maus. Lässt sich auf dem Tisch liegen und kriecht behände unter der Klassentür durch. Trippelt den Gang entlang und quetscht sich unter der Tür zu Thomas Klasse hindurch. Hebt schnuppernd das Näschen und klettert an Thomas Hosenbein hinauf, wo sie sich in seine Hosentasche kuschelt und friedlich, Thomas süsslichen, warmen Geruch, mit einer Spur Muff eines alten Hauses einatmend, träumt. 

Um sich in die Maus zu verwandelt, zieht sich Johanna innerlich zusammen. Die ganze Vorstellungskraft konzentriert sich auf die feine Nase, das Fellchen, das wächst, die kleinen Trippelpfoten. Erst, wenn die Geistermaus perfekt ist, darf sie los marschieren. Jeden Schritt muss die Maus machen, sie darf nicht plötzlich von hier nach dort springen, dadurch zerstäubt die Imagination, die Maus wird wieder Gedanke, Hirngespinst. Sie geht jeden Schritt und muss alle imaginären Muskeln anspannen, um unter den Türen hindurch zu kommen. Die Maus atmet den Schulmief auf dem Gang, hört die Geräusche aus den anliegenden Klassenzimmern. Sie nimmt die Dunkelheit im Gang war und das Licht und den veränderten Geruch, wenn sie durch die Halle des Treppenhauses läuft. Sie sieht das elektrische Licht in Thomas Klassenzimmer und hört das murmeln der Stimmen, ohne sie zu verstehen. Sie sucht und schnuppert an den Füssen und Beinen um Thomas zu finden. Johannas Vorstellung ist, Thomas sitzt in der Nähe der Fenster. Sie wacht fast auf, wenn sie Thomas sucht, will sich aufrichten und sehen, wo er sitzt, aber sie schafft es Maus zu bleiben und als Maus findet sie Thomas, jedes mal, nur sitzt er dann an der Tür. Seltsam, warum findet ich Thomas an der Tür, ich bin sicher, dass er am Fenster sitzt. Weil ich mir alles eh nur einbilde? Wahrscheinlich reicht meine Vorstellungskraft nur bis hinter die Tür, deshalb, aber es ist trotzdem schön, zu glauben, ganz dicht bei Thomas zu sein, in seiner Tasche, in seinem Schoss.

Was Johanna auffällt, dass die Maus geführt wird. Sie wandert an einer orangefarbenen Schnur entlang, die sie mit Thomas verbindet. Was ist das für ein Band? Johanna glaubt, es sei von ihr erfunden, ein imaginäres „Maus-zu-Thomas-Leitband“.

Erschreckt fährt sie aus der Dämmerung, wenn der Lehrer sich ihrer erinnert.  Das Kichern der Klasse darüber, dass sie aufwachen sollte, interessiert sie nicht. Die Geschwindigkeit, mit der die Maus aus der Hosentasche heraus gerissen und in sie hinein katapultiert wird, ist unangenehm. 

Viel mehr erschreckt ist sie, als sie eines Tages an Thomas geöffneter Klassenzimmertür vorbei geht und ihn genau an dem Platz sitzen sieht, an dem ihn die Maus findet und ihm in die Hosentasche klettert. Also, doch\dots ! Aber es war doch nur ein Spiel?



\section*{Galerienweg (Plön, 15.7.1988)}
\addcontentsline{toc}{section}{Galerienweg (Plön, 15.7.1988)}

\begin{verse}
In mir bilder\\
-deiner sekunden,\\
deiner taten-\\
von meinen Händen


falsche farbe\\
zerbricht
und \\
geblendet\\
von reiner\\
nackter\\
wahrheit\\
weiche ich\\
	dem akt
	

Die wahre kunst-\\
bildlos


\dots dass du\\
platz findest\\
	in mir.
\end{verse}	
	
	
\section*{ Bella Italia I( Ravenna)}
\addcontentsline{toc}{section}{ Bella Italia I( Ravenna)}	
	
	
	

Johanna schreibt:


\begin{tg}
Ravenna, Zeltplatz Ramazotti.

Wunderschön ist es hier. Riesige Felsmohlen reichen weit ins Meer und sind Vermittler zwischen dem Murmeln des Meeres und dem Menschenohr. Ein Crescendo rollt aus der stillen Weite an den willigen, weichen Sandstrand für das Menschenherz. Klagen, wehen, beten, liebevolles einbetten, zum Schlaf rauscht es wohl? Freude schöner Göttertropfen.
\end{tg}


Kunstreise mit der Schulklasse nach Italien.

Mit Johanna passieren hier in Italien seltsame Dinge. Sie bewegt sich wie im Trance und nimmt doch alles viel deutlicher wahr. Eine unbekannte, unbeschreibliche Kraft treibt sie und erzeugt ein wohlige, aber auch melancholische Sehnsucht.

In Ravenna kann sie nicht anders als jeden Raum, jedes Gebäude, das sie besichtigen zu zeichnen. In Windeseile, innerhalb von wenigen Minuten zeichnet sie, geführt wie von Geisterhand, die komplexen Strukturen von der Kirche San Vitale oder das Baptisterium der Orthodoxen. Die zeichnende Hand ist zu einem weiteren Sinnesorgan geworden, all die überschäumenden Gefühle über die Pracht zu fühlen und gleichzeitig der sich stauenden Emotionalität  ein Ventil zu schaffen.

In Florenz, das sie eine Woche besuchen, auf dem Campingplatz neben der Piazza Michelangelo, wird der Sog stärker. Dabei sind es weniger die Menschen, denen sie hier begegnen will, als der Stadt selbst, den Strassen, Gebäuden, Kirchen, Plätzen.

Jeden Tag wandern die Schüler runter in die Stadt. Der Arno, zu einem kleinen Rinnsal zusammen geschrumpft,  überzieht die Stadt mit einem fauligen, muffigen Geruch. Die Ponte Vecchio betritt Johanna mit Herzklopfen, denn einer der Strassenhändler dort hat verblüffende Ähnlichkeit mit Thomas. Ob er sie anschaut, oder in welcher Stimmung sie ihn sieht, scheint ihr ein Tor zu Thomas zu sein.

Santa Maria del fiore, die grosse Kathedrale,lässt sie erschauern, der Raum ist riesig und hallt wie eine Bahnhofshalle. Mit den Klassenkameraden zusammen macht sie sich an den Aufstieg. Über den Rundgang um den Altarraum, in das enge schneckenförmige Treppenhaus. 

Oben auf der runden Plattform, wo die anderen die Aussicht geniessen, erfasst Johanna ein starkes Schwindelgefühl, als ob sie fallen würde, gleichzeitig hat sie das überwältigende Gefühl runter zu springen zu müssen, damit dieses Schwindelgefühl schnell wieder verschwindet. Mit grosser Kraft muss sie sich losreissen von dem Bild über die Brüstung zu steigen und zu fallen. 

Sie stürzt zum Eingang ins Treppenhaus und stolpert mit wackeligen Beinen, schwindelig die Treppen hinunter, auf denen ihr von unten wieder und wieder Menschen entgegen kommen. Obwohl es schwierig ist in dem engen Gang aneinander vorbei zu kommen, ist sie froh über jedes Gesicht, das sie aus dem Wahn raus in die Gegenwart zurückholt. Verwunderte Blicke treffen sie, die keucht und stolpert und einen Schleier vor den Augen hat. Der offene Gang um den Altarraum ist die nächste Herausforderung, ihn, der offen die Höhe preisgibt, zu umrunden, ist nur mit der aller grössten Konzentration und Selbstdisziplin zu ertragen. Am liebsten möchte Johanna auf allen Vieren  dort entlang kriechen.

Unten angekommen sinkt sie neben der Treppe zu einem Häufchen Elend zusammen. Die Beine zittern unkontrolliert und der Schleier vor den Augen hebt und senkt sich. Sogar die Klassenkameradinnen bemerken es, zucken aber mit den Schultern.

Ganz anders geht es ihr mit San Miniato al Monte. Zu der Basilika auf dem Berg zieht es sie wie magisch hin. Vor allem in der Krypta fühlt sie sich wohl. Sie wandert alleine dort hin, wenn sie freie Zeit haben. Hier ist es friedlich.



\section*{Credo III (24.12.1988)}
\addcontentsline{toc}{section}{Credo III (24.12.1988)}

Johanna schreibt:


\begin{tg}
23.12.1988

Ein Tag vor Weihnachten, aber man könnte eher glauben, ein Tag vorm Weltuntergang.

Thomas, was bist du bloss geworden?

Läufst vor mir weg, als ob ich dich umbringen wollte, so ein Quatsch!
Du tust weh, Thomas!

Du tust masslos weh und ich will diesen Schmerz nicht fühlen müssen, weil du ihn auch nicht fühlst.

Aber ich kann nicht.

Du tust weh, Thomas, weil du einsam bist, aber du weisst es nicht. Und warum, in Teufels Namen, interessiert mich das dann?

Ich möchte fort, die „Geschichte“ soll endlich ein Ende haben.

Gestern hätte ich dich gerne bei der Hand genommen, süsses Kind, ich bin wieder über geschäumt, du hast es gemerkt, nicht wahr?

Ich bin auch dumm, dann gehe ich zu dir und sehe dich nicht an, weil ich doch plötzlich Angst habe.

Aber eines Tages habe ich keine Angst mehr, dann komme ich wirklich zu dir, Thomas, ganz lieb und vorsichtig.

Weisst du was, Thomas?

Gestern Abend, als du vor mir davon geflitzt bist, du bist nicht vor mir fortgelaufen, Thomas, nein, sondern vor dir. Ach, Thomas, manchmal tust du mir Leid, dann möchte ich dich nur knuddeln, bis du aufhörst vor Angst zu zittern und ganz ruhig und zufrieden bist.
\end{tg}


Es ist Heiligabend. „Wollen wir in die Kirche zur Mitternachtsmesse gehen?“ fragt die Mutter. Johanna überlegt. In „Thomas Kirche“? Sie hat Angst, sie will Thomas nicht begegnen. Der Ort gehört ihm, seiner Familie. 
Ein Gottesdienst ist eine ernsthafte Sache. Johanna ist vertraut mit Gott, es vergeht kein Tag, an dem sie nicht mit Gott redet, hadert, ihn ausfragt. An guten Tagen, stilleren, farbigen, bunten, nicht den schrillen, lärmigen, da kommt sie ihm nahe, dem Gott. Dann ist sie Einheit in der Einheit Welt und die ist All-eins in der himmlischen Macht. Es sind die Momente in denen die Zeit lauscht und den Atem anhält, die Seele, eingebettet in sich, all-ein(s) ist. Bis die Zeit wieder nach Luft schnappt, der Kopf, das rappelige Gehirn wieder zur Sprache zurückfindet und mit seinem Geplapper beginnt.

Nicht nur durch den Traum als Jüdin fühlt sich Johanna, im Gegensatz zu ihren Eltern, dem Gott näher als Christus. Der Christus ist ihr suspekt. Sie kann ihn nicht verstehen. Er hat einen dramatischen Schimmer um sich, er hat gelitten, ist gestorben, war allein unter den Menschen, unverstanden. Er ist ihr zu menschlich, Gott ist Gott, er spricht nicht, er ist da, er gibt nicht und nimmt nicht, er ist. Und, wenn alles still ist, schnurrt er\dots 

Aber, seit Johanna kopfüber in die Christengemeinschaft gerutscht ist, ist ihr Interesse geweckt. Warum nicht gehen? Eine christliche Kirche. Ein christlicher Gottesdienst um Mitternacht.

Sie fahren.

Die Kirche ist zum Bersten voll. Dicht gedrängt, winterbemantelt in die Bänke gequetscht, schläfrig vom Weihnachtsschmaus, rascheln die Menschen. Pflichtgefühl, wandert durch die Reihen, seht her, ich bin da. Hier und da, Freude, heute ist Heiliger Abend, auch sie wandert herum.
Thomas Vater, im Talar  erscheint, Johanna kann ihn kaum sehen, sie sitzen weit hinten am Rand, auch Thomas hat sie nicht gesehen. Sie ist froh, nimmt ihn auch nicht wahr im Raum, wie es passieren kann, wenn die „Mausaugen“ besser sind, als die menschlichen.

Es wird still, feierlich und dann bricht blechern und knarzend die verzerrte Lautsprecherstimme über sie herein. Was ist das? Kann der Priester die Kirche nicht mit seiner Stimme füllen? 

Herr? Was passiert hier, jedes Wort des Gottesmannes treibt mich weiter fort von der Weihnachtsstimmung. Verzweifelt sucht dich die Stimme, fleht und bittet durch das schreckliche Mikrofon, dass du, der immer und überall  ist, wo es leise sein darf, die Menschen erhören mögest in deiner Güte. Eine Stimme, gütig, freundlich, aber misstrauisch und auf der Suche, verbrennt die Andacht durch die scheppernde Verstärkung.

Johanna schluckt. 

Die Mutter ist ebenso erschrocken. Sie fahren mit einem Klos im Hals nach Hause.

Johanna kann nicht schlafen. Sie setzt sich in der heimatlichen Weihnachtsstube in den Sessel, die Beine hängen über die Lehne. Ihr ist kalt, erstarrt ist es in ihr. 

Dann sieht sie Thomas, er steht in diesem leeren, verlassenen Raum, den der Vater mit seiner blechernen Stimme beschworen hat. Inmitten dieser hilflosen,  ängstlichen und in tiefer Einsamkeit heraus schreienden Stimme, die in ihrem Misstrauen, ihrer Angst, ihre Menschenliebe bewahrt, bewahren will, steht Thomas. Allein. 

Johanna weint, erst leise, dann schluchzt es lauter und lauter aus ihr heraus. Ohne das sie es in Worte kleiden kann, begreift Johanna, die Nahrung, die innere Welt, die sie aus ihrem Inneren jeden Tag schöpfen kann, das Licht, die Farben, die Musik, das einsame Vertrauen in alles, in Gottes Schöpfung eingebettet und dort geliebt zu sein, diese Nahrung, die ihr vertraut ist wie ihr Frühstücksbrot, die gibt es dort nicht. Dort herrscht: Leere, leere Worte, leere Gesten, dort gibt es nur Aussen. Freundliches, hilfloses Aussen.

Sie möchte schreien und\dots 

Ihn halten, wiegen wie ein Baby, nie wieder los lassen,\dots 
Ihr ganzes Sein, ihr ganzes Wesen wird von Verzweiflung, dem geliebten Wesen nicht helfen zu können, überschwemmt.

„Es wird alles gut!“ Eine schöne, klare Stimme sanft wie murmelndes Wasser, spricht zu ihr. Johanna blickt auf. Vor ihr steht eine lichte Gestalt. Das Wesen hüllt sie mit seiner Stimme ein und spendet einen nie gekannten Trost. Sie muss weiter weinen, aber es ist leichter, der, der sie hält, der spricht, der weiss. Der weiss um alles Leid und um alle Liebe. „Wer bist du?“ „Ich bin Christus.“ Johanna fürchtet sich zu blinzeln, oder den Kopf zu wenden, das Wesen könnte verschwinden. Aber es bleibt. Es schwebt in einem Meter Entfernung. Obwohl es wie eine gelb, weisse Lichtsäule aussieht, ist ein Körper in einem langen Gewand, Arme und Kopf wie angedeutet.  Es steht nicht still und bewegt sich nicht. Das Licht fliesst um und in der Gestalt. Die Stimme ertönt in ihrem Kopf, füllt aber auch den Raum, es ist nicht die Wohnstube, die sie füllt, sondern einen Raum, der um den Christus entstanden ist und ihn und Johanna einhüllt, durchsichtig, zart, undurchdringlich.

Johanna sitzt still. Der Christus löst sich auf in seiner Form, jedoch den Raum zurücklassend, und mit ihm das Gefühl von Liebe und Geborgenheit. 
Johanna ist durcheinander. Die Begegnung unterscheidet sich stark von denen, die sie mit Gott hat. Gott ist überall, ein Abbild der Natur und somit allumfassend, wenn der Geist leer ist und bereit im Gleichklang innerlich und äusserlich zu schwingen. Gott schnurrt und Antworten erscheinen in Bildern, oder entstehen in der Hoffnung in den äusseren Dingen eine Antwort sehen zu können.

Der Christus ist direkt, ein Gegenüber und rührt dennoch die innersten Reserven. Eine lange Zeit braucht Johanna, dann kommt Panik in ihr auf. Wieso sie? Was soll sie davon halten? Letztlich fühlt sie sich, ihr Gewissen übernimmt den Fall, ängstlich, was erwartete der Christus, das sie tun soll? Erwartet er denn etwas? Mit Gott lässt sich leicht reden und hadern, er hört schläfrig zu, letztlich ist es die Stimme vom Gewissen, die leitet und bestimmt. Was für ein Durcheinander, was der Pastor selbst mit seinem Mikrofon nicht bewirken konnte, passiert hier mit ihr in der Stube, während sie im Sessel sitzt und über Thomas verzweifelt. Der Stolz meldet sich und schwellt die Brust, muss aber einsehen, dass er einen schweren Stand hat, zumal Johanna nur ahnt, wodurch ihr dies geschehen ist. Und, wem sollte sie davon erzählen, damit sich der Stolz richtig entfalten kann?

Eines ist jedoch sicher, dieses Erlebnis hatte sie, weil sie an ihr Gefühl, ihr tiefes Gefühl für Thomas  glaubt. Wenn selbst ein lichtes Wesen, der Christus selbst, sie tröstet, muss es einen Sinn ergeben\dots 

Johanna schreibt:

\begin{tg} 
Weihnacht:
Oh. Liebes, was bin ich froh, dass du nicht dort warst, in dieser schrecklichen Kirche! Es war alles leer und lieblos und du bist in dieser Leere aufgewachsen, ach, Thomas.

Was wundert es mich, dass du vor mir davon läufst. Wie kannst du Liebe fühlen, wenn von überall solche Leere in dich eindringt?

Thomas?

Du, ich habe Angst!

Ganz schreckliche Angst vor dir und vor meiner „Liebe“ oder diesen Gefühlen zu dir.

Ich habe Angst, Thomas, ich möchte lieber nichts von dir, sehe ich all das von dir, was ich sehen muss, wenn ich dich fühle.

Zum millionsten mal frage ich mich, ob das Gefühl Liebe ist, Thomas, ich weiss es nicht, ich wäre froh, wenn es nur Mitleid wäre, aber das glaube ich nicht, dafür bewundere ich dich zu sehr.

Ich WILL nicht, Thomas!

Ich WILL nicht!

Es sei denn, ich dürfte dich in den Arm nehmen und streicheln.

Aber, nein, erst mal muss ich eine Verfolgungsjagd hinter Herrn Thomas her veranstalten. Wo ich selber ein Angsthase bin. Wie kannst du verlangen, dass ich dir nachlaufe und stehen bleibe und wie ein Tiger kämpfe, wenn du mit deinen Ziegenbockhörnern nach mir stösst?

Du machst es dir einfach, Thomas! Du meinst, ich sollte dich in Ruhe lassen, aber du müsstest doch langsam kapiert haben, dass wir nie zur Ruhe kommen, wenn es so weiter geht.

Ach, Thomas, warum , warum bist du ein verflucht eingebildeter und stocksteifer Ziegenbock?

Thomas? Du bist wahnsinnig anstrengend, weisst du das?


\sterne


25.12.1988

Thomas, wir haben noch eine Chance uns zu entkommen:

Die, dass wir uns kennenlernen und trotz der letzten Jahre entdecken, „dass wir uns halt nur ganz sympathisch sind und sonst nichts weiter\dots “

Was hältst du davon?

Ich fände es grossartig!

Mein Gefühl zu dir ist im Moment Angst gelähmt.

Ich sehe so viel, Thomas, jetzt und damals, ich sehe Angst und kalte Wände, hinter denen wir gefangen sitzen.
 3 Jahre, Thomas, nur weggelaufen voreinander.

3 Jahre nach etwas gesucht, was uns beide zum Laufen bewegt hat.

3 Jahre haben wir nach Liebe gesucht und haben nicht gemerkt, dass wir vor der Liebe, die wir suchen, davon laufen.

Weisst du, Thomas, was das allerschlimmste war? Plötzlich, kurz vor dem Christgeburtsspiel, sah ich dein Gesicht vor mir, haargenau, dein Gesicht. Wo ich normalerweise nur weiss, dass du einen Ziegenbart hast.
Aus Quatsch habe ich mir vorgestellt dieses Gesicht zu küssen und bin dabei fast in Ohnmacht gefallen!

Glaubst du an Vorsehung, Thomas?

Ich finde es schrecklich, ich habe das Gefühl, Dinge entdeckt zu haben, die mir lieber für mein Leben verborgen hätten bleiben müssen.


\sterne


26.12.1988

Das erste mal, seit langer Zeit, dass du mich total nervst und frustrierst, Thomas.

Ich habe absolut kein Bock mehr auf dich, ich will fort und meine Ruhe haben.

Immer der gleiche Mist.

Und Thomas? Läuft weg, ist ja nicht sein Bier.

Deine Scheissangst, Thomas, die mir in die Knochen gestiegen ist, die mich lähmt und runter drückt.

Ich will sie nicht.

Ich kann sie nicht mehr ertragen.

Angst, Angst, Angst. Wo ich hinsehe bei dir Angst, oh Scheisse, Thomas!
Und ich verstehe es und lasse mich rein reissen.

Ich will aber nicht mehr.

Was habe ich mit deiner Angst zu schaffen?

Du lähmst mich, ich fühle nur Angst, DEINE Angst, Scheiss-Angst.

Und ich will sie nicht.

Und ich will sie nicht.

Ich will fühlen, Thomas, und wenn ich dich nicht fühlen kann, weil du alles Gefühl um dich lähmst, dann lass mich gehen!

Ich habe Angst vor dir und diese Scheissangst ist deine Angst und die hat bei mir nichts zu suchen. Deshalb will ich fort von dir. Du kannst Angst haben, soviel du willst und wovor du willst, aber ohne mich!

Null Bock!

Das ist alles blöde. Ich will mit dir reden, Thomas, weil ich dich vielleicht dann vergessen kann. Aber, ich werde es nicht können, weil du davon läufst.

Ich kann nicht vor dir davon laufen, Thomas, was soll ich machen?
Warum bin ich gelähmt?

WARUM KANN ICH DICH NICHT OHNE EIN WORT VERGESSEN? 

WARUM NICHT?

Und du? Warum läufst du vor mir davon?

 Warum?
 
Thomas? Wenn du mich nicht vergessen kannst, warum, in Gottes Namen, stellst  du dich der Situation nicht?
\end{tg}


\section*{Innenansicht }
\addcontentsline{toc}{section}{Innenansicht }


Jahresarbeiten wollen geschrieben werden. Johanna freut sich, denn sie trägt eigens dafür eine Idee mit sich, die unbedingt erforscht werden will. „Alles Lebende ist Bewegung“ hat der Eurythmielehrer gesagt, alles? Auch Farbe? Lebt Farbe? Das will Johanna herausfinden.

Es scheint ihr logisch für dieses Vorhaben den Eurythmielehrer als Mentor zu wählen, schliesslich ist er Experte für Bewegung. Über Farben, denkt Johanna, weiss ich selbst einiges.  

In dem Zeichenunterricht fragt der Kunstlehrer, der gerne mit Johanna plaudert und Scherzchen macht, ob sie ein Thema überlegt hätte für die Jahresarbeit. Da sprudelt es fröhlich aus ihr heraus, die Idee, die Verknüpfungen. Die Augen des Lehrers leuchten bald entzückt wie ihre, bis sie ihm treuherzig mitteilt, dass der Eurythmielehrer sie begleiten wird, wegen dem Aspekt der Bewegung. Da ist es aus mit dem Plaudern und Scherzen. Für den Rest der Schulzeit wird der Kunstlehrer Johanna ignorieren und im Jahr darauf wird eine Schülerin eine ähnliche Arbeit abgeben, die der Kunstlehrer betreuen haben wird. Die Schülerin wird sich zu Studienzwecken Johannas Arbeit ausleihen.

Johanna wird von den Abiturienten ihres Jahrgangs die einzige sein, die sich für Kunst interessiert und die einzige, die eine Zwei, statt einem Einer im Abschlusszeugnis hat. Sie wird die einzige sein, die sich fragt, was den Kunstlehrer getrieben hat, dass er ihr nach den drei Jahren, die von der Jahresarbeit bis zu ihrem Abschluss vergingen, diese Note gab. Später wird sie sich als einzige fragen, nachdem sie wegen dieser Note an einigen Kunstschulen abgelehnt wurde, ob es ihr Prinzip sein könnte, sich naiv und wissbegierig, Steine in den Weg zu legen, die ihres Gewichtes wegen, einen weiten Umweg erfordern.

Aber von all dem weiss Johanna nichts, sie stürzt sich glücklich in die Arbeit. Nach dem ersten Besuch bei ihrem Mentor, ist sie enttäuscht, sie hatte sich mehr erhofft als ein Stapel Bücher, von denen er sein Arbeitszimmer im Keller bis unter die Decke voll gestellt hat. Aber ihr Problem ist, dass sie nicht weiss, wo sie beginnen soll. Farbe und Bewegung, Farbe in Bewegung, Bewegung in der Farbe. Sie liest, macht sich Notizen, liest, findet hier ein Bröckchen und dort. Später ist Johanna froh, weil der Mentor ihr Freiraum gibt, sich nicht einmischt, hier und da Aspekte von einer anderen Seite beleuchtet, versucht Fragen zu beantworten oder Bücher zu finden, die dies tun könnten, und Fragen stellt, deren Antwort wichtig sein könnten. Johanna hat die richtige Wahl getroffen, dies wird ihre Arbeit werden, nicht die Ideen eines Lehrers.

Sie malt mit Aquarellfarben Farbe. Versucht die ungemischten Farben in Bewegung zu versetzten, sich hinein zu begeben. Sie spürt die Bewegungen, sie fühlt das Lebendige, das sich ohne die Form, ohne die Materie in einem anderen Raum bewegt. Ein innerer Raum, selbst eine einzelne Farbe auf einem grossen Blatt, ist gebunden, nur im Geist, in einem meditativen Zustand, lassen sich die Farben wirklich beobachten.

Johanna bemerkt, die Farben lieben Musik. Sie bittet Ulla, die in der Schule für ihren Klavierunterricht übt, ihr lauschen zu dürfen. Und Ulla spielt, während Johanna wie schlafend auf der Bank liegt, und übt. „Bilder einer Ausstellung“ von Mussorsky. Die Töne, Klänge wandern durch die Ohren in den Farbraum und beginnen ihn unregelmässig zu füllen, dabei wandern sie bis sie ihren Platz gefunden haben und bleiben dort, während mit neuen Akkorden neue Farben hinzu kommen. Die Farben sind nicht starr, sie halten sich aber gerne an einem bestimmten Ort auf und, bemerkt Johanna überrascht, bilden ein Perfektes Gemälde mit einem leuchtend gelben Tor von Kiew, im Hintergrund die Zwiebeltürme. Dass  das Stück so heisst, erfährt sie nach dem Hören.

Zufall? Ulla ist überrascht von Johannas Bericht, ist aber einverstanden, dass sie die Stücke wieder hören kann. Das Ergebnis ist wieder gleich. Es ist schwer, sich vorher ganz leer zu machen, das Experiment kann nur funktionieren, wenn das vorherige Bild komplett gelöscht ist. Sonst wäre es nicht richtig, eine schöne Erinnerung an das erste mal.

Es gelingt die Leere zu erzeugen. Die Farben kommen, spielen und wandern, suchen sich einen ähnlichen Platz, diesmal fügen sie spielerische Details hinzu, den Weg, der auf das Stadttor zu führt. Diesmal hört Johanna die Glocken stärker und sie bilden sich ab in den Farben.

In diesen Farbklängen bildet sich die Arbeit aus. Goethes Farbenlehre gibt den Schlüsselsatz dazu: „Die Farben sind die Taten des Lichts, die Taten und die Leiden“. Während des Schreibens gibt dieser Satz die Substanz, die Farben entstehen indem sie aus dem Licht hinein in die materielle Welt, in Greifbares gepresst werden, oder sie gebären sich aus der Dunkelheit, die ins Lichte strebt und sich von dieser Seite dem Fassbaren annähert. Welch eine Fülle von philosophischen Möglichkeiten wächst dort. Ein Schatz, den es zu bergen gilt.

Aber aus diesem Spiel ergibt sich keine Bewegung und die ist es, die alles Lebendige umgibt. Was ist Bewegung? Johanna findet eine strebende Bewegung und bemerkt, diese hat immer einen rhythmischen Aspekt. Im Gehen sind es die einzelnen Schritte, bei der Pflanze die spiralförmige Wuchsrichtung der Blätter, selbst die Planeten fahren in Schleifen um die Sonne herum und auch technische Geräte wie Autos brauchen die kreisende Bewegung um vorwärts zu kommen.

Bewegung=Entwickelung=Leben.

Wie bewegen sich die Farben? Einige kleine Hinweise haben die Meister zu bieten, Goethe und der König der Farben: Kandinsky. Die homöopathische Farbheilmethode des Heilpraktikers gibt zusätzliche Hinweise, wie die Farben den menschlichen Körper bewegen. Aber Johanna ist nicht einverstanden mit Kandinsky und Goethe will die Wirkung untersuchen, nicht die Bewegung, die sich aus den Farben selbst ergibt und für Johanna ist dies nicht dasselbe. 

Klang hilft, Barockmusik, Vivaldi. Mozarts Requiem, das Johanna liebt, zerstäubt die Gedanken, will sich in die Seele, die Empfindungen bohren. 
Der Abgabetermin rückt näher, aber Johanna hat einzig Notizen, im besten Fall und viele Gedanken.

Unruhig wird sie, sie kann nicht ewig in ihrem Sessel hocken und die Farben tanzen lassen. Weiter, wieder Vivaldi, Corelli, klare Strukturen, saubere Töne, Ordnung in sich wiederholenden Spiralen, die tiefer und tiefer, höher und höher führen, die Gedanken, können sich daran festhalten, während sich die Schädeldecke dem allwissenden Raum öffnet, der über dem Kopf schwebt und angezapft werden kann, sobald der Kreisel der Gedanken im Kopf zur Ruhe kommt, anderweitig beschäftigt wird. An den Gedanken, da muss man sich vorbei schleichen, der Kopf ist klein und die Gehirnwindungen führen schliesslich in Kreis herum, von dort kann nichts Neues kommen. Wer weiss das besser als ich, denkt Johanna, ich, die Meisterin der sich im Kreis drehenden Fragen und Antworten.

Eine Woche vor dem Termin ist es soweit, die Schleusen öffnen sich, die Farben ergiessen sich über das Blatt, ordnen sich, bilden Strukturen, die verknüpft sind mit dem Menschenwesen, die sich um die Erde legen und bis ins Weltall hinaus strahlen. Sie zeigen ihre Bewegung, ihren Sinn, ihre Taten und Leiden.

Johanna fühlt sich wie eine Glucke, die ihr erstes Ei gelegt hat, ist glücklich und stolz, sie hat die Farben in einem eigenen, neuen Licht gefunden, hat sich weder von Goethe, noch von Kandinsky einschüchtern lassen. Gut, das Ergebnis ist anthroposophischer ausgefallen, als sie zu Beginn jemals geahnt hätte, aber es bleibt ihr Baby und ausser Goethe hat sie keine „Anthroschrift“ benutzt.

Der Vortragstag ist da. Johanna ist aufgeregt und sie ist als Vorletzte dran, abends, und sitzt den ganzen Tag zwischen den aufgeregten Kameraden, die, einer nach dem anderen, erlöst werden.

Thomas ist nicht da. Eine Enttäuschung. Aber Frieder, sein bester Freund.
Bald ist es soweit. Christian ist als einziger noch vor ihr dran. Und der Christian ist ein stiller, der wird nicht viel sagen. Stille Wasser sind tief, Johanna kann es nicht fassen, der gute Christian erzählt von jedem Schräubchen, das er in sein Teleskop eingebaut hat, zärtlich erklärt er ihren Sitz, wie fest sie angezogen sind, was sie zusammen halten und wie schwer es war, sie an ihrem Platz unter zu bringen. Eine Teleskop hat viele Dinge, aus denen es zusammen gebaut ist, viele erwähnenswerte Dinge. Nach einer Stunde zappelt das Publikum nervös auf den Stühlen, Johanna ist erschöpft, jetzt hat sie eine Stunde lang Herzklopfen gehabt und die Aufregung weicht einer bleiernen Müdigkeit. 

Christian ist fertig. Johanna geht nach vorne, hängt ihre Bilder auf. Wie beim Klassenspiel wird sie dabei ganz ruhig. Sie dreht sich zum Publikum und wird traurig. Traurig, weil Thomas nicht da ist und traurig, weil sie keine Lust mehr hat ihren Vortrag zu halten. Zuhause, da hat sie geübt, sich vorgestellt wie schön es wird, ihre Arbeit in Worte zu kleiden. Jetzt steht sie da, erschöpft vom Warten, vor einem Publikum mit erschöpften Ohren.

Sie redet und es ist einfach, die Worte kommen von allein, kleine Scherze. Das Publikum kann sich über ihre saloppe Art ihre Entdeckung dar zu legen, das Lachen nicht verkneifen und wird zusehend wieder munter. Was bin ich für ein trauriger Clown. Dabei fällt ihr Blick auf Frieder, wirst du Thomas berichten? Ja, sie sieht es an seinem Blick, er saugt auf, wie sie spricht, wie sie sich verhält, er ist Kundschafter. Feigling, hättest ruhig selber kommen können. Frieder mit seinen blonden Locken und dem lachenden Gesicht, der scheint wie Thomas guter Zwilling, er wird ein guter Bote sein. Er ist freundlich zu Johanna, er scheint mehr zu wissen und zu verstehen, als Thomas, der auf seiner wilden Flucht ist.

Als der letzte Vortrag vorbei ist, erlebt Johanna eine Überraschung, Boris kommt zu ihr und bedankt sich, mit feuchten Augen. „Toll, deine Arbeit, Wahnsinn, du hast was ganz Eigenes gemacht. So genial. Du, und Ulla, ihr seit die einzigen, die etwas Eigenes gemacht haben.“ Die stille Lisa aus Thomas Klasse, die die schönsten Bilder zeichnet, die Johanna je gesehen hat, kommt zu ihr und bittet sie schüchtern die Arbeit ausleihen zu dürfen, sie wäre tief beeindruckt von Johannas Idee. Es gibt sie, die Menschen, die sehen können. Die, die spüren, wenn etwas besonderes den Raum betritt. Warum sind es aber die stillen, die das bemerken? Wie können sich die Dinge, die aussergewöhnlich sind, einen Weg in die Welt bahnen, wenn nur die stillen sie  sehen?

Was bleibt ist eine Sucht. Die Sucht sich leer zu machen und den Antworten zu lauschen, die sich oberhalb der Gehirnwindungen befinden. Die Sucht den Raum zu betreten, wo alle Antworten enthalten sind, der alles enthält und, wenn sich ein Fokus einstellt, aus der Weit sich hin spiralt auf diesen einen Punkt. 
Ich bin Farbe, ich bin \dots Farbe, leise beginnt es zu vibrieren, Ideen, in Bilder abgebildete Ideen kommen und gehen, zerspringen wie Seifenblasen, weiter, ich bin\dots  solange, bis alles sich in die Einheit auflöst und eine Landschaft sichtbar wird, solange, bis reine Struktur sich knisternd kristallisiert, mit scharfen Kannten, Ordnung sich einstellt, die Landschaft lebendig wird in Musik, im Klang, frei von Materie, sich der Geist zeigt. 



\section*{Karmaloka II (11.7.1989)}
\addcontentsline{toc}{section}{Karmaloka II (11.7.1989)}



Es gibt eine zweite Sucht. Den Weg über das Herz. Einsteigen in die Flatterkammern, einsteigen in das Gefühl, dem Strick folgen. Die Empfindungen beschreiben, mit Wörtern versehen, benennen, bis sie den richtigen Namen tragen. Johanna schreibt. Keine Träumereien, Worte, keine sanftmütigen; Worte, die bohren, die Ant-worten fordern, Worte, die Herz und Gefühl sprengen wollen, einkreisen, die Decken heben, Kellertüren öffnen und Leichen suchen, finden und flettern. Warum? Warum?

Längst ist klar, was sie empfindet, abgesehen von den Gefühlskapriolen, ist nicht Liebe, nicht allein. Aber dieses winzige, was übrig bleibt, wenn die Verliebtheit zerrupft ist auf dem Papier, es bleibt im Dunkel, lauert dort und zieht sich zurück, sobald Johanna es anschauen will. Es ist nicht zu fassen, eine Erscheinung, die sich im Augenwinkel spiegelt und im vollen Blick unsichtbar wird. Dennoch verfolgt sie dieses Winzige, springt hervor, wenn Johanna nicht daran denkt und wenn Thomas unvorbereitet ist. Es macht krank, vergiftet das Gemüt. Es schreit, ich bin da und lacht wahnsinnig. Und Johanna ist sich sicher, sie spürt es nicht allein, Thomas muss es auch wahrnehmen, allerdings muss ihn dieses Winzige zu Tode erschrecken.

Johanna will es wissen, was ist das? Zumal sie, je weniger sie sich sicher ist, dass sie Thomas liebt, dennoch das Gefühl einer Verbindung nicht verlieren kann. Wie kann es sein, dass sie ihn sieht, bemerkt, wie er mehr und mehr die Zartheit, sein liebevolles, naives Sein verliert, sich vor ihm körperlich und im Herzen ekelt und sich trotzdem, wenn dieses Winzige hervorgesprungen ist, wieder in lodernden, unangenehm peinlichen Liebeswahn steigert? Es ist nichts schönes mehr daran, wie von Aussen beobachtet sie mal diese, dann diese Regung, schreibend kämpft sie um Worte, die sich im Kreis drehen, weil sich der Winzling versteckt hat und ganz leise, verstohlen in seiner dunklen Ecke irre kichert.

Ich werde verrückt. 

Stille hilft, oder dem Liebegefühl, wenn es da ist, nachgeben und auf ihm fliegend die Schönheit der Welt betrachten, den Himmel, der dann seine Geschichte offenbart, die Vögel, die ihr frisches Konzert pfeifen, das satte, volle Grün an grauen Tagen, aus der nicht vorhandenen, dennoch empfundenen Zweiheit, eine Vielheit, eine bunte Welt machen.

Ich will nicht abhängig sein, von diesem Mensch, der mich flieht und greift, ohne, dass er es merkt. Das orangefarbene Band, es ist kein Mausband, es ist ein Strick. Es lässt sich nicht entfernen, selbst wenn Johanna alle Tricks versucht, es sich aus zu reissen. 

Johanna verliert die Geduld, was ihr nichts hilft, weil es die Gefühle nicht ändert. Ich bin in Thomas gefangen, eingesperrt in einen Bereich, den er in sich verschlossen hat, den er selbst nicht wahrnimmt, nicht spüren will, aber aus dem heraus mich dieses Winzige anfällt und wieder in ihn hineinzieht. Und der Idiot bekommt genug davon mit, schliesslich läuft er vor mir davon und merkt nicht, dass er mich wie an einem Gummiband hinter herzieht. Und ich spicke mal an ihn dran, oder mache an dem gedehnten Band einen Flug durch den Orbit. Und dieses Winzige, es kichert in seiner dunklen Ecke und erschreckt uns beide und das Spielchen beginnt\dots 

Ich weiss einzig einen Weg, der das Spektakel beenden kann, Johanna ist sich sicher, dieser Winzling muss raus gezerrt werden aus seinem Versteck, er muss sichtbar werden, in Worten. Ich muss mit Thomas reden. Reden, selbst, wenn ich über dieses Ding selbst nicht sprechen kann, ich muss Thomas gegenüber sein, ihn um Hilfe bitten, damit wir beide Ruhe haben voreinander.

Teeschuppen, Punkkonzert. Johanna lauert auf eine günstige Gelegenheit Thomas allein zu erwischen und hat Glück. Er geht raus, allein. Sie läuft hinterher. „He, warte mal.“ Er dreht sich um und versucht sie mit seinem Blick zu erschlagen, was nicht funktioniert, weil Johanna in seine Augen kippt und fasziniert staunt. „Ich muss mit dir sprechen.“ „Hatten wir das nicht schon mal?“ tönt es arrogant unter den saugenden braunen Augen hervor. Johanna spürt tief unten, wie sich ihr Stolz zur vollen Grösse aufrichtet und die Faust ballt. „Ich muss, können wir nicht kurz reden?“ „Äh, jetzt ist ganz schlecht, ich fahr gleich nach Kiel, zur Kieler Woche.“ „Wann denn sonst?“ „Tja, man sieht sich ja, in der Stadt und so\dots “ Johanna lässt ihn stehen. Geht durch den Park ins Cafe „Zeit“. Sie braucht erst mal Ruhe, erholt sich von den Augen, bemüht sich, unter der Wasserglocke zum Vorschein zu kommen.

Es ist spät, als sie durch den Park nach Hause fährt, kein Mensch ist mehr beim Teeschuppen, die Lichter aus, dort, an dem dicken Baum gelehnt sitzt noch einer. Das\dots , das ist Thomas!  Soll ich anhalten? Da springt der Stolz vor und haut ihr kräftig eine Ohrfeige: „Von wegen nach Kiel fahren\dots “ heult er: „\dots  zermalmen, zertrampeln, zerfetzen, zerreissen\dots , soll er in der Hölle schmoren\dots  tausend und tausend Tode sterben\dots “

Die nächsten Tage machen es nicht besser. „Man sieht sich in der Stadt\dots “, was denkt sich dieser Vollidiot, dass Johanna ihm nach rennt, ihn um ein Treffen anbetteln, wenn er ihr begegnet? „Hatten wir das nicht schon mal?“\dots  Schäm` dich, du Arschloch! Du, echtes, dreimal ausgekotztes Arschloch, was glaubst du, es macht mir Freude eure „Hoheit“, eure „Erhabenheit“, eure „Arroganzheit“ um eine aus Gnade gewährte „Audienz“ zu bitten? Kriegst du von dem ganzen Scheiss wirklich nichts mit? Warum rennst du dann wie der Teufel, wenn du mich siehst? Oder, was viel unerträglicher ist, schaust mich an und lässt den Winzling auf mich los, damit es wieder nicht zu Ende geht, damit die verdammte Geschichte kein Ende findet? Johanna könnte vor Wut und Verzweiflung platzen.

Fete am nächsten Wochenende. Johanna weiss, Thomas wird da sein und ihr ist schlecht. Aber, sie geht, wer A sagt, muss B schlucken. Es gibt nichts zu schlucken. Thomas rennt. Johanna ignoriert ihn, was sie sich sparen kann, weil Thomas es nicht bemerkt.

Es reicht! Ich muss einen anderen Weg finden, den Winzling ans Licht zu zerren, bevor meine Jugend verschwendet ist. Was bleibt? Ein Brief. Ein saftiger Brief, kein netter, ergebener Liebesbrief. Johanna scheisst Thomas vielmehr zusammen, was er sich für Scherze mit ihr erlaube, scheinbar erwarte, dass sie ihm nachläuft. „Der Fuchs ist im Haus\dots “ schreibt sie. Wie soll sie ausdrücken, was sie seit drei Jahren nicht in Worten packen kann, um es endgültig zu machen. Kein Liebesbrief, nein, nichts von dem, Verzweiflung? Vielleicht. Verletztes Stolz und Wut.
Zwei Wochen später kommt Johanna am Mittag aus der Schule und die Mutter gibt ihr einen Brief.

 Thomas!

Johanna schreibt:


\begin{tg}
11.7.1989

Thomas hat geantwortet. Ich bin ein böser Dämon. Erbarmungslos habe ich dich zerflettert und zerfetzt. Wie ein Kind, das einem Insekt aus purer Neugierde ein Beinchen nach dem anderen auszieht. Es tut mir weh!

Es zehrt an mir, habe ich einen Menschen, ohne die Erlaubnis, ohne Liebe in seinen Grundfesten zerwühlt.

Ohne Liebe, ich verstehe, warum alle anderen mir nicht nahe kommen mögen. Wer nähert sich freiwillig einer hinterhältigen, bösen Gestalt? Ich bin zu tiefst verschämt, fühle den Schmerz, den ich verbreitet habe. 

Thomas hat keine Lust mir sein „Inneres zu offenbaren“, nein, natürlich nicht. Hat er ja schon gesagt, das Problem ist nur, dass ich ohne Vorwarnung mich in ihm breit gemacht habe, mir, ohne seine Erlaubnis, sein „Inneres“ selbst offenbart habe.

Das Problem ist, mir gefällt es bei dir, Thomas! Vielleicht „liebe“ ich dich auch. Nur, ist es kein grösserer „Liebesbeweis“, wenn ich versuche mich ehrlich aus dir heraus zu ziehen?

Du bist mir wie ein zuhause, ein zuhause, dass ich gerne habe, bei dem ich wohnen bleiben möchte, ob es nun Liebe ist\dots , zumal ich mich wie ein Dieb in „meinem“ Zuhause verhalten und schleichen muss.

Was dich anbelangt, bist du ein sehr gewissenhafter und ordentlicher Mensch und auch, wenn du nicht erkannt hast, dass „der Fuchs schon in dir haust“, so hast du ihn doch genau am Gewissen gepackt und durch gewackelt! (Wieder ein Grund, weshalb ich dich liebe, verehre!)

Ich bin ein Dieb, ein kleiner Dieb, der sich in einen fremden Garten geschlichen hat und diesen in seiner Schönheit so lieb gewonnen hat, dass er lieber sein Leben her gäbe, als diesen Garten wieder verlassen zu müssen. Es wäre so schön, wenn es, wie du sagst, nur „Phantasievorstellungen“ wären. Ich möchte dich verlassen, aber du bist so schön, Thomas. So schön, dass ich lieber ein verachtenswerter, kleiner Dieb bleiben möchte. Du müsstest mich töten, Thomas, damit ich deinen Garten verlasse.

Oh, Herr!

Du hast ja so recht, Thomas, dass du mir den Kopf gewaschen hast. 
 Ich bin wirklich ein ungezähmter Fuchs geworden, so wild und unberechenbar, selbst zu mir. Aber ist nicht genau das meine Kraft? Meine Eigenschaft, die mir einen unermesslichen Weg gewährt?
 
Ich darf keinen Menschen mit meinen Gedanken, die ich speziell an ihn richte, die durch einfaches Fühlen begründet sind, wie du sagst, „bombardieren“. Aber den unermesslichen  Reichtum, den ich bei dir finde, darf ich auch den nicht verschenken?

Ich bin wild und ungezähmt, aber daraus kann ich dich fühlen, wie kein anderer Mensch es je könnte. Wie bei allen „Wilden“ habe auch ich Kräfte, die „gezähmte“ Menschen, Geschöpfe nicht haben und wohl kaum begreifen können, begreifen wollen, weil es sie von ihrem Weg bringen könnte!

Ich weiss nicht, ob du begreifst, was mein Zutrauen, Vertrauen in dich bedeutet. Es ist, als ob ich hoffte, dass du mich „zähmtest“, Thomas. Frage mich nicht, warum du es sein sollst, wahrscheinlich, weil ich spüre, dass du gut bist. Ich kann niemand anderen mit dieser Aufgabe betrauen, weil ich nur in dich dieses Vertrauen hege. Und so werde ich, wild und unberechenbar, immer wieder deine Nähe suchen.

Ein Dieb, ja, der bin ich auch.

Denn auch Diebe sind unberechenbar, die Liebe zu dir macht mich zu einem guten Dieb. Wahrscheinlich könntest auch du mich nicht zähmen. Aber durch dich könnte sich vielleicht meine gute Seite besser entfalten\dots 

Ich bin auch jetzt zufrieden und glücklich, Thomas, weil ich dich fühle und liebe, spüre. Ich kenne dich nicht, sagst du, aber wenn ich dich so spüre? Weiss ich dich?

Du hast recht, Thomas, nur vergisst du, dass ich dich liebe und die Liebe jede Schranke aufhebt, wenn sie wirklich und ehrlich ist. Jede, Thomas. Auch die, eines Diebes.

Ich bin zufrieden, weil ich dich spüre, Thomas und ganz sicher bin, dass du das fühlen kannst.

Was in meinen Büchern steht, ist sicher auch „Phantasievorstellung“, wie du sagst, aber sie hat ihren Ursprung in wortloser Liebe. Es liegt am Wesen der Vorstellung, weil sie in Worte gekleidet ist, dass sie Unsagbares“ wörtlich verzerrt. Wenn sie sich aber auf Liebe begründet, findet sie immer zu dieser zurück und jedes mal, wenn die wortlose Liebe hervor glüht, bin ich dir einen Schritt näher gekommen. So sind es nicht die Vorstellungen, die mir die Liebe geben, aber ihre Überwindung lässt alles auflodern. Du hast recht, Thomas, aber, was tut ein reger Mensch, wenn ihm die Liebe versperrt wird?

Wer macht sich in drei Hungerjahren nicht jedenfalls in der Phantasie einen „saftigen Braten“?

Du bist kein „saftiger Braten“, den ich aus der Vorstellung lieb gewonnen habe, Thomas, sondern Blatt für Blatt sind meine Vorstellungen von dir abgefallen, je weiter mein Herz nach dir wurde.
\end{tg}


Johanna ist auf sich allein gestellt mit dem Winzling und den Ungereimtheiten, die durch ihn entstehen. Thomas hat sich deutlich ausgedrückt, er verbittet sich jegliche „aufdringlichen Gespräche und obskuren Briefe“! 

Gespräche? Warum, fragt sich Johanna, benützt er die Mehrzahl? Und, es war kein Gespräch, es war eine Bitte, um ein Gespräch. Drei Jahre sind seit dem Telefonat vergangen, in denen sie kein Wort gewechselt haben. 

Eine. Eine Bitte. Eine Bitte, um ein Gespräch. Einer. Ein Brief. Nach drei Jahren. Nach drei Jahren alleine, schweigend kämpfen.

Johanna schaut im Duden nach, hat Thomas das auch gemacht? Obskur =  „dunkel, unbekannt, verdächtig, [von] zweifelhafter Herkunft“. Das Wort gefällt Johanna, obwohl sie weiss, wie es gemeint ist.

Der Brief stört sie zu nehmend, je mehr sie, aus der ersten Euphorie, eine Antwort erhalten zu haben, wieder auf den Boden kommt. Er will etwas Unwahres erzwingen. 

Er spricht aus der Anderswelt, der Welt, die sich vor dem Inneren befindet, die der Maske, die, die völlig uninteressant, weil oberflächlich und durchschaubar ist. Er ist das einzige, was Johanna von Thomas in den Händen hält, das einzige, was er zu geben bereit ist. Der Idiot. Der Brief hilft nicht und bringt auch den Winzling, der wie rasend meckerndes Gelächter aus stösst, nicht zum Schweigen.

Johanna nimmt den Brief mit in den Park. Sie setzt sich an das Ufer des Stadtsees. Milchig, wässrig ist der Himmel. Gräulich weiss verschwimmen die Konturen. Das Wasser dümmpelt, klitsche, klatsche träge an die hölzerne Umrandung. Möwen und Enten schwimmen weiter draussen. Still ist es. Sie nimmt das weisse Blatt mit der tintenblauen Schrift, die klein und kritzlig ist, um sich steil in übergrosse Höhen und Tiefen zu schwingen, aus dem Umschlag. Sie liest ein letztes mal, nein, dies berührt nicht das Wesentliche, im Gegenteil, es füttert die Lüge, die Lüge von der hässlichen Johanna, die hoffnungslos in den coolen Thomas verknallt ist. Langsam faltet sie ein Papierschiffchen. 

„Hallo, was machst du da?“ Johanna schreckt zusammen. „Oh, hallo, Jens! Ich schicke einen Brief ab, den ich nicht haben will.“ „Ja. Manchmal gibt es solche Briefe.“ Jens setzt sich neben sie, ganz leise, er hebt eine weisse Möwenfeder auf. Sanft nimmt er Johanna das Schiffchen aus der Hand und schmückt es mit der Feder. Er gibt Johanna das Schiffchen zurück und sie lässt es ruhig und zärtlich ins Wasser gleiten. Sie sitzen lange dort, beobachten, schweigend, wie das Schiffchen hin und her schaukelt, von unten der dunkle Wasserrand sich herauf zieht, die Tinte sich von dem Papier zu lösen beginnt, selbst Wellen malend, eingebettet in den Dunst über dem Wasser.

Als sich das Wasser die Schrift zurück erobert hat, gehen sie ins Café „Zeit“ und trinken einen Tee. Johanna ist unendlich dankbar, weil Jens nicht fragt und weil sie im Café lachen darf und so tun, als wäre alles gut. Es gibt sie, die perfekten Momente, die aus der Stille geboren sind, die der ganze Leib fühlt, weil man nicht alleine ist, weil gemeinsames, doppelt fühlen kann. Und es gibt Menschen, die es ertragen. Warum sind es die stillen Menschen und warum kann Thomas, der doch mal ein stiller, sanfter Geist war, es nicht mehr? Will es nicht mehr? Wovor hat Thomas Angst, wenn diese Stille einsetzt? Als Johanna durch den Park nach Hause fährt, hält sie an und geht ans Ufer. Das Schiffchen ist verschwunden. Den Briefumschlag nimmt sie mit nach Hause. Nach vielen Jahren, trennt sie die Briefmarke ab und wirft den Briefumschlag fort, nach mehr als zwei Jahrzehnten, trennt sie sich von der Briefmarke.

Nach vielen Jahren wird Johanna von einer Freundin folgendes erfahren: Thomas habe, als er den Brief bekam, einen Freund, den Johanna selbst flüchtig kennt, zur Hilfe geholt und sie hätten eine Woche gegrübelt, was der Brief bedeute und was Thomas antworten solle. Einerseits geht es Johanna runter wie Öl, weil Thomas sich Gedanken über sie gemacht hat, andererseits, der gute alte Stolz kommt wieder zum Vorschein, wie ums alles in der Welt, konnte er nach einer Woche grübeln, so eine Antwort schicken? Und – warum hat er Johanna mit „Phantasievorstellungn bombardiert“, statt mit ihr zu sprechen? Wovor fürchtete er sich? Eine Liebeserklärung zu hören? Oder, kein Liebesgeständnis zu erhaschen? Da, da ist er wieder, der Winzling, hahaha\dots 



\section*{Prinzessin II}
\addcontentsline{toc}{section}{Prinzessin II}




Während die Innenansicht Johanna eine vertraute Welt ist, sie im Gegenteil wieder und wieder staunt, wie wenig Zugang zu dieser die Menschen um sie herum zu haben scheinen, um so mehr rückt ein anderes Mysterium in den Mittelpunkt, das scheinbar ausser ihr, keinem Schwierigkeiten macht: Frau und Mann sein.

Ihre Kameradinnen haben damit keine Mühe. Wenn ein männliches Wesen in der Nähe ist entfalten sie eine Palette verschiedenster Tricks und Verhaltensweisen, um die Aufmerksamkeit auf sich zu lenken. Manche sagen nichts, weil sie nicht auffallen wollen, aber, dann sind sie wie erstarrt und kichern und schwatzen erst wieder, wenn sich der Junge entfernt hat.
Für Johanna ist dies eine Rätsel mit sieben Siegeln. Dabei ist es nicht, dass sie kein Interesse hätte, natürlich, allein wegen Thomas, findet sie all das aufregend, aber es erschliesst sich ihr nicht. Selbst bei Elisabeth, ihrer Vertrauten und Cousine, bemerkt sie diese Veränderungen. Wie kann ich mit einem Mann zusammen sein, als liebender Mensch zusammen sein, wenn ich mich als Frau in meinem Verhalten gleichsam wie verstelle, um ihn um den Finger zu wickeln? Wenn ich ihm hingebungsvoll lausche, wenn er von seinen Taten prahlt, und zu seinen Witzen bedingungslos kichere, selbst wenn sie dumm sind, wie kann er  mich dann als Person erkennen? Schliesslich wäre ich nicht das einzige Mädchen, das sich so verhält, er wird, wenn er ein begehrter Typ ist, also wählen können und das allein nach dem äusseren Erscheinungsbild. 

Ein erschreckendes Bild für jemanden wie Johanna, die sich denkend durch die Welt träumt. Was sie quält ist, dass sie selbst beginnt, die Menschen um sich herum nach ihrem Äusseren zu betrachten. Es gibt wenige, die ihr schön erscheinen und selbst die müssen auf das Klo und kacken. Im Gegensatz zu den Tieren, die zwar auch kacken, aber in sich perfekt und vollständig gebaut sind, sind Menschen seltsame, wenn nicht hässliche Wesen, findet Johanna.  

Sie schliesst weder sich, noch Thomas von der Betrachtungsweise aus. Zumal sie, seit sie die erste Regelblutung bekommen hat,  mehr und mehr unter Akne leidet und, wenn sie in den Spiegel starrt, sich selbst als eine Anhäufung roter und gelber, eitriger Pusteln wahr nimmt. Gelingt es ihr, lange genug das Spiegelbild zu ertragen, kann sie ihr Gesicht erkennen, fragt sich aber sofort, ob die anderen ihr Gesicht auch sehen können.

Durch Elisabeth lernte sie, ausserhalb von der Anderswelt, einige Mädchen kennen und gehört bald dazu. Sie bewegt sich am Rand, aber Jasmine hat sie wirklich gern. Eine ungleiche Freundschaft, Jasmine ist gross, hat ein schönes, orientalisch geschnittenes Gesicht und schwarze Haare, einen grossen Busen und überschäumendes Temperament. Sie lacht, sie weint, sie schreit, sie macht all das laut und emotional. Johanna staunt über die Dinge, die Jasmines Mund über Gefühle, Regelblutung, Jungen und das Leben verkündet. Jedes Gefühl, jede Freude, jeder Schmerz wird gepackt und bis zum Ende durchlebt und mit Worten in die Welt befördert. Sofort, dann, wenn es passiert, unreflektiert und wild. Wie macht sie das?

Mit der Ablösung von den Klassenkameraden, beginnt Johanna einen eigenen, speziellen Kleidungsstil zu entwickeln. Sie kauft sich Wanderstrümpfe, weil es die später beliebten Overkneestrümpfe noch nicht gibt, die sich über die Knie ziehen lassen. Sie reist begeistert mit den Mädels aus dem Café Zeit nach Hamburg, wo sie die Secondhandshops stürmen. Besonders die Satinoberteile der alten Herrenpyjamas haben es ihr angetan, es gibt sie, sanft glänzend in allen Farben. 

Bunt darf es sein. Die Haare werden in stundenlangen Prozeduren Schichtweise mit verschiedenen Hennatönen gefärbt. Aber schminken, das mag sie nicht. Die Mädels bestürmen sie, sie sollte mehr aus sich machen. Sie geht in den laden für Pflanzenkosmetik und verbringt eine entsetzte viertel Stunde, in der ihr die freundliche Ladenbesitzerin eine Schicht Make Up nach der anderen auf der Nase verteilt, um ihr die verschiedenen Produkte und ihre Zusammengehörigkeit zu demonstrieren. Stark Abdeckendes nach unten, zarter Abdeckendes oben drüber, um das stark Abdeckende abzudecken und zum Schluss Puder für den matten Schimmer. Und das ist nur der Anfang, denn bei einer dicken Schicht Make Up, bräuchte es auf den Wangen auch etwas Rouge, um die frische Gesichtsfarbe wieder zurück zu täuschen. 

Johanna ist benommen von der Schichtarbeit. Erstmal einen Tee trinken\dots  Im Café "Zeit" sind alle begeistert, da sieh nur wie schön es aussehen könnte. Johanna spürt eine zentnerschwere Last auf ihrer Nase und erträgt kaum den muffigen, öligen, mit Duftölen zusätzlich bereicherten Geruch.

Zuhause schaut sie in den Spiegel und erschrickt, die Nase ragt gleichmässig braun-beige betoniert aus dem Gesicht, das hell\-häutig, blass unter der Aknerötung der Pickel vor schimmert\dots{} Mein Gott, so bin ich in der Stadt herumgelaufen\dots 

Ausserdem wissen alle, dass ich Pickel habe, wenn ich die plötzlich unter Schichten von make Up vergrabe, dann kann ich mir gleich Hinweisschilder ins Gesicht stecken. Puder ist da anders, damit kann sie sich anfreunden, ausserdem geht es schnell\dots 



Credo IV (Rendsburg 1990)



Sabine hat über den Teufel erzählt. Einige Mädchen sitzen bei Jasmine im Zimmer und Sabine erzählt, sie liebt den Teufel. Er lässt sie Menschen verfluchen und gibt ihr alles, was sie sich wünscht. Sie wollte ihm ihre Seele verkaufen, vertraglich, mit Blut unterschrieben, aber sie hat es nicht gemacht. Die anderen lauschen und, obwohl Jasmine das Fenster geschlossen hat, zittern alle vor Kälte. Sie kriecht von Innen heraus an die Oberfläche der Haut. Sabines Gesicht leuchtet, sie verrät, wer noch des Teufels ist und über Zaubermächte verfügt, wer Flüche ausgesprochen hat. Die übrigen Mädchen schweigen, denn sie kennen die Namen. Haben sie denjenigen etwas gesagt, getan,  was sie bereuen werden? Einige schlottern vor Angst. Das Zimmer ist angefüllt mit einer starken Energie, die jeder verheissungsvolles ins Ohr säuselt und sich das Maul leckt.

Einigen wird es zu viel. Alle fahren mit den Rädern in die Stadt, Abenteuer muss her. Im Park rauchen sie einen Joint. Zu viel des Ernstes bedarf der Ablenkung, es wird gelacht. Gelacht über Jens, der vom Kiffen immer kotzen muss und sauer meint, ihm wäre nicht übel.

Der Teufel hat jede berührt, ihnen mit seinen scharfen Fingernägeln über das Rückgrat gestrichen. Dabei glauben sie nicht an so`n Zeug. Sie sind jung, wollen erleben, wo die Welt anfängt und wo sie endet.

Johanna fährt nach Hause, der Tag ist grau und das Grün leuchtet grell. Johanna weiss, sie braucht die Hand ein klein wenig ausstrecken, sie kennt den Teufel und sie kennt seine Kraft.Woher, sie weiss nicht woher, nur, dass es ihr leichter fallen würde als allen anderen, diese Kraft zu nutzen. Es ist ein kleiner Schritt, böse zu sein, die Angst, die Wut heraus zu schleudern. Die Macht kosten und ein Feld der Verwüstung zu hinterlassen, das Wehklagen geniessen. Sie zittert, ein Kind, das überlegt verbotenes Gebiet zu betreten, vor sich einen Zauberstab liegen sieht und überlegt, ob es ihn aufheben will.

Johanna legt sich ins Bett. Sie dämmert weg, die Gedanken auf eine Entscheidung fokussiert, welchem Herrn sie dienen will. Auch wenn sie sich diese Frage nicht vorher gestellt hatte, ist sie sich ihrer spirituellen Fähigkeiten bewusst.

Sie hört die Gedanken der anderen wie ein Radio ohne Knopf, ein lebendes Radar Menschlichens, warum sollte sie das nicht nutzen?

Sie erwacht aus cannabischer Dämmerung. Auf der grün schillernden Bettdecke sieht sie viele Gestalten. Sie wird ins Bild gesogen und steht am Rand eines Kampfplatzes. Schwarze Gestalten mit langen Stöcken bewehrt, stehen hell gleissenden, weissen gegenüber. In zweier Scharmützeln sind sie im Kampf entbrannt. Fasziniert beobachtet Johanna wie Engel mit Teufeln kämpfen. Beide Seiten scheinen gleich stark zu sein, wo eine lichte Gestalt vor einer schwarzen zurückweicht, da bedrängt ein Stück weiter eine lichte die dunkle. Auf Johannas Bettdecke. Sie schliesst die Augen, was passiert hier? 

Als sie die Augen öffnet in der Hoffnung, dass der Spuk verschwunden ist, tobt der Kampf weiterhin, die lichten Engelsgestalten sind dem Rand des Bettes näher gerückt.

Entscheide dich, schiesst es ihr durch den Kopf, du musst dich entscheiden! Sie muss nicht überlegen: Sie will die Herausforderung. 
Sie will sich selbst verpflichtet sein, und dem Licht.

Die Lichtgestalten stürzen sich auf die schwarze Brut und in kurzer Zeit ist die Bettdecke von Teufeln befreit.

Sie hat ihren Vertrag geschlossen!

Mit sich selbst, dem inneren Licht.

Sie betritt neues  Land und dennoch ist es vertraut.

In der Nacht erwacht sie und sieht die Engellein sich auf der Decke ausruhen.

Sie begibt sich auf ihren Weg, den schweren, unsichtbaren und leisen einer Lichtträgerin.

Wer sich ins Licht stellt, wird sichtbar - und verletzlich.

Wer Licht ins Dunkel trägt, sollte sich nicht vor dem eigenen Schatten fürchten.



\section*{Feigenbaum II (Landesmuseum Schleswig-Holstein, Schloss Gottorf, Raum der Moorleichen, 1990)}
\addcontentsline{toc}{section}{Feigenbaum II (Landesmuseum Schleswig-Holstein, Schloss Gottorf, Raum der Moorleichen, 1990)}



Da bin ich wieder, denkt Johanna mit Herzklopfen. Sie steht vor dem Raum der Moorleichen, nachdem sie sich ausgiebig bei dem Nydamboot aufgehalten hat. Die Mädels hatten die Idee das Museum zu besuchen und Johanna hat sich vor genommen, dem Schreckgespenst aus der Kindheit erneut ins Auge zu blicken, oder dem, was von dessen Augen übrig ist.

Langsam betritt sie den Raum. Eine Gestalt nach der anderen genau betrachtend, Tafeln studierend\dots  ausser einem leichten Ekelgefühl, spürt sie nichts, was beunruhigend wäre\dots 

Abends im Bett kommen sie. Schwarze, vermoderte mit Pergamenthaut überzogene Schreckensgestalten. Sie stürzen sich auf sie und reissen sie ins Moor. Mehr und mehr, sie heben die vertrockneten Arme und drücken Johanna in den Sumpf.
 
Nicht nur dort, wenn sie im Dunkel abends nach Hause fährt, über den unbeleuchteten Markt- und Jahrmarktsplatz, springen sie aus den Löchern, dem hohen Gras. Während die Beine hektisch in die Pedalen treten, versucht sie verzweifelt die Wahnbilder in den Kopf zurück zu drängen, damit sie auf dem Weg bleibt, nur nicht hinfallen, dann sterbe ich vor Angst\dots 

Sie hat Glück, weil sie nach drei Tagen Qual mit den Eltern und dem kleinen Bruder zu dem Heilpraktiker fährt. Johanna erzählt von beiden Besuchen im Museum und den Folgen. „Verbrennen!“ lautet die Antwort. „Du musst sie in dir verbrennen. Stell dir einen Scheiterhaufen vor und schmeiss sie ins Feuer.“ „Aber, was ist das? Was passiert da?“  „ Die toten Körper ziehen niedere, bösartige Wesen an, die gerne einen Körper wollen, weil sie selbst keinen haben. Sie fallen auch die Lebenden an und, wenn man feinfühlig ist, spürt man das.“

Niedere Wesen? Johanna schaut zurück auf den Raum des Museums und meint Schatten darin umher huschen zu sehen. Schatten, die über den Leichen schwebend hin zu den Lebenden schweben und versuchen sich an ihrer Hülle fest zu saugen, um aus dem Raum hinaus zu gelangen. 

Verbrennen! Abends liegt sie im Bett. Sie muss nicht lange warten, da stürmen sie wieder heran. Aber sie hat vorgesorgt. Einen Raum hat sie sich gebastelt, einen viereckigen Raum aus roten Ziegelsteinen, Stein für Stein hat sie ihn aufgeschichtet. Und mit aller Kraft stopft sie die schwarzen Leichen in den Raum und verschliesst ihn fest mit einem Deckel, der auf die zappelnde Masse kracht, die versucht heraus zu springen. Sie schafft es, sie spürt wie die schwarzen Monster im inneren des Raums zetern und toben. 

Sie sammelt sich und ruft alle Imaginationskräfte zusammen und entfacht mit einem Schlag den Raum mit einem gleissend hellen, explodierenden Feuer. Es kostet Kraft, dieses Feuer soll lange brennen, bis auch der letzte Hautfetzen verbrannt und zu Asche geworden ist. Mit der Zeit wird es still in dem Raum, kein Geist ist mehr zu spüren.

Es hilft, sie muss es eine Zeit lang jeden Abend vor dem Einschlafen machen, aber es hilft und schliesslich scheinen die Leichen auf zu geben und fern zu bleiben.

Was Johanna kein Ruhe lässt, ist, sie spürt tief in sich, weit in den hintersten Winkeln eine Erinnerung vibrieren. 



\section*{La Silfide ( 7.11.1989)}
\addcontentsline{toc}{section}{La Silfide ( 7.11.1989)}


 
Johanna schreibt:

\begin{tg}




\section*{\em Gesang der Geister über dem Wasser}
\addcontentsline{toc}{section}{Gesang der Geister über dem Wasser}


\begin{verse}
Des Menschen Seele\\
Gleicht dem Wasser.\\
Vom Himmel kommt es,\\
Zum Himmel steigt es,\\
Und wieder nieder\\
Zur Erde muss es,\\
Ewig wechselnd.
\end{verse}
\begin{dichter} Goethe \end{dichter}


Heute morgen war mir, als sähe ich leibhaftig die Geister über dem Wasser.
Denn es war schwerer Nebel auf dem Kanal und strahlend blauer, verhangener Himmel. Das Wasser löste sich in den Nebel auf und so schien es, als begänne dieser nur drei Schritte entfernt von mir. Der Nebel tanzte, schwebte und neue gestalten, Figuren bildeten sich vor meinen Augen. Eine wahre Geisterschar zog vorüber und kam, verschwand in dem violetten Schimmer des Himmlischen, das so nahe war. Das Schiff fuhr in den Himmel, die Enten schwammen hinein und vor mir die dunstschweren Wesenheiten, die mir mit feuchten, erfrischenden, kalten Händen ins Gesicht fuhren, mich um schwebten. Und konnte ihnen nur mit dem Herzen folgen, weil mein erdenschwerer Körper den Nebel nicht zu  begehen weiss. Wie zog es mich fort, wie nahe war mir der Himmel, die mystische, lichte, vollkommene Sphäre und ich Mensch stehe mitten in der Materie eine Körpers, muss ihn für das Tun behalten, darf nur mein Herz aus schicken die Fremdlinge zu grüssen Und wie schwer war allein dieser Tanz durch feuchte, winzigste Tropfen vor das Auge hingezogen, schon so schwer, wie schwer erst der Ballast eines Körpers für die freie Seele? (Danke, HERR! Amen!)

11.November 1989

Mir ist aufgefallen, dass die Sonnenstrahlen, die auf das Wasser fallen immer wie ein heller Weg auf den Beschauer gerichtet sind. Fahre ich mit der Fähre über das Wasser, wandern die Strahlen mit meinen Augen über das Wasser mit. Jemand anderes wiederum sieht die Strahlen auf sich zu strömen.Demnach muss das ganze Wasser leuchten. Jeder Mensch sieht aber nur „seinen“ Strahl auf dem Wasser.

HERR!

So zeigst du jedem deine Strahlen und jeder sieht nur seinen eigenen, nie den, dessen, der neben einem steht. Niemand sieht alle Strahlen, die das Wasser zu einem Glitzermeer machen, weil sie viel zu weit für das Auge sind. So sieht jeder Mensch nur sich. Nicht den Weg, die Strahlen des anderen, nie dich selbst im ewigen Licht. Arme Menschen und selbst das Schillern ist ein Spiegelbild!
\end{tg}


\section*{Knoten im Mond I (Januar 1990)}
\addcontentsline{toc}{section}{Knoten im Mond I (Januar 1990)}




Johanna schreibt:
\begin{tg}
6. Oktober 1989:

Als ich heute in der Schule war und aus dem Fenster schaute, überkam mich das Gefühl, dass Thomas und meine Wenigkeit sich begegnen werden. Das Empfinden war stark und eindeutig, dass es mich im selben Augenblick wunderte. Es lag in einiger Ferne, als ob es hinter einem Weg von anderen Ereignissen läge.

17. Dezember 1989:

Thomas hat eine Freundin, eine echte, mit allem drum und dran. Gestern hat es mich kaum berührt. Ich habe deutlich gespürt, dass dies nicht der Mann ist, wegen dem ich in mir ein Leid entfache. Habe mich ganz von ihm zurückgezogen, den Dingen ihren Lauf gelassen, ohne sie zu beachten. Anstatt um Thomas zu „trauern“, habe ich meine freundlichen Empfindungen an all die anderen Menschen gegeben, die ich kannte, und es war ein grosser Trost, diese Menschen zu treffen.

Aber nun bin ich durcheinander. Ehrlich gesagt musste ich mir gestern Abend eingestehen, dass ich in mir den Kern dieses Menschen immer noch bis ins äusserste empfinde. Ich bekomme Angst, verstehst du, Angst vor mir selber. Warum ich mich nicht damit zufrieden geben will, bis jetzt mit „heiler Haut“ davon gekommen zu sein? Ich weiss nicht was ich glauben soll, denn es fühlt sich in mir sicher an, fest, klar und stark, macht es mir nichts?

Ich weiss nicht, ich will stark sein, aber ich möchte nicht alles verdrängen, weil ich denke, dass ich stark bin.

Mir ist Angst und Bange, Johanna, wo willst du noch ganz hin?
Bin ich einfach stur?

20. Januar 1990

Heute ist der Stichtag für meinen Mondknoten, aber mir gefällt der Tag nicht. Ich fühle mich heute morgen erbärmlich und frage mich vergeblich, warum ich in die Schule fahre. Mich vier Stunden bemühe etwas über die Weimarer Republik zu schreiben, was dann wieder nur schlecht ist, bzw. sein soll.

Es ist mir nicht die Stimmung danach, dass etwas Seltsames, Sonderbares, Wichtiges passiert, obwohl ich mir natürlich schon ein wenig Spannung eingeredet habe und mein Gefühl zu hoffen beginnt, dass\dots 

Aber, vielleicht ist es so klein und leise, oder dort, wo ich nicht suche (- denke ja doch nur an Thomas). Finden, nicht mal finden, sondern ereignen! Na, denn!

Es regnet draussen, eben schien noch die Sonne, richtiges Aprilwetter, aber vielleicht ist es auch nur der Wind, der die verschiedenen Wetter so schnell vor sich hertreibt.

Als ich nach der Schule mit Jasmine im Anno zusammen sass, kam Thomas rein gerannt. Er drehte sich um, sah mich und guckte, blieb unschlüssig stehen, rannte vor und zurück, beäugte mich aus der Ferne und blieb zuletzt in der Nähe von uns. Er hatte eine schöne Halskette um, ich glaube, sie war aus Kupfer, sie bestand aus vielen, langen Dreiecken.

Es war schön, dass Thomas sich mal in meine Nähe traute, auch wenn ich ihn kaum ansah, aber warum soll ich ihn ansehen, wenn ich mit Jasmine rede? Er sieht schön aus, ich glaube, es gibt kein Wort für diese Schönheit.
Aber heute hast nicht einmal du mich froh gemacht, Thomas, nicht mal jetzt, wo ich schreibe. Heute ist es unangenehm, irgendwie fühle ich mich stumpf.

21.Januar 1990

Oh, HERR, was war das?

Gerade ergeht mein Kopf sich in wilder Schwärmerei über die Zukunft, ich weiss jetzt, ob es am Mondknoten liegt, dass ich hier in der kleinen Stadt Maler werde mit deiner gütigen Hilfe und deinem Einverständnis, da fliegen zwei Schwäne über meinen Kopf.

Beim Fliegen geben sie einen surrenden Ton von sich. Ihre Hälse sind ganz lang gestreckt, sie bleiben dicht nebeneinander und es schien mir, sie würden sich manchmal kurz ansehen. Grosse weisse Flügel und zwei verletzliche, helle Bäuche und vier schwarze Füsse, die den Vögeln, die doch elegant sind, ein grotesk-menschlichen Zug geben. Wie sie über den Bahndamm flogen, hätte man meinen können, dass sie gleich vom Himmel fielen.

HERR? Vielen Dank, oder ist es nicht wunderlich, dass zwei Schwäne über meinen Kopf fliegen, wenn ich am Simulieren und Festklopfen bin? Sie haben alle Gedanken mit genommen. Wie dumm, sich etwas aus zu denken, denn mein Weg ist doch schon lange beschlossen, ich brauche nur losgehen.

Gestern Abend, als ich durch den Stadtpark nach Hause fuhr, lagen zwei Schwäne dicht aneinander geschmiegt auf der Enteninsel, nur die weissen Leiber waren zu sehen und der ruhige Schlaf darin. Ob es die zwei Schwäne waren, die mir vorhin über den Kopf flogen?

Nicht, dass ich unbescheiden sein will, aber ich dachte, dass um mich herum auch etwas Unterhaltsames passiert, an meinem Mondknoten, anstatt, dass ich in Todessehnsucht verzweifelt um Fassung kämpfe, weil mal wieder alles zusammenstürzt und ich mich wieder mit Träumen überschütte, weil ich nicht weiss, wohin und warum?

(Ich war aber auch dumm, von Thomas zu träumen und zu denken, dass mein Mondknoten ihn was schert, obwohl er es gestern Abend ausnahmsweise mal drei Stunden an einem Platz ausgehalten hat und nicht mal allein weggegangen ist, allerdings ist er drei mal fort gerannt.)

Umso mehr zehrt es an meiner Kraft und der Mut zu einer Entscheidung wird schwächer. Ich habe Angst meine Ideale zu leben, mir fehlt das Vertrauen darauf, dass Du, HERR, mir beistehen wirst, wenn ich nur den mir innigsten Weg begehe.  Ich habe auch Angst auf zu fallen, lang und breit mein Handeln jedem erklären zu müssen und zu hören, dass ich es falsch gemacht habe.

Ich, Johanna, die immer zu auf die Trägheit der heutigen „Gleichschaltung“ schimpft, lasse mit mir das gleiche tun, aus Angst anders und konsequent in mir, meiner Ansicht zu sein!

Ich bin viel schlimmer, als die, die mit ihrem Willen darauf bestehen, sich in die Haut stecken zu lassen, weil ich spüre, dass es mit mir dann zu Ende geht. Ich glaube, diese anderen können auch diese Prüfung machen, weil sie in dieses Siechtum hineingehören und ihren Weg darin finden.

Aber ich kann es nicht, ich gehe, wenn ich recht überlege in die Falle?
Vor allem ab von meinem Weg!

25.Januar 1990

Ich glaube, irgendjemand versucht mich vehement fertig zu machen, Esoteriker würden es „Prüfung“ nennen, dann schleppe ich mich wirklich auf allen Vieren zur Schule und bin in jeder Pause dem endgültigen Zusammenbruch nahe, da muss just am Abend, wo ich mich endlich halbwegs einmal sicher fühle, Thomas seinen Anhang mitbringen, den ich seit vor den Weihnachtsferien nicht mehr gesehen hatte.

OH, HERR im Himmel, wie soll ich das alles durchstehen und gesund an Leib und Seele bleiben? Draussen stürmt es, dass sich die Balken biegen, der Sturm peitscht den kalten Regen durch alle Gassen, genau wie es in mir aussieht. Ich fühle nicht mehr, ich spreche nur über anderes.  Ich fühle nicht mehr, HERR, mein Herz lässt mich davor stehen, weil ich sonst „knacks“ einfach entzwei breche.

Ich habe alles runter gerissen, seit ich Thomas gesehen hatte und er wagt es noch, nach mir zu schauen – natürlich gerade, als ich an einem Stuhl fest hänge.

Ich glaube, ich kann nicht mehr!

Ich bin  nur ein kleiner Mensch, den allein tiefe Empfindungen zerfressen und jetzt reicht es mir!

Tod! - Tot, tot, tot!

HERR, werde ich das alles überstehen?

- Wo ich nicht mal eine Stimme habe, die schreien kann, selbst, wenn das Innere schreit und schreit!
\end{tg}




 
 
 \end{document}